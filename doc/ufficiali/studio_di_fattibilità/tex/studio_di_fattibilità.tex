\documentclass[11pt,a4paper]{article}
\usepackage{amsmath}
\usepackage{amsfonts}
\usepackage{amssymb}
\usepackage{fancyhdr}
\usepackage{lastpage}
\usepackage{graphicx}
\usepackage{ucs}
\usepackage[utf8x]{inputenc}
\usepackage[greek, italian]{babel}
%\usepackage{eurosym}

\renewcommand{\headrulewidth}{0.6pt}
\renewcommand{\footrulewidth}{0.6pt}
% impostazione dello stile per le pagine interne del documento
\lhead{\leftmark}
\chead{}
\rhead{\includegraphics[scale=0.15]{logo.png} }
\lfoot{Studio di fattibilità v1.1.1}
\cfoot{}
\rfoot{\thepage \ di \pageref{LastPage}}
% ridefinizione dello stile plain per il frontespizio
\fancypagestyle{plain}{
\fancyhf
}
% impostazione dello stile per l'indice
\fancypagestyle{indice}{
\lhead{\leftmark}
\chead{}
\rhead{\includegraphics[scale=0.15]{logo.png}}
\lfoot{Studio di fattibilità v1.1.1}
\cfoot{}
\rfoot{}
}
\headheight = 46pt
%definizione del comando "\modifiche" per la creazione del diario delle modifiche
\newcommand{\modifiche} 
{
\newpage
\begin{center}
\textbf{Diario delle modifiche} \\
\bigskip
\begin{tabular}{|c|c|p{0.62\textwidth}|}
\hline
\textsc{Data} & \textsc{Versione} & \textsc{Modifica} \\
\hline
\hline
\textit{21-01-2009} & 1.1.1 & Aggiunta della sezione ''Valutazione dei capitolati d'appalto'' e correzioni generali  \\
\hline
\textit{09-12-2008} & 1.0.0 & Approvazione del Responsabile e passaggio di stato in ''Formale''\\
\hline
\textit{08-12-2008} & 0.1.1 & Correzioni varie  \\
\hline
\textit{17-11-2008} & 0.1.0 & Stesura studio di fattibilità \\
\hline
\end{tabular}
\end{center}
}
%definizione del comando "\info" per la creazione delle informazioni del documento
\newcommand{\info} {
\bigskip
\begin{tabbing}
	\hspace*{0.3\textwidth} \= \hspace*{0.5\textwidth} \kill
	\parbox{0.3\textwidth}{\textbf{Verifica: }} \> \parbox{0.5\textwidth}{Freo Matteo} \\
	\parbox{0.3\textwidth}{\textbf{Approvazione: }} \> \parbox{0.5\textwidth}{Scortegagna Carlo} \\
	\parbox{0.3\textwidth}{\textbf{Stato: }} \> \parbox{0.5\textwidth}{Formale} \\
	\parbox{0.3\textwidth}{\textbf{Uso: }} \> \parbox{0.5\textwidth}{Interno} \\
	\parbox{0.3\textwidth}{\textbf{Distribuzione: }} \> \parbox{0.5\textwidth}{QuiXoft} \\
\end{tabbing}
}
%definizione del comando "\frontespizio" per la creazione del frontespizio
\newcommand{\frontespizio} {
\thispagestyle{plain}
\title{\begin{Huge}\textsc{Progetto SIGEOL}\end{Huge} \\ \textit{Studio di fattibilità \\ v1.1.1}}
\author{Redazione: Grosselle Alessandro, Scarpa Davide }
\maketitle
\medskip
\begin{center}
\includegraphics[scale=0.5]{logo.png} \\
\textit{quixoft.sol@gmail.com}
\end{center}
\medskip
\info
\begin{center}
\textbf{Sommario} \\
Documento contenente lo studio di fattibilità per il progetto \textit{SIGEOL} commissionato dalla prof. Rossi Francesca.
\end{center}
\newpage
}
%definizione del comando "\indice" per la creazione dell'indice
\newcommand{\indice} {
\thispagestyle{indice}
\tableofcontents
\newpage
}
\pagestyle{fancy}
\begin{document}
\frontespizio
\indice
\setcounter{page}{1}
\section{Introduzione}
\subsection{Scopo del documento}
Il presente documento è un report che analizza in breve i capitolati d'appalto presentati dai committenti e stabilisce l'opportunità o meno di procedere allo sviluppo del sistema software scelto dal gruppo.
\subsection{Glossario}
Le definizioni dei termini specialistici usati nella stesura di questo e di tutti gli altri documenti possono essere trovate nel documento \textsc{Glossario} al fine di eliminare ogni ambiguità  e di facilitare la comprensione dei temi trattati. Ogni termine la cui definizione è disponibile all'interno del glossario verrà  marcato con una \underline{sottolineatura}.
\subsection{Riferimenti}
\begin{itemize}
\item Capitolato d'appalto reperibile all'indirizzo: \\ \textit{http://www.math.unipd.it/~tullio/IS-1/2008/Progetti/SIGEOL.html}
\end{itemize}
\section{Valutazione dei capitolati d'appalto}
In data 12 Novembre 2008 vengono diffusi cinque capitolati d'appalto, consultabili sul sito internet del corso di Ingegneria del Software all'indirizzo \\ \textit{http://www.math.unipd.it/~tullio/IS-1/2008/Progetti/Capitolati.html}.
Il team QuiXoft ha analizzato aspetti positivi e negativi per ogni capitolato.

\subsection{C01: SIGEOL}
Il capitolato C01 denominato ''SIGEOL'' si prefigge di automatizzare la generazione, la gestione, l'ottimizzazione e la consultazione degli orari di lezione.

\subsubsection{Aspetti positivi}

\begin{itemize}
 \item Il team reputa l'argomento del capitolato d'appalto interessante
 \item Previsto un utilizzo ampio di tecnologie (sopratutto web) già viste
 \item Soddisfazione personale dei membri del team nel vedere il software utilizzato nell'ambiente universitario
 \item Capitolato d'appalto ritenuto chiaro
\end{itemize}

\subsection{C02: AUTOMATICEXPLORER}
Il capitolato C02, denominato ''AUTOMATICEXPLORER'', si prefigge la realizzazione di un sistema software per simulare esplorazioni di superfici planetarie.

\subsubsection{Aspetti positivi}

\begin{itemize}
 \item Il team giudica l'argomento del capitolato d'appalto interessante
\end{itemize}

\subsubsection{Aspetti negativi}

\begin{itemize}
 \item Non previsto l'utilizzo di tecnologie web
\end{itemize}

\subsection{C03: SQL\_Plug-out}
Il capitolato C03, denominato ''SQL\_Plug-out'', si prefigge lo sviluppo di un sistema software per la generazione di codice dichiarativo SQL da un diagramma delle classi.

\subsubsection{Aspetti negativi}

\begin{itemize}
 \item Il team reputa l'argomento del capitolato poco stimolante, troppo tecnico e limitato al linguaggio SQL
 \item Il capitolato d'appalto è ritenuto poco esplicativo
\end{itemize}

\subsection{C04: AJAXDRAW}
Il capitolato C04, denominato ''AJAXDRAW'', si prefigge la realizzazione di un sistema software per il disegno grafico in ambito web tramite browser.

\subsubsection{Aspetti positivi}

\begin{itemize}
 \item Il team giudica l'argomento del capitolato d'appalto interessante
 \item Previsto l'utilizzo di tecnologie web
\end{itemize}

\subsection{C05: AEG}
Il capitolato C05, denominato ''AEG'', si prefigge lo sviluppo di un sistema software per la gestione delle risorse umane nell'ambito aziendale.

\subsubsection{Aspetti positivi}

\begin{itemize}
 \item Il team reputa l'argomento del capitolato d'appalto interessante
\end{itemize}

\subsubsection{Aspetti negativi}

\begin{itemize}
 \item Il team giudica il capitolato d'appalto poco chiaro
\end{itemize}

\section{Studio di fattibilità}
Analizzando gli aspetti positivi e negativi sopra descritti il team QuiXoft ha deciso di proporsi per la realizzazione del progetto denominato ''SIGEOL''. In particolare nella scelta finale tra il capitolato ''SIGEOL'' e il capitolato ''AJAXDRAW'' ha pesato molto la soddisfazione personale del team nel caso in cui il software realizzato venga utilizzato nell'ambiente universitario.

Dal capitolato d'appalto si può evincere che il sistema dovrà essere realizzato tramite  \underline{servizio web}, accessibile da ogni postazione dotata di connessione Internet e browser.
Tutti i componenti del team possiedono buone conoscenze di tecnologie web come html e css.
In particolar modo alcuni hanno anche esperienza lavorative nello sviluppo e progettazione di \underline{software} e applicazioni web.

Per la realizzazione di tale progetto è necessario avere anche conoscenze di utilizzo e gestione di \underline{database}: tutti i membri del team QuiXoft possiedono tali conoscenze, dimostrate anche dal pieno superamento degli esami di Basi di Dati e Sistemi Informativi 1 e 2.

Il team, non essendo impegnato in nessun altro progetto, si dichiara quindi disponibile allo sviluppo del prodotto ''SIGEOL'' cercando di contenere i costi, ben sapendo che il committente non accetterà offerte con costi inferiori a 13000\textgreek{\euro}.

Dall'analisi di quanto sopra affermato e delle capacità tecniche dei membri del team QuiXoft, si può dedurre che tale capitolato verrà sviluppato con cura e dedizione, realizzando un prodotto di qualità che soddisfi i requisiti del committente.

Sarà ulteriore motivo di soddisfazione il vedere il nostro software, dopo la consegna del prodotto finale, adottato all'interno dell'infrastruttura universitaria: i componenti  del team si impegneranno al massimo per raggiungere tale obiettivo.
\modifiche
\end{document}