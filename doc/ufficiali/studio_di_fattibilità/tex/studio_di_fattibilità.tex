\documentclass[11pt,a4paper]{article}
\usepackage{amsmath}
\usepackage{amsfonts}
\usepackage{amssymb}
\usepackage{fancyhdr}
\usepackage{lastpage}
\usepackage{graphicx}
\usepackage{ucs}
\usepackage[utf8x]{inputenc}
\usepackage[italian]{babel}
\usepackage{eurosym}

\renewcommand{\headrulewidth}{0.6pt}
\renewcommand{\footrulewidth}{0.6pt}
% impostazione dello stile per le pagine interne del documento
\lhead{\leftmark}
\chead{}
\rhead{\includegraphics[scale=0.15]{logo.png} }
\lfoot{Studio di fattibilità v0.1.1}
\cfoot{}
\rfoot{\thepage \ di \pageref{LastPage}}
% ridefinizione dello stile plain per il frontespizio
\fancypagestyle{plain}{
\fancyhf
}
% impostazione dello stile per l'indice
\fancypagestyle{indice}{
\lhead{\leftmark}
\chead{}
\rhead{\includegraphics[scale=0.15]{logo.png}}
\lfoot{Studio di fattibilità v0.1.1}
\cfoot{}
\rfoot{}
}
\headheight = 46pt
%definizione del comando "\modifiche" per la creazione del diario delle modifiche
\newcommand{\modifiche} 
{
\newpage
\begin{center}
\textbf{Diario delle modifiche} \\
\bigskip
\begin{tabular}{|c|c|p{0.62\textwidth}|}
\hline
\textsc{Data} & \textsc{Versione} & \textsc{Modifica} \\
\hline
\hline
\textit{08-12-2008} & 0.1.1 & Correzioni varie  \\
\hline
\textit{17-11-2008} & 0.1.0 & Stesura studio di fattibilità \\
\hline
\end{tabular}
\end{center}
}
%definizione del comando "\info" per la creazione delle informazioni del documento
\newcommand{\info} {
\bigskip
\begin{tabbing}
	\hspace*{0.3\textwidth} \= \hspace*{0.5\textwidth} \kill
	\parbox{0.3\textwidth}{\textbf{Verifica: }} \> \parbox{0.5\textwidth}{Freo Matteo} \\
	\parbox{0.3\textwidth}{\textbf{Approvazione: }} \> \parbox{0.5\textwidth}{Scortegagna Carlo} \\
	\parbox{0.3\textwidth}{\textbf{Stato: }} \> \parbox{0.5\textwidth}{Formale} \\
	\parbox{0.3\textwidth}{\textbf{Uso: }} \> \parbox{0.5\textwidth}{Interno} \\
	\parbox{0.3\textwidth}{\textbf{Distribuzione: }} \> \parbox{0.5\textwidth}{QuiXoft} \\
\end{tabbing}
}
%definizione del comando "\frontespizio" per la creazione del frontespizio
\newcommand{\frontespizio} {
\thispagestyle{plain}
\title{\begin{Huge}\textsc{Progetto SIGEOL}\end{Huge} \\ \textit{Studio di fattibilità \\ v0.1.1}}
\author{Redazione: Grosselle Alessandro }
\maketitle
\medskip
\begin{center}
\includegraphics[scale=0.5]{logo.png} \\
\textit{quixoft.sol@gmail.com}
\end{center}
\medskip
\info
\begin{center}
\textbf{Sommario} \\
Documento contenente lo studio di fattibilità per il progetto \textit{SIGEOL} commissionato dalla prof. Rossi Francesca.
\end{center}
\newpage
}
%definizione del comando "\indice" per la creazione dell'indice
\newcommand{\indice} {
\thispagestyle{indice}
\tableofcontents
\newpage
}
\pagestyle{fancy}
\begin{document}
\frontespizio
\indice
\setcounter{page}{1}
\section{Introduzione}
\subsection{Scopo del documento}
Il presente documento è un report che stabilisce l'opportunità o meno di procedere allo sviluppo del sistema software denominato ''SIGEOL''.
\subsection{Glossario}
Le definizioni dei termini specialistici usati nella stesura di questo e di tutti gli altri documenti possono essere trovate nel documento ''Glossario'' al fine di eliminare ogni ambiguità  e di facilitare la comprensione dei temi trattati. Ogni termine la cui definizione è disponibile all'interno del glossario verrà  marcato con una \underline{sottolineatura}.
\subsection{Riferimenti}
\begin{itemize}
\item Capitolato d'appalto reperibile all'indirizzo: \\ \textit{http://www.math.unipd.it/~tullio/IS-1/2008/Progetti/SIGEOL.html}
\end{itemize}
\section{Studio di fattibilità}
In data 12 Novembre 2008 vengono diffusi cinque capitolati d'appalto, consultabili sul sito internet del corso di Ingegnerie del Software all'indirizzo \\ \textit{http://www.math.unipd.it/~tullio/IS-1/2008/Progetti/Capitolati.html}.

Il team QuiXoft si propone come possibile fornitore per il C01, denomi\-nato ''SIGEOL'': tale progetto si prefigge di automatizzare la generazione, la gestione, l'ottimizzazione e la consultazione degli orari di lezione.

Dal capitolato d'appalto si può intuire che il sistema dovrà essere un \underline{servizio web} accessibile tramite rete internet.
Tutti i componenti del team possiedono buone conoscenze di tecnologie web come html e css.
In particolar modo alcuni hanno anche esperienza lavorative nello sviluppo e progettazione di \underline{software} e applicazioni web.

Per la realizzazione di tale progetto è necessario avere anche conoscenze di utilizzo e gestione di \underline{database}: tutti i membri del team QuiXoft possiedono tali conoscenze, dimostrate anche dal pieno superamento degli esami di Basi dati e Sistemi informativi 1 e 2.

Il team, non essendo impegnato in nessun altro progetto, si dichiara disponibile allo sviluppo del prodotto ''SIGEOL'' cercando di non superare esageratamente i 13000\euro \space di tetto minimo.

Dall'analisi di quanto sopra affermato e delle capacità tecniche dei membri del team QuiXoft, si può dedurre che tale capitolato verrà sviluppato con cura e dedizione, realizzando un prodotto di qualità che soddisfi i requisiti del committente.

Sarà ulteriore motivo di soddisfazione il vedere il nostro software, dopo la consegna del prodotto finale, adottato all'interno dell'infrastruttura universitaria: i componenti  del team si impegneranno al massimo per raggiungere tale obiettivo.
\modifiche
\end{document}