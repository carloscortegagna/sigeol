\documentclass[11pt,a4paper]{article}
\usepackage{amsmath}
\usepackage{amsfonts}
\usepackage{amssymb}
\usepackage{fancyhdr}
\usepackage{lastpage}
\usepackage{graphicx}
\usepackage{ucs}
\usepackage[utf8x]{inputenc}
\usepackage[italian]{babel}
\usepackage[colorlinks=true,linkcolor=black]{hyperref}

\renewcommand{\headrulewidth}{0.6pt}
\renewcommand{\footrulewidth}{0.6pt}
% impostazione dello stile per le pagine interne del documento
\lhead{\leftmark}
\chead{}
\rhead{\includegraphics[scale=0.15]{logo.png} }
\lfoot{Manuale Installazione v1.0.0}
\cfoot{}
\rfoot{\thepage \ di \pageref{LastPage}}
% ridefinizione dello stile plain per il frontespizio
\fancypagestyle{plain}{
\fancyhf
}
% impostazione dello stile per l'indice
\fancypagestyle{indice}{
\lhead{\leftmark}
\chead{}
\rhead{\includegraphics[scale=0.15]{logo.png}}
\lfoot{Manuale Installazione v1.0.0}
\cfoot{}
\rfoot{}
}
\headheight = 46pt
%definizione del comando "\modfiche" per la creazione del diario delle modifiche
\newcommand{\modifiche} 
{
\newpage
\begin{center}
\textbf{Diario delle modifiche} \\
\bigskip
\begin{tabular}{|c|c|p{0.62\textwidth}|}
\hline
\textsc{Data} & \textsc{Versione} & \textsc{Modifica} \\
\hline
\hline
\textit{15-06-2009} & 1.0.0 & Approvazione del responsabile\\
\hline
\textit{14-06-2009} & 0.2.0 & Stesura manuale\\
\hline
\textit{19-05-2009} & 0.1.0 & Stesura indice\\
\hline
\end{tabular}
\end{center}
}
%definizione del comando "\info" per la creazione delle informazioni del documento
\newcommand{\info} {
\bigskip
\begin{tabbing}
	\hspace*{0.3\textwidth} \= \hspace*{0.5\textwidth} \kill
	\parbox{0.3\textwidth}{\textbf{Verifica: }} \> \parbox{0.5\textwidth}{Alberti Andrea} \\
	\parbox{0.3\textwidth}{\textbf{Approvazione: }} \> \parbox{0.5\textwidth}{Beggiato Andrea} \\
	\parbox{0.3\textwidth}{\textbf{Stato: }} \> \parbox{0.5\textwidth}{Formale} \\
	\parbox{0.3\textwidth}{\textbf{Uso: }} \> \parbox{0.5\textwidth}{Esterno} \\
	\parbox{0.3\textwidth}{\textbf{Distribuzione: }} \> \parbox{0.5\textwidth}{QuiXoft} \\
							\> \parbox{0.5\textwidth}{Rossi Francesca} \\
							\> \parbox{0.5\textwidth}{Vardanega Tullio} \\
							\> \parbox{0.5\textwidth}{Conte Renato} \\
\end{tabbing}
}
%definizione del comando "\frontespizio" per la creazione del frontespizio
\newcommand{\frontespizio} {
\thispagestyle{plain}
\title{\begin{Huge}\textsc{Progetto SIGEOL}\end{Huge} \\ \textit{Manuale Installazione\\ v1.0.0}}
\author{Redazione: Barbiero Mattia}
\maketitle
\medskip
\begin{center}
\includegraphics[scale=0.5]{logo.png} \\
\textit{quixoft.sol@gmail.com} \\
\end{center}
\medskip
\info
\begin{center}
\textbf{Sommario} \\
Manuale per l'installazione del prodotto \textit{SIGEOL} , contenente la spiegazione dell'installazione.
\end{center}
\newpage
}
%definizione del comando "\indice" per la creazione dell'indice
\newcommand{\indice} {
\thispagestyle{indice}
\tableofcontents
\newpage
}
\pagestyle{fancy}
\begin{document}
\frontespizio
\indice
\setcounter{page}{1}
\section{Premessa}
Il seguente manuale definisce le modalità d'installazione del prodotto.
Sono previste 2 modalità d'installazione del prodotto:
\begin{itemize}
 \item su singolo \underline{web server} 
\item su due \underline{web server}: 
\begin{enumerate}
 \item \underline{web server} contenente la parte amministrativa dell'applicazione (Sigeol)
 \item \underline{web server} utilizzato per il calcolo e generazione degli orari (moduli Middleman e Algorithm)

in questa modalità l'utilizzo dei database è a scelta tra:
\begin{description}
 \item[default] unico database sul \underline{server} Sigeol
 \item[distribuita] due database risiedenti uno sul \underline{server} Sigeol e l'altro sul \underline{server} MiddleMan 
\end{description}
\end{enumerate}
\end{itemize}

E' altamente consigliato l'utilizzo del Sun GlassFish Enterprise Server (v3) in quanto il prodotto 
è stato testato su di esso e garantisce i seguenti vantaggi nell'esecuzione di applicazioni Rails rispetto a un qualsiasi altro \underline{web server}:
\begin{itemize}
 \item fornisce un semplice ambiente di distribuzione intregrato 
 \item permette la distribuzione di applicazioni Rails multiple in una singola istanza di Glassfish
 \item permette alle applicazioni Rails di gestire richieste multiple
\end{itemize}

Per maggiori dettagli vedere:

 \href{http://developers.sun.com/appserver/reference/techart/rails_gf/#advantages}{Advantages of JRuby-on-Rails with the GlassFish Application Server}

\newpage


\section{Introduzione}
\subsection{Come leggere il manuale}
Il presente manuale descrive brevemente la parte relativa all'installazione dell'applicazione.

Saranno illustrate e descritte le principali modalità di installazione del prodotto sul \underline{web server} Glassfish.

Il manuale termina con un Glossario, contenente la spiegazione di alcuni termini usati nel corso di questo documento.
I termini che possiedono una descrizione all'interno del Glossario saranno riconoscibili perchè presentano una \underline{sottolineatura}.

\subsection{Come riportare problemi e malfunzionamenti}
La segnalazione di problemi o manfulzionamenti del sistema SIGEOL andrà fatta inviando un email all'indirizzo \textit{quixoft.sol@gmail.com}.
Quest'ultima dovrà contenere le seguenti informazioni:
\begin{itemize}
 \item Nome e cognome del mittente
 \item Data e ora in cui il problema si è manifestato
 \item Tipo d'utenza
 \item Informazioni sull'ambiente in cui è stato rilevato l'errore (sistema operativo, \underline{browser}, ecc...) o qualsiasi altra informazione d'utilità ritenuta importante dal mittente (configurazione hardware, risoluzione dello schermo, ecc...)
 \item Descrizione del malfunzionamento riscontrato, dei messaggi d'errore visualizzati, delle eventuali operazioni svolte prima del manifestarsi del problema.
\end{itemize}
Le segnalazioni saranno prese in considerazione il prima possibile e i problemi riscontrati saranno risolti al più presto dai membri del team QuiXoft.
\section{Requisiti richiesti}
\begin{itemize}
 \item J2EE \underline{web server} (consigliato Glassfish v3)
 \item MySql (5.xx)
 \item JRuby 1.1.5+ 
\end{itemize}
\bigskip
\section{Installazione da pacchetto Glassfish\_Sigeol.zip}
Estrarre il contenuto del file compresso nella cartella di destinazione scelta.\\
Assicurarsi di avere il server MySql in esecuzione e correttamente configurato.\\
Eseguire il file install.sh da terminale per far partire l'installazione del \underline{web server} con l'applicazione

\section{Installazione da pacchetto Sigeol+Middleman.zip}
\subsection{Installazione veloce di Glassfish v3}
\begin{enumerate}
 \item Seleziona la directory dove installare Glassfish
\begin{itemize}
 \item Usa la directory di default.

Se non specifichi una directory d'installazione, il software verrà installato nelle seguenti directory:
\begin{description}
 \item[SolarisTM systems, Linux, o MacOS X systems] 

user-home-directory/glassfishv3-prelude
\item[Windows systems] 

C:\\glassfish-v3-prelude
 \end{description}
\item Prepara la directory per l'installazione di Glassfish

In questo documento, la directory che verrà scelta per l'installazione sarà considerata \textit{as-install}.
\end{itemize}
\item Vai alla pagina di \href{http://www.sun.com/software/products/glassfishv3_prelude/get.jsp}{Download}  di Glassfish v3 Prelude.
\item Click sul link Download. 

Verrà visualizzata la pagina relativa a Glassfish.
\item Seleziona le opzioni di download
\begin{itemize}
 \item da Platform selezionare \textbf{Multi-Platform Zip}
 \item da Language selezionare la propria lingua
 \item selezionare la casella relativa alla \textbf{licenza}
\end{itemize}
\item Click su \textbf{Continue}.
\item Seleziona la casella \textbf{Sun GlassFish Enterprise Server v3 Prelude}.
\item Click sul link \textbf{glassfish-v3-prelude.zip}.

Il file glassfish-v3-prelude.zip verrà copiato nella directory d'installazione.
\item Usa il comando \textbf{cd} per cambiare la directory d'installazione.

cd as-install
\item Decomprimi il pacchetto.

unzip gfv3-prelude.zip

La distribuzione v3 Prelude è installata in una nuova directory glassfishv3-prelude sotto la tua directory d'installazione corrente.
\end{enumerate}

Note – L'installazione dell'Enterprise Server richiede:
JDK 5 disponibile al \url{http://java.sun.com/javase/downloads/index_jdk5.jsp} o 
JDK 6 disponibile al \url{http://java.sun.com/javase/downloads/index.jsp}. 

\bigskip
Per maggiori informazioni sull'installazione di Glassfish vedere : 

\url{http://docs.sun.com/app/docs/doc/820-5968}


\subsection{Installazione di JRuby}

L'installazione di JRuby viene effettuata dalla GlassFish Update Center GUI.
E' possibile vedere un dettagliato screencast sull'installazione all'indirizzo  \url{http://blogs.sun.com/arungupta/entry/screencast_web_9_jruby_on}, in alternativa visitare l'indirizzo \url{http://wiki.updatecenter.java.net/Wiki.jsp?page=GettingStarted}.

\newpage
\section{Configurazione dell'applicazione Sigeol}
Come già detto in precedenza sono previste più modalità di installazione dell'applicazione.
Ora verranno illustrate le configurazioni per ogni modalità di utilizzo.
\subsection{Modalita default}
L'applicazione verrà installata su singolo \underline{server}.
\subsubsection{Installare l'applicazione Sigeol}
In questa sezione verranno illustrate due diverse modalità di pubblicazione dell'applicazione nel \underline{web server} precedentemente installato.
\begin{description}
 \item[Creazione dominio per l'applicazione] 
Con la creazione di un dominio verrà preparato lo spazio necessario per l'\underline{hosting} dell'applicazione sigeol.

In questo documento, il nome del dominio che verrà scelto per l'installazione sarà considerata \textit{as-domain}.

Eseguire da terminale:\\
\verb|[as-install/bin/asadmin create-domain --adminport 4848 as-domain|
Estrarre il pacchetto Sigeol\_Middleman.zip in \url{as-install/domains/as-domain/applications} 

Eseguire da terminale:\\
\verb|[as-install/bin/asadmin start-domain as-domain|\\
\verb|[as-install/bin/asadmin deploy --path as-install/domains/as-domain/applications/sigeol|

\item[\underline{Deploying} automatico dell'applicazione]
Per utilizzare il \underline{deployment} automatico, estrarre il pacchetto Sigeol\_Middleman.zip  in \url{as-install/domains/as-domain/autodeploy} 

Glassfish automaticamente rileverà ed eseguirà l'applicazione Sigeol.
\end{description}

\subsubsection{Configurazione applicativo Sigeol}

Nel file \url{sigeol/config/config.yml} modificare le impostazioni relative alla configurazione dell'email per l'inoltro dei messaggi ed i dettagli per la comunicazione con la servlet:\\
\verb|mail:|\\
\verb|  tls: | modalità tls (true o false)\\
\verb|  address: | indirizzo del server smtp (esempio: smtp.google.com)\\
\verb|  port: | porta del server smtp (esempio: 578)\\
\verb|  domain: | dominio del server smtp (esempio: sigeol.com)\\
\verb|  authentication: | tipo di autenticazione (esempio: plain)\\
\verb|  user_name: | nome utente di accesso alla casella di posta\\
\verb|  password: | password di accesso alla casella di posta\\
\verb|  host: | url base del web server (esempio: www.sigeol.it)\\

\verb|servlet:|
\verb| address: | url della servlet (esempio: \url{http://www.sigeol.it/middleman/scheduler.do}
\verb| timeout: | tempo massimo di generazione degli orari in secondi (esempio: 1800)

\subsubsection{Configurazione database}
Nel file \url{sigeol/config/database.yml} modificare le impostazioni relative al database utilizzato dall'applicazione Sigeol come riportato in seguito:\\
\verb|production:| \\
\verb|  adapter: jdbcmysql |\\
\verb|  database: | nome database Sigeol (es: \verb|sigeol_development|)\\
\verb|  encoding: utf8 |\\
\verb|  username: |username di accesso al database\\
\verb|  password: |password di accesso al database\\
\verb|  host: |host contenente il database (es: \verb|localhost|)
\\
Assicurarsi di aver installato Jruby e di aver il server MySql in esecuzione e correttamente configurato.\\
Copiare il file install\_standalone.sh nella cartella contenente il server ed eseguirlo

Modificare poi le seguenti righe nel file \url{sigeol/WEB-INF/classes/sigeol_quartz.properties} dove sono presenti le impostazioni per il modulo MiddleMan:\\
\verb|org.quartz.dataSource.myDS.URL = | path relativo al database MiddleMan (es: \verb|jdbc:mysql://localhost:3306/sigeol_scheduler|)\\
\verb|org.quartz.dataSource.myDS.user = |username di accesso al database\\
\verb|org.quartz.dataSource.myDS.password = |password di accesso al database\\

(nel caso in cui l'applicazione Sigeol e il modulo MiddleMan condividano lo stesso database, i valori dei rispettivi campi saranno gli stessi)

\subsubsection{Configurazione modulo MiddleMan}
Modificare le cartelle di destinazione dei file generati dall'algoritmo presenti nel file \url{sigeol/WEB-INF/web.xml}:
\verb|<param-name>input-itc-path</param-name>|\\
\verb|<param-value>|path cartella contentente file di input (es: \url{/WEB-INF/itc/input/})\verb|</param-value>|\\
\verb|<param-name>output-itc-path</param-name>|\\
\verb|<param-value>|path cartella contentente file di output (es: \url{/WEB-INF/itc/output/})\verb|</param-value>|\\
Importante!\\
Assicurarsi che le cartelle abbiano i permessi in lettura e scrittura per l'applicazione Sigeol.

\subsection{Modalità distribuita}
\subsubsection{Installare l'applicazione Sigeol e Middleman}
In questa sezione verranno illustrate due diverse modalità di pubblicazione dell'applicazione nei \underline{web server} precedentemente installato (Sigeol e Middleman).
\begin{description}
 \item[Creazione dominio per l'applicazione]
Con la creazione di un dominio verrà preparato lo spazio necessario per l'\underline{hosting} dell'applicazione sigeol.

In questo documento, il nome del dominio che verrà scelto per l'installazione sarà considerata \textit{as-domain}.
Le seguenti procedure dovranno essere eseguite su entrambi i \textit{server} utilizzati.

Eseguire da terminale:\\
\verb|[as-install/bin/asadmin create-domain --adminport 4848 as-domain|
Estrarre il pacchetto Sigeol.zip e Middleman.zip in \url{as-install/domains/as-domain/applications} dei rispettivi \textit{server} 

Eseguire da terminale:\\
\verb|[as-install/bin/asadmin start-domain as-domain|\\
\verb|[as-install/bin/asadmin deploy --path as-install/domains/as-domain/applications/sigeol|

\item[\underline{Deploying} automatico dell'applicazione]
Per utilizzare il \underline{deployment} automatico, estrarre il pacchetto Sigeol.zip e Middleman.zip  in \url{as-install/domains/as-domain/autodeploy} dei rispettivi \textit{server}. 

Glassfish automaticamente rileverà ed eseguirà l'applicazione Sigeol e Middleman.
\end{description}

\subsubsection{Configurazione database \underline{server} Sigeol}
Nel file \url{sigeol/config/database.yml} modificare le impostazioni relative al database utilizzato dall'applicazione Sigeol come riportato in seguito:\\
\verb|production:| \\
\verb|  adapter: jdbcmysql |\\
\verb|  database: | nome database Sigeol (es: \verb|sigeol_development|)\\
\verb|  encoding: utf8 |\\
\verb|  username: |username di accesso al database\\
\verb|  password: |password di accesso al database\\
\verb|  host: |host contenente il database (es: \verb|localhost|)

\subsubsection{Configurazione applicativo Sigeol}

Nel file \url{sigeol/config/config.yml} modificare le impostazioni relative alla configurazione dell'email per l'inoltro dei messaggi ed i dettagli per la comunicazione con la servlet:\\
\verb|mail:|\\
\verb|  tls: | modalità tls (true o false)\\
\verb|  address: | indirizzo del server smtp (esempio: smtp.google.com)\\
\verb|  port: | porta del server smtp (esempio: 578)\\
\verb|  domain: | dominio del server smtp (esempio: sigeol.com)\\
\verb|  authentication: | tipo di autenticazione (esempio: plain)\\
\verb|  user_name: | nome utente di accesso alla casella di posta\\
\verb|  password: | password di accesso alla casella di posta\\
\verb|  host: | url base del web server (esempio: www.sigeol.it)\\

\verb|servlet:|
\verb| address: | url della servlet (esempio: \url{http://www.sigeol.it/middleman/scheduler.do}
\verb| timeout: | tempo massimo di generazione degli orari in secondi (esempio: 1800)

Copiare il file install\_sigeol.sh nella cartella contenente il server ed eseguirlo.



\subsubsection{Configurazione database \underline{server} MiddleMan}

Copiare il file install\_middleman.sh nella cartella contenente il server ed eseguirlo.

Modificare poi le seguenti righe nel file \url{sigeol/WEB-INF/classes/sigeol_quartz.properties} dove sono presenti le impostazioni per il modulo MiddleMan:\\
\verb|org.quartz.dataSource.myDS.URL = | path relativo al database MiddleMan (es: \verb|jdbc:mysql://localhost:3306/sigeol_scheduler|)\\
\verb|org.quartz.dataSource.myDS.user = |username di accesso al database\\
\verb|org.quartz.dataSource.myDS.password = |password di accesso al database\\

\subsubsection{Configurazione modulo MiddleMan}
Infine modificare le cartelle di destinazione dei file generati dall'algoritmo presenti nel file \url{sigeol/WEB-INF/web.xml}:\\
\verb|<param-name>input-itc-path</param-name>|\\
\verb|<param-value>|path cartella contentente file di input (es: \url{/WEB-INF/itc/input/})\verb|</param-value>|\\
\verb|<param-name>output-itc-path</param-name>|\\
\verb|<param-value>|path cartella contentente file di output (es: \url{/WEB-INF/itc/output/})\verb|</param-value>|\\
Importante!\\
Assicurarsi che le cartelle abbiano i permessi in lettura e scrittura per l'applicazione Sigeol.

Inoltre modificare l'url che indirizza all'applicazione Sigeol:
\verb|<param-name>url-client</param-name>|\\
\verb|<param-value>|URL applicazione Sigeol (es: \url{http://www.math.unipd.it:8080/sigeol} \verb|</param-value>|\\

\subsection{Verifica installazione}
Per assicurarsi di aver eseguito l'installazione correttamente verificare che i seguenti indirizzi siano accessibili e producano una pagina relativa al sito:
\url{http://<nome host>:8080/sigeol/timetables}\\
\url{http://<nome host>:8080/middleman/scheduler.do}
\section{Appendice}
\subsection{Glossario}
\flushleft 
\Huge B \bigskip
\hrule
\smallskip
\normalsize
\begin{description}
	\item[Browser:] software che consente agli utenti di visualizzare e interagire con testi, immagini e altre informazioni, tipicamente contenute in una pagina web di un sito. Il browser è in grado di interpretare il codice HTML (e più recentemente XHTML) e visualizzarlo in forma di ipertesto. L'HTML è il codice col quale la maggioranza delle pagine web nel mondo sono composte: il web browser consente perciò la navigazione nel web.
\end{description}
\bigskip
\Huge D \bigskip
\hrule
\smallskip
\normalsize
\begin{description}
	\item[Deployment:] messa in opera, in funzione, di un'applicazione, dopo la fase di programmazione e quella di beta test.
\end{description}
\bigskip
\Huge H \bigskip
\hrule
\smallskip
\normalsize
\begin{description}
	\item[Hosting:] in ambito informatico si definisce hosting (dall'inglese to host, ospitare) un servizio che consiste nell'allocare su un server web le pagine di un sito web, rendendolo cosí accessibile dalla rete Internet.

\end{description}
\bigskip
\Huge I \bigskip
\hrule
\smallskip
\normalsize
\begin{description}
	\item[Istanza:] software che consente agli utenti di visualizzare e interagire con testi, immagini e altre informazioni, tipicamente contenute in una pagina web di un sito. Il browser è in grado di interpretare il codice HTML (e più recentemente XHTML) e visualizzarlo in forma di ipertesto. L'HTML è il codice col quale la maggioranza delle pagine web nel mondo sono composte: il web browser consente perciò la navigazione nel web.
\end{description}
\bigskip
\Huge S \bigskip
\hrule
\smallskip
\normalsize
\begin{description}
	\item[Server:] un server (detto in italiano anche servente o serviente) è una componente informatica che fornisce servizi ad altre componenti (tipicamente chiamate client) attraverso una rete.
\end{description}
\bigskip
\Huge U \bigskip
\hrule
\smallskip
\normalsize
\begin{description}
	\item[Undeploy:] rimozione di un'applicazione precedentemente in stato di deploy.
\end{description}
\bigskip
\Huge W \bigskip
\hrule
\smallskip
\normalsize
\begin{description}
	\item[Web server:] un server web è un processo, e per estensione il computer su cui è in esecuzione, che si occupa di fornire, su richiesta del browser, una pagina web.
\end{description}
\modifiche
\end{document}
