\documentclass[11pt,a4paper]{article}
\usepackage{amsmath}
\usepackage{amsfonts}
\usepackage{amssymb}
\usepackage{fancyhdr}
\usepackage{lastpage}
\usepackage{graphicx}
\usepackage{ucs}
\usepackage[utf8x]{inputenc}
\usepackage[italian]{babel}
\usepackage[colorlinks=true,linkcolor=black]{hyperref}

\renewcommand{\headrulewidth}{0.6pt}
\renewcommand{\footrulewidth}{0.6pt}
% impostazione dello stile per le pagine interne del documento
\lhead{\leftmark}
\chead{}
\rhead{\includegraphics[scale=0.15]{logo.png} }
\lfoot{Norme di Progetto v1.2.1}
\cfoot{}
\rfoot{\thepage \ di \pageref{LastPage}}
% ridefinizione dello stile plain per il frontespizio
\fancypagestyle{plain}{
\fancyhf
}
% impostazione dello stile per l'indice
\fancypagestyle{indice}{
\lhead{\leftmark}
\chead{}
\rhead{\includegraphics[scale=0.15]{logo.png}}
\lfoot{Norme di Progetto v1.2.1}
\cfoot{}
\rfoot{}
}
\headheight = 46pt
%definizione del comando "\modfiche" per la creazione del diario delle modifiche
\newcommand{\modifiche} 
{
\newpage
\begin{center}
\textbf{Diario delle modifiche} \\
\bigskip
\begin{tabular}{|c|c|p{0.62\textwidth}|}
\hline
\textsc{Data} & \textsc{Versione} & \textsc{Modifica} \\
\hline
\hline
\textit{21-01-2009} & 1.2.1 & Correzioni varie\\
\hline
\textit{20-01-2009} & 1.2.0 & Aggiunta la sezione ''Strumenti''\\
\hline
\textit{18-01-2009} & 1.1.0 & Aggiunta la sezione ''Presentazioni'', correzioni varie\\
\hline
\textit{09-12-2008} & 1.0.0 & Approvazione del Responsabile e passaggio di stato in ''Formale''\\
\hline
\textit{07-12-2008} & 0.3.1 & Correzioni varie \\
\hline
\textit{05-12-2008} & 0.3.0 & Aggiunta delle norme tipografiche \\
\hline
\textit{04-12-2008} & 0.2.0 & Aggiunta del sommario \\
\hline
\textit{28-11-2008} & 0.1.1 & Correzione di alcuni errori grammaticali riportati dal verificatore \\
\hline
\textit{27-11-2008} & 0.1.0 & Prima stesura del documento \\
\hline
\end{tabular}
\end{center}
}
%definizione del comando "\info" per la creazione delle informazioni del documento
\newcommand{\info} {
\bigskip
\begin{tabbing}
	\hspace*{0.3\textwidth} \= \hspace*{0.5\textwidth} \kill
	\parbox{0.3\textwidth}{\textbf{Verifica: }} \> \parbox{0.5\textwidth}{Grosselle Alessandro} \\
	\parbox{0.3\textwidth}{\textbf{Approvazione: }} \> \parbox{0.5\textwidth}{Scarpa Davide} \\
	\parbox{0.3\textwidth}{\textbf{Stato: }} \> \parbox{0.5\textwidth}{Formale} \\
	\parbox{0.3\textwidth}{\textbf{Uso: }} \> \parbox{0.5\textwidth}{Interno} \\
	\parbox{0.3\textwidth}{\textbf{Distribuzione: }} \> \parbox{0.5\textwidth}{QuiXoft} \\
\end{tabbing}
}
%definizione del comando "\frontespizio" per la creazione del frontespizio
\newcommand{\frontespizio} {
\thispagestyle{plain}
\title{\begin{Huge}\textsc{Progetto SIGEOL}\end{Huge} \\ \textit{Norme di Progetto \\ v1.2.1}}
\author{Redazione: Scortegagna Carlo}
\maketitle
\medskip
\begin{center}
\includegraphics[scale=0.5]{logo.png} \\
\textit{quixoft.sol@gmail.com} \\
\end{center}
\medskip
\info
\begin{center}
\textbf{Sommario} \\
Norme interne per il progetto \textit{SIGEOL}, necessarie per regolamentare lo svolgersi delle operazioni di sviluppo.
\end{center}
\newpage
}
%definizione del comando "\indice" per la creazione dell'indice
\newcommand{\indice} {
\thispagestyle{indice}
\tableofcontents
\newpage
}
\pagestyle{fancy}
\begin{document}
\frontespizio
\indice
\setcounter{page}{1}
\section{Introduzione}
\subsection{Scopo del documento}
Lo scopo del presente documento è quello di fissare linee guida e norme precise per quanto riguarda la comunicazione aziendale, la stesura della documentazione, la codifica su calcolatore e l'amministrazione del progetto denominato SIGEOL.

Ogni membro del team QuiXoft è tenuto a leggere con attenzione il presente documento e applicarne in modo più coerente possibile il contenuto, al fine di ottimizzare il lavoro in team e le risorse a disposizione.

Sarà compito del Responsabile verificare che tali norme vengano applicate ed, eventualmente, notificare all’Amministrazione di progetto qualsiasi insubordinazione che mini il corretto procedere dello stesso.
\subsection{Definizioni}
Le definizioni dei termini specialistici usati nella stesura di questo e di tutti gli altri documenti  possono essere trovate nel documento \textsc{Glossario} al fine di eliminare ogni ambiguità e di facilitare la comprensione dei temi trattati. Ogni termine la cui definizione è disponibile all'interno del glossario verrà marcato con una \underline{sottolineatura}.
\section{Comunicazioni}
\subsection{Interne}
Le comunicazioni interne ufficiali (quali incontri, scadenze, appuntamenti obbligatori o altre comunicazioni ritenute importanti) si appoggiano al portale \textit{http://www.assembla.com}, che mette a disposizione del team una bacheca messaggi.

Ogni nuova comunicazione pubblicata nella bacheca verrà notificata auto\-maticamente via e-mail a tutti membri del team QuiXoft, qualsiasi ruolo essi ricoprano; la pubblicazione della stessa, invece, spetta al Responsabile.

Per tutte le altre comunicazioni interne al team è inoltre a disposizione un \underline{forum}, raggiungibile all'indirizzo \textit{http://quixoft.forumup.it/}, la cui struttura è decisa preventivamente dall'Amministratore di progetto.

All'interno di tale \underline{forum} ogni membro del team ha la possibilità di discu\-tere, liberamente, argomenti annessi al
progetto, o che possano in qualsiasi modo giovare al proseguimento dello stesso.

Sono sconsigliate, ma non vietate, le comunicazioni dirette tra singoli membri del team tramite qualsivoglia metodo: ogni comunicazione ineren\-te al progetto potrebbe infatti tornare utile a qualche altro membro del gruppo, oppure tornare utile in futuro.
E' consigliabile quindi che ogni comunicazione sia fatta usando i metodi sopra elencati.
Nonostante ciò, sono a disposizione gli indirizzi e-mail e i recapiti telefonici di ogni singolo membro del team QuiXoft.
\subsection{Esterne}
Ogni comunicazione con il commitente o con altri soggetti esterni al team QuiXoft potrà essere proposta da qualsiasi membro del gruppo, e spetterà al Responsabile deciderne l'utilità e la fattibilità.
In caso di responso positivo, al termine di ogni comunicazione esterna dovrà essere redatto un resoconto che ne elenchi sinteticamente i risultati. Per le modalità di creazione e diffuzione di tale resoconto, si faccia riferimento a quanto spiegato in seguito per i resoconti di incontro con il committente.

Il committente o altri soggetti esterni potranno invece comunicare con il team QuiXoft tramite l'indirizzo e-mail \textit{quixoft.sol@gmail.com}.
\section{Convocazioni ed incontri}
Ogni incontro, sia esso interno o esterno, dovrà essere preceduto da un annuncio in bacheca e dall'approvazione del Responsabile.
Al termine di qualsiasi incontro dovrà essere redatto un resoconto che contenga sinteticamente i risultati o i problemi emersi durante la discussione.
Per ulteriori informazioni sui documenti di resoconto di un incontro consultare il paragrafo 6.1.1.
\subsection{Interni}
\subsubsection{Obbligatori}
La cadenza degli incontri interni obbligatori è fissata dal Responsabile e comunicata ai membri del team QuiXoft tramite segnalazione con messaggio in bacheca.
Notifica via e-mail della convocazione verrà automaticamente mandata a tutti i membri del gruppo.
Riunioni di questo genere implicano la presenza obbligatoria di tutti i componenti del gruppo.

Il Responsabile si occuperà di volta in volta di selezionare l'ordine del giorno appropriato e di controllare il corretto svolgimento dell'incontro.
Gli argomenti da discutere e da inserire nell'ordine del giorno potranno essere proposti da qualsiasi membro del gruppo e saranno inseriti nell'annuncio di convocazione in bacheca, previa approvazione del Responsabile.
\subsubsection{Facoltativi}
La cadenza degli incontri facoltativi non è fissata a priori ed è legata a particolari necessità di uno o più membri del team.
Gli interessati dovranno richiedere la possibilità di svolgere tale incontro al Responsabile, il quale provvederà all'eventuale approvazione.
Sarà poi compito del Responsabile stesso comunicare, tramite pubblicazione sulla bacheca, gli estremi di svolgimento dell'incontro.

Riunioni di questo tipo non implicano la presenza obbligatoria di tutti i componenti, ma solamente dei ruoli interessati e specificati nel messaggio di convocazione in bacheca; non si esclude comunque la presenza degli altri elementi.
\subsection{Esterni}
\subsubsection{Incontri con il committente}
Ogni membro del gruppo ha facoltà di richiedere al Responsabile un incontro con il Commitente.
Spetta al Responsabile valutare la richiesta e, se lo ritiene opportuno e utile al progetto, approvare l'incontro.

Per formalizzare l'incontro dovrà essere pubblicato in bacheca il messaggio di convocazione, la cui notifica arriverà a tutti i membri del team.
All'incontro è obbligatoria solamente la presenza del richiedente, ma non si esclude la presenza di qualsiasi altro membro del gruppo che reputi di interesse tale incontro con il Committente.
\subsubsection{Altri incontri esterni}
Oltre che con il commitente, è data la possibilità di organizzare incontri con altri soggetti esterni al team QuiXoft, per qualsiasi argomento che possa aiutare il proseguimento del progetto.
La procedura di richiesta di tale tipo di incontro è analoga a quella sopra citata per gli Incontri con il Commitente.
\section{Repository}
Tutto il materiale prodotto dal team QuiXoft è disponibile online all'indirizzo \textit{http://trac.assembla.com/sigeol/browser}.

Il \underline{software} utilizzato per il \underline{versionamento} dei file è \underline{Subversion}.
O\-gni \underline{commit} effettuata da un qualsiasi membro del team QuiXoft deve essere accompagnata da una chiara e sintetica descrizione. Tale descrizione deve contenere traccia della nuova versione dei file che vengono aggiornati o aggiunti.
\subsection{Versione}
Ogni file o documento deve essere accompagnato dalla sua relativa versione. Si useranno 3 cifre (es: 0.1.0): la prima cifra sta ad indicare il rilascio di una nuova versione del file, la seconda cifra rappresenta una modifica e la terza cifra indica una correzione. La versione di un documento, oltre che nel commento alla \underline{commit}, deve assolutamente essere presente all'interno di ogni file di testo creato dal team QuiXoft, sia esso un documento o un file di codice.
\section{Documentazione}
\subsection{Template documenti}
Ogni documento redatto dal team QuiXoft dovrà essere creato utilizzando il relativo \underline{template} \LaTeX \space e successivamente esportato in \underline{Pdf} per garantirne la giusta \underline{portabilità}.

Se il \underline{template} non dovesse più calzare con le esigenze di documentazione del team QuiXoft, qualsiasi componente potrà richiederne la modifica al Responsabile di Progetto.
\subsubsection{Documenti di resoconto incontri}
I documenti di resoconto incontri dovranno essere creati usando il \underline{template} \textsc{template\_resoconto.tex} disponibile all'interno del \underline{repository} nella car\-tella \textsc{/doc/incontri/template/}. Non saranno ammesse modifiche al \underline{layout} e alla formattazione di tale \underline{template}.
\subsubsection{Documenti ufficiali interni ed esterni}
I documenti ufficiali interni ed esterni prodotti durante le varie attività aziendali dovranno essere redatti usando il \underline{template} \textsc{template\_ufficiale.tex} disponibile nella cartella \textsc{/doc/ufficiali/template/}.
Come sopra, si raccomanda di aderire coerentemente al \underline{template}, completando tutti i campi richiesti, senza apportare alcuna modifica.
\subsection{Norme tipografiche}
I documenti redatti, oltre che seguire la struttura  dei \underline{template} messi a disposizione, dovranno anche rispettare le seguenti convenzioni tipografiche:
\begin{itemize}
	\item gli indirizzi internet e i percorsi di cartella dovranno essere scritti in corsivo utilizzando il comando \LaTeX \space /textit 
\\ (esempio: \textit{http://www.assembla.com});
	\item i riferimenti a documenti esterni o a file (comprensivi di \underline{estensione}) andranno scritti usando il comando \LaTeX \space /textsc 
\\ (esempio: \textsc{template\_resoconto.tex});
	\item i termini definiti nel glossario verranno contrassegnati da una sottolineatura con il comando \LaTeX \space /underline (esempio: \underline{template});
	\item punto, virgola, punto e virgola, due punti, punto esclamativo e punto interrogativo sono attaccate alla parola che li precede, mentre sono separate con uno spazio dalla parola che li segue;
	\item parentesi (di ogni tipo), virgolette e trattini sono sempre attaccate al testo che delimitano, e separate con uno spazio dal resto del testo;
	\item i puntini di sospensione si scrivono attaccati alla parola che li precede;
	\item gli accenti in italiano sono sempre gravi, tranne nelle parole accentate che finiscono in -ché, in -né e in -sé.
\end{itemize}
\subsection{Nominazione dei file di documentazione}
\subsubsection{Resoconto incontri interni ed esterni}
Ogni documento di resoconto di un incontro dovrà essere salvato sul \underline{repository} del team QuiXoft in doppio formato:
\begin{itemize}
	\item in formato \underline{Pdf} nella cartella \textsc{/doc/incontri/pdf}, con \underline{estensione} .pdf
	\item in formato \underline{sorgente} \LaTeX \space nella cartella \textsc{/doc/incontri/tex}, con \underline{estensione} .tex
\end{itemize}
Al fine di non creare incongruenze, file con estensioni diverse da quelle sopra indicate saranno eliminati dal \underline{repository}.

I file dovranno essere nominati secondo i seguenti criteri: \\

\{data\}\_\{sigla\} \\

dove:

\begin{itemize}
	\item \{data\} indica, nel formato AAAA-MM-GG, la data dell'incontro
	\item \{sigla\} indica il tipo di incontro
\end{itemize}
\bigskip \medskip
I tipi di resoconto degli incontri che è possibile organizzare sono caratterizzati dalle seguenti sigle:

\begin{itemize}
	\item RIO: resoconto incontro interno obbligatorio
	\item RIF: resoconto incontro interno facoltativo
	\item RIC: resoconto incontro con il committente
	\item RIE: resoconto altri incontri esterni
\end{itemize}
\subsubsection{Documenti ufficiali interni ed esterni}
I documenti ufficiali prodotti dal team QuiXoft dovranno essere salvati sul \underline{repository} in due modi:
\begin{itemize}
	\item in formato \underline{Pdf} nella cartella 
\\ \textsc{/doc/ufficiali/\{nome\_documento\}/pdf}
	\item in \underline{sorgente} \LaTeX \space nella cartella 
\\ \textsc{/doc/ufficiali/\{nome\_documento\}/tex}
\\ Insieme al file \underline{sorgente} \LaTeX \space dovranno essere inseriti nella medesima cartella anche tutti gli altri file necessari alla compilazione (immagini che differiscano da quelle usate nel \underline{template}, grafici, ecc...)
\end{itemize}
I file dei documenti ufficiali interni ed esterni dovranno essere composti solamente dal nome del documento e dalla relativa \underline{estensione}. Si potrà risalire alla versione consultando l'\underline{intestazione} del file o il registro delle modifiche.
\subsection{Approvazione e verifica documentazione}
Una volta redatto, un qualsiasi documento deve essere marcato come ''Preliminare'' nel campo ''Stato'' del relativo \underline{template} \LaTeX.

In seguito tale documento deve essere sottoposto ad una fase di controllo da parte del Verificatore, il cui compito è quello di notificare eventuali incongruenze al redattore del documento.
Spetta a quest'ultimo apportare le dovute modifiche e sottoporre nuovamente al Verificatore il documento.
Una volta che il Verificatore avrà segnalato come ''verificato'' il documento, il Responsabile provvederà all'approvazione ed al cambio di stato in ''Formale''.
\section{Presentazioni}
In occasione di ogni revisione, sia formale sia informale, deve essere organizzata una presentazione che illustri il proseguimento del progetto al committente o al docente di riferimento.

Ogni membro del team QuiXoft è tenuto a partecipare alle revisioni ed a esporre la sua parte dei contenuti richiesti.
Tutti i componenti devono dare il loro contributo alla presentazione: il carico temporale durante la presentazione deve essere equamente distribuito tra tutti i presenti.

Assenze a tali revisioni dovranno essere giustamente motivate.

Il tempo massimo che una presentazione può occupare è fissato in 20 minuti: il superamento di tale limite avrà effetto sanzionatorio sul risultato del gruppo.
Se la presentazione dovesse durare meno di 20 minuti, sarà dato spazio alle domande del committente o del docente di riferimento per colmare le eventuali lacune lasciate dall'esposizione.
\subsection{Template presentazioni}
Per ogni revisione a cui il team QuiXoft parteciperà dovrà essere creata una presentazione che, con una serie di slide, accompagni il proseguimento dell'esposizione dei contenuti.

Le presentazioni dovranno essere create usando il relativo \underline{template} \LaTeX \space messo a disposizione nel repository del team.
Dovranno poi essere esportate in \underline{Pdf} e mostrate agli interessati tramite proiettore.
L'elaboratore usato per tale operazione sarà di volta in volta messo a disposizione da uno dei membri del team.

La scrittura dei contenuti dovrà seguire le norme tipografiche usate per la creazione della documentazione.
Particolare attenzione dovrà essere posta nel limitare la quantità di testo presente in ogni slide: dovrà essere data priorità alla presenza di concetti e parole chiave, che poi saranno adeguatamente spiegati dal relatore.

E' consigliato l'uso di figure e illustrazioni, che aiutino a rendere più agevole la comprensione dei concetti discussi.

Non è fissato a priori un numero massimo di slide per ogni presentazione. Si tenga comunque conto del limite temporale sopra illustrato.
Caratteristica obbligatoria è la presenza del numero della slide corrente e del totale, per dare modo al docente o al committente di quantificare i tempi.
\section{Codice}
\subsection{Intestazione}
Ogni file di codice prodotto dal team QuiXoft durante lo sviluppo del progetto SIGEOL dovrà contenere un \underline{intestazione} che ne certifichi l'origine e le modifiche per facilitarne la tracciabilità e le operazioni di \underline{manutenzione}.
Tale \underline{intestazione} deve essere cosi formattata: \\

\textit{QuiXoft - Progetto ''SIGEOL''}

\textit{NOME FILE: \{nome del file, completo di \underline{estensione}\}}

\textit{AUTORE: \{cognome e nome dell'autore del file\}}

\textit{DATA CREAZIONE: \{data di creazione nel formato GG-MM-AAAA\}}

\textit{REGISTRO DELLE MODIFICHE:}

\textit{\{data e resoconto delle modifiche, in ordine cronologico decrescente\}}
\subsection{Altre convenzioni}
Altre norme riguardanti i file \underline{sorgente} verranno decise dal Responsabile del team QuiXoft al momento della scelta dell'\underline{ambiente di sviluppo} e dei linguaggi di programmazione utilizzati. Questo perchè alcuni linguaggi (es: ''\underline{Ruby}'') ed alcuni ambienti di programmazione possiedono già delle regole prestabilite. Le norme che saranno decise nelle fasi successive dello sviluppo riguarderanno:
\begin{itemize}
	\item struttura e posizionamento dei commenti
	\item nomi di variabili, classi, metodi, costanti, ecc...
	\item lunghezza massima delle righe
	\item lunghezza delle \underline{indentazioni}
	\item posizionamento delle parentesi
	\item spaziature
\end{itemize}
\section{Compiti}
\subsection{Ticket}
Appoggiandosi al portale \textit{http://www.assembla.com} il Responsabile dovrà creare un \underline{ticket} per ogni singolo compito da assegnare ad uno o più membri del team QuiXoft, comprese le operazioni di verifica.
Sarà poi cura di quest'ultimo accettare o rifiutare il \underline{ticket}.

Al momento della creazione di ciascun \underline{ticket} è indispensabile stimare le ore necessarie per portarlo a termine e completare correttamente tutti i campi dati richiesti.

Una volta completato, il \underline{ticket} assegnato dovrà essere marcato come ''Close as fixed''. Dovranno altresi essere indicate con precisione le ore che sono state necessarie al suo completamento.

Caso particolare di tutto ciò appena elencato sono i ticket relativi al Responsabile: quest'ultimo dovrà crearli, accettarli e gestirli autonomamente, e segnalare le ore impiegate per portarli a compimento come qualsiasi altro membro del team QuiXoft.

E' importante e necessario che tutti i ticket siano tracciati con le ore impiegate a completarli per tenere sotto controllo le risorse e i relativi costi e per non rischiare di sforare le stime illustrate nel \textsc{Piano di Progetto}.
\subsection{Milestone}
I \underline{ticket} assegnati ai vari membri del team QuiXoft sono raggruppati in \underline{milestone} e gestiti sempre attraverso il portale \textit{http://www.assembla.com}.

Il raggiungimento di un \underline{milestone} rappresenta il completamento di uno degli obiettivi che il Responsabile ha stabilito in fase di definizione del progetto stesso.
\section{Strumenti}
Per facilitare il lavoro di tutti i membri del team QuiXoft e per rendere interoperabili i loro risultati saranno ora elencati gli strumenti da usare per svolgere i vari compiti assegnati.

Salvo diverse indicazioni, è fatto obbligo lo sviluppo di tutto il progetto con i medesimi software e lo stesso sistema operativo, per non incappare nei vari problemi che sorgerebbero nell'usare una gran varietà di ambienti diversi.

Tutti i software elencati dovranno essere aggiornati all'ultima versione stabile disponibile: non si escludono aggiornamenti anche in corso d'opera.

Eccezione a quanto appena detto, durante la fase di collaudo del sistema SIGEOL sarà necessario testare il prodotto sotto vari sistemi operativi, per verificarne l'effettiva portabilità.
\begin{itemize}
 \item tutto il progetto dovrà necessariamente essere sviluppato all'interno di un sistema operativo \textbf{GNU/Linux}. La scelta di distribuzione e versione è lasciata ai singoli membri del team QuiXoft, ma è comunque consigliabile l'utilizzo di un sistema operativo basato su Debian (come, per esempio, Ubuntu) per la facilità di gestione e per la qualità dell'architettura e dell'infrastruttura.
 \item per la scrittura e la compilazione di tutta la documentazione e delle presentazioni verrà usato il programma \textbf{\underline{Kile}} (KDE Integrated LaTeX Environment). Per la compilazione dei sorgenti \LaTeX \space verrà utilizzato PDFLaTeX.
 \item l'intero progetto verrà sviluppato usando \textbf{\underline{NetBeans}} alla sua ultima versione. La scelta del linguaggio di programmazione e del relativo framework è caduta su \textbf{\underline{Ruby on Rails}}, eccellente per le esigenze del progetto SIGEOL e perfettamente integrato all'interno di \underline{NetBeans}. Quest'ultimo mette a disposizione notevoli caratteristiche cha facilitano lo sviluppo con \underline{Ruby on Rails}. \underline{NetBeans} verrà anche utilizzato come editor per la creazione di tutti gli schemi \underline{UML} necessari.
 \item il \underline{DataBase} Management System (DBMS) usato durante lo sviluppo potrà essere scelto a piacimento da ognuno dei membri del team QuiXoft: è raccomandato l'uso di svariati DBMS proprio per evidenziare eventuali errori che potrebbero sfuggire se si decidesse di usare sempre lo stesso.
Ci si potrà facilmente interfacciare ai vari DBMS scelti fruttando le funzionalità offerte da Active Record, comoda libreria messa a disposizione per \underline{Ruby on Rails}. Il codice sorgente dell'applicazione SIGEOL sarà completamente indipendente dal DBMS scelto, rendendo possibile lo sviluppo, i test e l'installazione del prodotto finito lasciando la scelta finale del DBMS da usare al Committente. Si consiglia comunque la scelta di un \underline{database} management system con licenza d'uso free oppure open source, per non gravare ulteriormente sul costo del progetto (come, per esempio, MySQL, PostgreSQL o SQLite).
\end{itemize}

Per gli specifici strumenti da adottare per lo svolgimento delle attività di verifica, validazione e controllo qualità si prega di consultare il \textsc{Piano di qualifica} alla sua ultima versione.

La lista appena conclusa è da intendersi provvisoria ed incompleta: col procedere dello sviluppo del progetto potranno sorgere ulteriori esigenze da parte del team QuiXoft, che dovranno essere soddisfatte con un aggiornamento o una estensione degli strumenti da utilizzare.


\modifiche
\end{document}
