\documentclass[11pt,a4paper]{article}
\usepackage{amsmath}
\usepackage{amsfonts}
\usepackage{amssymb}
\usepackage{fancyhdr}
\usepackage{lastpage}
\usepackage{graphicx}
\usepackage{ucs}
\usepackage[utf8x]{inputenc}
\usepackage[italian]{babel}

\renewcommand{\headrulewidth}{0.6pt}
\renewcommand{\footrulewidth}{0.6pt}
% impostazione dello stile per le pagine interne del documento
\lhead{\leftmark}
\chead{}
\rhead{\includegraphics[scale=0.15]{logo.png} }
\lfoot{Analisi dei requisiti v0.3.0}
\cfoot{}
\rfoot{\thepage \ di \pageref{LastPage}}
% ridefinizione dello stile plain per il frontespizio
\fancypagestyle{plain}{
\fancyhf
}
% impostazione dello stile per l'indice
\fancypagestyle{indice}{
\lhead{\leftmark}
\chead{}
\rhead{\includegraphics[scale=0.15]{logo.png}}
\lfoot{Analisi dei requisiti v0.3.0}
\cfoot{}
\rfoot{}
}
\headheight = 46pt
%definizione del comando "\modfiche" per la creazione del diario delle modifiche
\newcommand{\modifiche} 
{
\newpage
\begin{center}
\textbf{Diario delle modifiche} \\
\bigskip
\begin{tabular}{|c|c|p{0.51\textwidth}|}
\hline
\textsc{Data} & \textsc{Versione} & \textsc{Modifica} \\
\hline
\hline
\textit{29 novembre 2008} & 0.3.0 & Inizio stesura della descrizione del prodotto \\
\hline
\textit{28 novembre 2008} & 0.2.0 & Stesura dell'introduzione \\
\hline
\textit{28 novembre 2008} & 0.1.0 & Stesura dell'indice \\
\hline
\end{tabular}
\end{center}
}
%definizione del comando "\info" per la creazione delle informazioni del documento
\newcommand{\info} {
\bigskip
\begin{tabbing}
	\hspace*{0.3\textwidth} \= \hspace*{0.5\textwidth} \kill
	\parbox{0.3\textwidth}{\textbf{Verifica: }} \> \parbox{0.5\textwidth}{Verificatore} \\
	\parbox{0.3\textwidth}{\textbf{Approvazione: }} \> \parbox{0.5\textwidth}{Responsabile} \\
	\parbox{0.3\textwidth}{\textbf{Stato: }} \> \parbox{0.5\textwidth}{Preliminare} \\
	\parbox{0.3\textwidth}{\textbf{Uso: }} \> \parbox{0.5\textwidth}{Esterno} \\
	\parbox{0.3\textwidth}{\textbf{Distribuzione: }} \> \parbox{0.5\textwidth}{QuiXoft} \\
\end{tabbing}
}
%definizione del comando "\frontespizio" per la creazione del frontespizio
\newcommand{\frontespizio} {
\thispagestyle{plain}
\title{\begin{Huge}\textsc{Progetto SIGEOL}\end{Huge} \\ \textit{Analisi dei requisiti \\ v0.3.0}}
\author{Redazione: Beggiato Andrea, Scarpa Davide, Barbiero Mattia }
\maketitle
\medskip
\begin{center}
\includegraphics[scale=0.5]{logo.png} \\
\textit{quixoft.sol@gmail.com}
\end{center}
\medskip
\info
\newpage
}
%definizione del comando "\indice" per la creazione dell'indice
\newcommand{\indice} {
\thispagestyle{indice}
\tableofcontents
\newpage
}
\pagestyle{fancy}
\begin{document}
\frontespizio
\indice
\setcounter{page}{1}
\section{Introduzione}
\subsection{Scopo del documento}
Il presente documento denominato \textit{Analisi dei requisiti} ha lo scopo di delineare tutti i bisogni espressi dal committente Prof. Rossi Francesca per il sistema \textit{SIGEOL}, nonchè tutti i requisiti intrinsechi nello sviluppo di un tale prodotto. 
\subsection{Scopo del prodotto}
Il progetto sotto analisi, denominato \textit{SIGEOL}, si prefigge di automatizzare la generazione, la gestione, l'ottimizzazione e la consultazione degli orari di lezione. Il committente richiede l'applicazione del sistema al solo corso di laurea in informatica, ma, constatato che la complessità non aumenta notevolmente, il team QuiXoft prevede lo sviluppo e la messa in opera dell'applicazione per tutti i corsi di laurea dell' Università degli studi di Padova.

Il prodotto sarà implementato come un servizio web portabile, facilmente manutenibile ed accessibile agli utenti da una qualsiasi postazione con accesso alla rete Internet.
\subsection{Glossario}
Le definizioni dei termini specialistici usati nella stesura di questo e di tutti gli altri documenti possono essere trovate nel documento ''Glossario'' al fine di eliminare ogni ambiguità e di facilitare la comprensione dei temi trattati. Ogni termine la cui definizione è disponibile all’interno del Glossario verrà marcato con una \underline{sottolineatura}.
\subsection{Riferimenti}
\begin{itemize}
 \item Capitolato d'appalto reperibile all'indirizzo: \\ http://www.math.unipd.it/~tullio/IS-1/2008/Progetti/SIGEOL.html
 \item Statuto di Ateneo reperibile all'indirizzo \\ http://www.unipd.it/organizzazione/statuto/statuto.htm
 \item Informativa sulla privacy (Legge per il trattamento dei dati personali)
 \item Incontri con il committente
\end{itemize}

\section{Descrizione generale}
\subsection{Contesto d'uso del prodotto}
\subsubsection{Processi produttivi e modalità d'uso}
Il funzionamento del sistema \textit{SIGEOL}, a processo produttivo concluso, sarà in grado di guidare ogni singolo utilizzatore (confronta sezione \ref{utenti}) allo svolgimento delle proprie azioni (confronta sezione \ref{funzioni}). In altre parole sarà disponibile un servizio che offrirà, dopo un'opportuna autenticazione, un insieme di strumenti che permetteranno l'inserimento guidato dei dati che serviranno al fine ultimo di generare un orario per le lezioni.
\subsubsection{Piattaforma d’esecuzione ed interfacciamento con l’ambiente di installazione e uso}
Il prodotto sarà realizzato tramite un'applicazione web supportata da un database e dovrà essere accessibile da un qualsiasi tipo di browser. Data la mancanza di un applicativo preesistente al quale aggiungere le funzioni del sistema \textit{SIGEOL}, il team QuiXoft dovrà definirne ogni aspetto creando un prodotto che sia accessibile, manutenibile, portabile e soprattutto sicuro. A tal scopo verranno adottate tecnologie gratuite e moderne per lo svilippo di pagine dinamiche e sarà necessario disporre di un server affidabile.
\subsection{Funzioni del prodotto} \label{funzioni}

\subsection{Caratteristiche degli utenti} \label{utenti} 
\subsection{Vincoli generali}
\subsection{Assunzioni e dipendenze}
\section{Requisiti}
\subsection{Requisiti funzionali}
\subsection{Requisiti di qualità}
\subsection{Requisiti d'interfacciamento e d'ambiente}
\section{Use case e descrizioni narrative}
\subsection{Use case generale}
\modifiche
\end{document}
