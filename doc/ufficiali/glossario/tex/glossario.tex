\documentclass[11pt,a4paper]{article}
\usepackage{amsmath}
\usepackage{amsfonts}
\usepackage{amssymb}
\usepackage{fancyhdr}
\usepackage{lastpage}
\usepackage{graphicx}
\usepackage{ucs}
\usepackage[utf8x]{inputenc}
\usepackage[italian]{babel}

\renewcommand{\headrulewidth}{0.6pt}
\renewcommand{\footrulewidth}{0.6pt}
% impostazione dello stile per le pagine interne del documento
\lhead{QuiXoft - Progetto SIGEOL}
\chead{}
\rhead{\includegraphics[scale=0.15]{logo.png} }
\lfoot{Glossario v0.2.0}
\cfoot{}
\rfoot{\thepage \ di \pageref{LastPage}}
% ridefinizione dello stile plain per il frontespizio
\fancypagestyle{plain}{
\fancyhf
}
% impostazione dello stile per l'indice
\fancypagestyle{indice}{
\lhead{\leftmark}
\chead{}
\rhead{\includegraphics[scale=0.15]{logo.png}}
\lfoot{Glossario v0.2.0}
\cfoot{}
\rfoot{}
}
\headheight = 46pt
%definizione del comando "\modfiche" per la creazione del diario delle modifiche
\newcommand{\modifiche} 
{
\newpage
\begin{center}
\textbf{Diario delle modifiche} \\
\bigskip
\begin{tabular}{|c|c|p{0.55\textwidth}|}
\hline
\textsc{Data} & \textsc{Versione} & \textsc{Modifica} \\
\hline
\hline
\textit{02-12-2008} & 0.2.0 & Aggiunta delle definizioni dell'analisi dei requisiti \\
\hline
\textit{01-12-2008} & 0.1.0 & Prima bozza del glossario \\
\hline
\end{tabular}
\end{center}
}
%definizione del comando "\info" per la creazione delle informazioni del documento
\newcommand{\info} {
\bigskip
\begin{tabbing}
	\hspace*{0.3\textwidth} \= \hspace*{0.5\textwidth} \kill
	\parbox{0.3\textwidth}{\textbf{Verifica: }} \> \parbox{0.5\textwidth}{Freo Matteo} \\
	\parbox{0.3\textwidth}{\textbf{Approvazione: }} \> \parbox{0.5\textwidth}{Grosselle Alessandro} \\
	\parbox{0.3\textwidth}{\textbf{Stato: }} \> \parbox{0.5\textwidth}{Preliminare} \\
	\parbox{0.3\textwidth}{\textbf{Uso: }} \> \parbox{0.5\textwidth}{Interno} \\
	\parbox{0.3\textwidth}{\textbf{Distribuzione: }} \> \parbox{0.5\textwidth}{QuiXoft} \\
\end{tabbing}
}
%definizione del comando "\frontespizio" per la creazione del frontespizio
\newcommand{\frontespizio} {
\thispagestyle{plain}
\title{\begin{Huge}\textsc{Progetto SIGEOL}\end{Huge} \\ \textit{Glossario \\ v0.2.0}}
\author{Redazione: Carlo Scortegagna}
\maketitle
\medskip
\begin{center}
\includegraphics[scale=0.5]{logo.png} \\
\textit{quixoft.sol@gmail.com}
\end{center}
\medskip
\info
\newpage
}
%definizione del comando "\indice" per la creazione dell'indice
\newcommand{\indice} {
\thispagestyle{indice}
\tableofcontents
\newpage
}
\pagestyle{fancy}
\begin{document}
\frontespizio
\setcounter{page}{1}
\smallskip
\flushleft \Huge A \bigskip
\hrule
\smallskip
\normalsize
\begin{description}
	\item[Accessibile:]
	\item[Ambiente di sviluppo:] \underline{software} che serve a sviluppare altro software. Può essere costituito da semplici e singoli programmi fino a complesse suite di sviluppo (vedi anche \underline{framework}).
	\item[Amministratore:] all'interno di un team è colui che è responsabile dell'efficienza e dell'operatività dell'\underline{ambiente di sviluppo}. Si occupa inoltre della redazione dei piani di Gestione per la Qualità, del controllo delle \underline{versioni} del prodotto, della gestione dell'archivio della documentazione di progetto. Partecipa inoltre alla redazione del Piano di Progetto e delle Norme di Progetto per conto del \underline{Responsabile}.
	\item[Analista:] è colui che all'interno del team di occupa delle attività di analisi. Redige lo Studio di Fattibilità e l'Analisi dei Requisiti.
	\item[Autenticazione:]
\end{description}
\bigskip
\Huge B \bigskip
\hrule
\smallskip
\normalsize
\begin{description}
	\item[Browser:]
\end{description}
\bigskip
\Huge C \bigskip
\hrule
\smallskip
\normalsize
\begin{description}
	\item[Commit:] operazione che si effettua quando si copiano le modifiche fatte su file locali nella corrispondente directory remota del \underline{repository}.
\end{description}
\bigskip
\Huge D \bigskip
\hrule
\smallskip
\normalsize
\begin{description}
	\item[Database:]
\end{description}
\bigskip
\Huge E \bigskip
\hrule
\smallskip
\normalsize
\begin{description}
	\item[Estensione:] breve sequenza di caratteri alfanumerici aggiunti dopo il nome di un file e separati da quest'ultimo da un punto. L'estensione permette all'utente di un computer, ma anche ad alcuni programmi, di distinguere tra i vari formati di file.
\end{description}
\bigskip
\Huge F \bigskip
\hrule
\smallskip
\normalsize
\begin{description}
	\item[Forum:] struttura informatica contenente discussioni e messaggi scritti dagli utenti. Viene utilizzato per mettere in comunicazione i componenti del team e permette loro di reperire informazioni riguardanti il progetto.
\end{description}
\bigskip
\Huge I \bigskip
\hrule
\smallskip
\normalsize
\begin{description}
	\item[Indentazione:] inserimento di una certa quantità di spazio vuoto all'inizio di una riga di testo.
	\item[Intestazione:] porzione di testo che si trova all'inizio di un file \underline{sorgente}. Contiene solitamente l'autore del file, la \underline{versione} e altre informazioni utili a descriverlo.
\end{description}
\bigskip
\Huge L \bigskip
\hrule
\smallskip
\normalsize
\begin{description}
	\item[Latex:] linguaggio usato per la preparazione e formattazione di testi. Fornisce mezzi per l'automazione della maggior parte della composizione tipografica, inclusa la numerazione, i riferimenti incrociati, tabelle e figure, organizzazione delle pagine, bibliografie e molto altro.
	\item[Layout:] impaginazione e la struttura grafica di un documento.
	\item[Linguaggio di Programmazione:] linguaggio formale, dotato di una sintassi e di una semantica ben definite, utilizzabile per il controllo del comportamento di una computer.
\end{description}
\bigskip
\Huge M \bigskip
\hrule
\smallskip
\normalsize
\begin{description}
	\item[Manutenzione:] processo correttivo e di sviluppo che avviene dopo il rilascio del prodotto finale.
	\item[Milestone:] traguardi intermedi nello svolgimento di un progetto che indicano il raggiungimento di obiettivi precedentemente stabiliti. Il raggiungimento delle milestones viene decretato tramite documenti ufficiali redatti dai vari componenti del team e grazie ad esse risulta possibile fornire una stima della bontà del progetto e del suo stato di avanzamento.
\end{description}
\bigskip
\Huge P \bigskip
\hrule
\smallskip
\normalsize
\begin{description}
	\item[Pdf:] il Portable Document Format, comunemente abbreviato Pdf, è un formato di file basato su un linguaggio di descrizione di pagina sviluppato da Adobe per rappresentare documenti in modo indipendente dall'hardware e dal \underline{software} utilizzati per generarli o per visualizzarli. Ogni documento redatto dal team QuiXoft viene esportato in questo formato.
	\item[Portabilità:] adattamento o una modifica di un componente, volto a consentirne l'uso in un ambiente di esecuzione diverso da quello originale. In una più moderna accezione, si parla di portabilità di un componente \underline{software} o di un documento intendendo che può essere utilizzato su diverse piattaforme senza apportare alcuna modifica.
	\item[Progettista:] è il responsabile delle attività di progettazione all'interno del team. Redige la Specifica Tecnica, la Definizione di Prodotto e la parte programmatica del Piano di Qualifica.
	\item[Programmatore:] è responsabile delle attività di codifica miranti alla realizzazione del prodotto e delle componenti di ausilio necessarie per l'esecuzione delle prove di verifica e validazione.
\end{description}
\bigskip
\Huge R \bigskip
\hrule
\smallskip
\normalsize
\begin{description}
	\item[Repository:] punto centrale in cui i dati vengono immagazzinati e manutenuti. Può essere un punto in cui sono localizzati database o file multipli per la distribuzione via rete, oppure una collocazione direttamente accessibile dall'utente. Utilizzato principalmente in un contesto di sviluppo concorrente.
	\item[Responsabile:] è colui che si fa carico, per conto del suo team, dei risultati del progetto. Elabora ed emana piani e scadenze, approva l'emissione di documenti, coordina le attività del gruppo e si relaziona con il controllo di qualità interno al progetto. Redige Organigramma e Piano di Progetto e approva l'Offerta ed i relativi allegati.
	\item[Ruby:] \underline{linguaggio di programmazione} completamente a oggetti ed interpretato.
\end{description}
\bigskip
\Huge S \bigskip
\hrule
\smallskip
\normalsize
\begin{description}
	\item[Server:]
	\item[Servizio web:]
	\item[Sigeol:] sistema software per la gestione dell'orario di lezione a uso del CCS di Informatica.
	\item[Software:] vocabolo creato a partire da due termini della lingua inglese, soft (morbido) e ware (manufatto, oggetto) che sta ad indicare un programma o un insieme di programmi in grado di funzionare su un elaboratore.
	\item[Sorgente:] insieme di istruzioni appartenenti ad un determinato \underline{linguaggio di programmazione}, utilizzato per realizzare un programma per computer.
	\item[Subversion:] sistema di controllo versione, noto anche come Svn.
\end{description}
\bigskip
\Huge T \bigskip
\hrule
\smallskip
\normalsize
\begin{description}
	\item[Template:] documento o programma dove, come in un foglio semicompilato cartaceo, su una struttura generica o standard esistono spazi temporaneamente "bianchi" da riempire successivamente.
	\item[Ticket:]
\end{description}
\bigskip
\Huge V \bigskip
\hrule
\smallskip
\normalsize
\begin{description}
	\item[Verificatore:] all'interno del team è responsabile delle attività di verifica. Redige la parte retrospettiva del Piano di Qualifica che illustra l'esito e la completezza delle verifiche e delle prove effettuate secondo il piano.
	\item[Versionamento:] termine che sta ad indicare il controllo versione, la gestione cioè di versioni multiple di un insieme di informazioni.
\end{description}
\bigskip
\modifiche
\end{document}
