\documentclass[11pt,a4paper]{article}
\usepackage{amsmath}
\usepackage{amsfonts}
\usepackage{amssymb}
\usepackage{fancyhdr}
\usepackage{lastpage}
\usepackage{graphicx}
\usepackage{ucs}
\usepackage[utf8x]{inputenc}
\usepackage[italian]{babel}

\renewcommand{\headrulewidth}{0.6pt}
\renewcommand{\footrulewidth}{0.6pt}
% impostazione dello stile per le pagine interne del documento
\lhead{QuiXoft - Progetto SIGEOL}
\chead{}
\rhead{\includegraphics[scale=0.15]{logo.png} }
\lfoot{Glossario v1.2.1}
\cfoot{}
\rfoot{\thepage \ di \pageref{LastPage}}
% ridefinizione dello stile plain per il frontespizio
\fancypagestyle{plain}{
\fancyhf
}
% impostazione dello stile per l'indice
\fancypagestyle{indice}{
\lhead{\leftmark}
\chead{}
\rhead{\includegraphics[scale=0.15]{logo.png}}
\lfoot{Glossario v1.2.1}
\cfoot{}
\rfoot{}
}
\headheight = 46pt
%definizione del comando "\modfiche" per la creazione del diario delle modifiche
\newcommand{\modifiche} 
{
\newpage
\begin{center}
\textbf{Diario delle modifiche} \\
\bigskip
\begin{tabular}{|c|c|p{0.62\textwidth}|}
\hline
\textsc{Data} & \textsc{Versione} & \textsc{Modifica} \\
\hline
\hline
\textit{24-01-2009} & 1.2.1 & Correzioni varie\\
\hline
\textit{23-01-2009} & 1.2.0 & Aggiunta di alcune definizioni\\
\hline
\textit{18-01-2009} & 1.1.0 & Eliminazione di alcuni termini ritenuti superflui\\
\hline
\textit{09-12-2008} & 1.0.0 & Approvazione del Responsabile e passaggio di stato in ''Formale''\\
\hline
\textit{08-12-2008} & 0.4.1 & Correzioni varie \\
\hline
\textit{07-12-2008} & 0.4.0 & Aggiunta delle definizioni dei termini ricavati dal Piano di Qualifica \\
\hline
\textit{04-12-2008} & 0.3.0 & Stesura del sommario \\
\hline
\textit{03-12-2008} & 0.2.1 & Aggiornamento di alcune definizioni \\
\hline
\textit{02-12-2008} & 0.2.0 & Aggiunta delle definizioni derivate dall'analisi dei requisiti \\
\hline
\textit{01-12-2008} & 0.1.0 & Prima bozza del glossario \\
\hline
\end{tabular}
\end{center}
}
%definizione del comando "\info" per la creazione delle informazioni del documento
\newcommand{\info} {
\bigskip
\begin{tabbing}
	\hspace*{0.3\textwidth} \= \hspace*{0.5\textwidth} \kill
	\parbox{0.3\textwidth}{\textbf{Verifica: }} \> \parbox{0.5\textwidth}{Freo Matteo} \\
	\parbox{0.3\textwidth}{\textbf{Approvazione: }} \> \parbox{0.5\textwidth}{Grosselle Alessandro} \\
	\parbox{0.3\textwidth}{\textbf{Stato: }} \> \parbox{0.5\textwidth}{Formale} \\
	\parbox{0.3\textwidth}{\textbf{Uso: }} \> \parbox{0.5\textwidth}{Esterno} \\
	\parbox{0.3\textwidth}{\textbf{Distribuzione: }} \> \parbox{0.5\textwidth}{QuiXoft} \\
	\> \parbox{0.5\textwidth}{Rossi Francesca} \\
	\> \parbox{0.5\textwidth}{Vardanega Tullio} \\
	\> \parbox{0.5\textwidth}{Conte Renato} \\
\end{tabbing}
}
%definizione del comando "\frontespizio" per la creazione del frontespizio
\newcommand{\frontespizio} {
\thispagestyle{plain}
\title{\begin{Huge}\textsc{Progetto SIGEOL}\end{Huge} \\ \textit{Glossario \\ v1.2.1}}
\author{Redazione: Carlo Scortegagna}
\maketitle
\medskip
\begin{center}
\includegraphics[scale=0.5]{logo.png} \\
\textit{quixoft.sol@gmail.com}
\end{center}
\medskip
\info
\begin{center}
\textbf{Sommario} \\
Documento che definisce i termini usati nei documenti ufficiali dell’azienda QuiXoft per il progetto ``SIGEOL'' al fine di eliminare ogni ambiguità relativa al linguaggio.
\end{center}
\newpage
}
%definizione del comando "\indice" per la creazione dell'indice
\newcommand{\indice} {
\thispagestyle{indice}
\tableofcontents
\newpage
}
\pagestyle{fancy}
\begin{document}
\frontespizio
\setcounter{page}{1}
\smallskip
\flushleft \Huge A \bigskip
\hrule
\smallskip
\normalsize
\begin{description}
	\item[Accessibile:] capacità di un servizio d'essere fruibile con facilità da una qualsiasi categoria d'utente. 
	\item[Account:] insieme di funzionalità, strumenti e contenuti attribuiti ad un utente in determinati contesti operativi. In informatica, attraverso il meccanismo dell'account, il sistema mette a disposizione dell'utente un ambiente con contenuti e funzionalità personalizzabili, oltre ad un conveniente grado di isolamento dalle altre utenze parallele.
	\item[Ambiente di sviluppo:] software che serve a sviluppare altro software. Può essere costituito da semplici e singoli programmi fino a complesse suite di sviluppo (vedi anche framework).
	\item[Autenticazione:] processo tramite il quale un computer, un software o un utente verifica la corretta identità di un altro computer, software o utente che vuole comunicare attraverso una connessione. Un sistema di elaborazione, progettato per essere usato soltanto da utenti autorizzati, deve essere in grado di rilevare ed escludere i non autorizzati. L'accesso ad esso, dunque, viene garantito solo dopo aver eseguito con successo una procedura di autenticazione, di solito attraverso una username e una password personale.
\end{description}
\bigskip
\Huge B \bigskip
\hrule
\smallskip
\normalsize
\begin{description}
	\item[Backup:] importante operazione tesa a duplicare su differenti supporti di memoria le informazioni (dati o programmi) presenti sui dischi di una stazione di lavoro o di un server. Normalmente viene svolta con una periodicità stabilita.
	\item[Benchmark:] insieme di test svolti per determinare la capacità di un software di svolgere più o meno velocemente, precisamente od accuratamente, un particolare compito per cui è stato progettato.
	\item[Browser:] software che consente agli utenti di visualizzare e interagire con testi, immagini e altre informazioni, tipicamente contenute in una pagina web di un sito. Il browser è in grado di interpretare il codice HTML (e più recentemente XHTML) e visualizzarlo in forma di ipertesto. L'HTML è il codice col quale la maggioranza delle pagine web nel mondo sono composte: il web browser consente perciò la navigazione nel web.
\end{description}
\bigskip
\Huge C \bigskip
\hrule
\smallskip
\normalsize
\begin{description}
	\item[Collaudo:] procedimento utilizzato per individuare le carenze di correttezza, completezza e affidabilità delle componenti software in corso di sviluppo. Consiste nell'eseguire il software da collaudare, da solo o in combinazione ad altro software di servizio, e nel valutare se il comportamento del software rispetta i requisiti.
	\item[Commit:] operazione che si effettua quando si copiano le modifiche fatte su file locali nella corrispondente directory remota del repository.
	\item[CSS:] acronimo inglese significante Cascading Style Sheets, detti semplicemente fogli di stile, vengono usati per definire la rappresentazione di documenti HTML e XHTML.
\end{description}
\bigskip
\Huge D \bigskip
\hrule
\smallskip
\normalsize
\begin{description}
	\item[Database:] archivio strutturato in modo tale da consentire la gestione de dati (l'inserimento, la ricerca, la cancellazione ed il loro aggiornamento) da parte di applicazioni software. Informalmente e impropriamente, la parola "database" viene spesso usata come abbreviazione dell'espressione Database Management System (DBMS), che si riferisce a una vasta categoria di sistemi software che consentono la creazione e la manipolazione efficiente di database.
	\item[Diagramma use-case:] diagramma dedicato alla descrizione delle funzioni o servizi offerti da un sistema, così come sono percepiti e utilizzati dagli attori che interagiscono col sistema stesso.
\end{description}
\bigskip
\Huge E \bigskip
\hrule
\smallskip
\normalsize
\begin{description}
	\item[Estensione:] breve sequenza di caratteri alfanumerici aggiunti dopo il nome di un file e separati da quest'ultimo da un punto. L'estensione permette all'utente di un computer, ma anche ad alcuni programmi, di distinguere tra i vari formati di file.
\end{description}
\bigskip
\Huge F \bigskip
\hrule
\smallskip
\normalsize
\begin{description}
	\item[Forum:] struttura informatica contenente discussioni e messaggi scritti dagli utenti. Viene utilizzato per mettere in comunicazione i componenti del team e permette loro di reperire informazioni riguardanti il progetto.
	\item[Framework:] struttura di supporto su cui un software può essere organizzato e progettato. Alla base di un framework c'è sempre una serie di librerie di codice utilizzabili con uno o più linguaggi di programmazione, spesso corredate da una serie di strumenti di supporto allo sviluppo del software ideati per aumentare la velocità di sviluppo del prodotto finito.
\end{description}
\bigskip
\Huge H \bigskip
\hrule
\smallskip
\normalsize
\begin{description}
	\item[HTML:] acronimo inglese significante Hyper Text Mark-Up Language; linguaggio di formattazione usato per descrivere i documenti ipertestuali disponibili nel World Wide Web. Tutti i siti web sono scritti in HTML, codice che viene letto ed elaborato dal browser, il quale genera la pagina che viene visualizzata sullo schermo del computer.
\end{description}
\bigskip
\Huge I \bigskip
\hrule
\smallskip
\normalsize
\begin{description}
	\item[Indentazione:] inserimento di una certa quantità di spazio vuoto all'inizio di una riga di testo.
	\item[Intestazione:] porzione di testo che si trova all'inizio di un file sorgente. Contiene solitamente l'autore del file, la versione e altre informazioni utili a descriverlo.
\end{description}
\bigskip
\Huge K \bigskip
\hrule
\smallskip
\normalsize
\begin{description}
	\item[Kile:] editor per \LaTeX \space sviluppato per l'ambiente desktop KDE, ma disponibile per tutte le distribuzioni Linux. Per maggiori informazioni si prega di consultare il sito web ufficiale http://kile.sourceforge.net/
\end{description}
\bigskip
\Huge L \bigskip
\hrule
\smallskip
\normalsize
\begin{description}
	\item[Latex:] linguaggio usato per la preparazione e formattazione di testi. Fornisce mezzi per l'automazione della maggior parte della composizione tipografica, inclusa la numerazione, i riferimenti incrociati, tabelle e figure, organizzazione delle pagine, bibliografie e molto altro.
	\item[Layout:] impaginazione e la struttura grafica di un documento.
	\item[Linguaggio di Programmazione:] linguaggio formale, dotato di una sintassi e di una semantica ben definite, utilizzabile per il controllo del comportamento di una computer.
	\item[Link:] letteralmente indica un collegamento. In informatica è usato per indicare l'indirizzo web di una risorsa.
	\item[Login:] termine inglese per indicare la procedura di accesso ad un sistema o un'applicazione informatica.
	\item[Logout:] termine inglese per indicare la procedura di uscita da un sistema o un'applicazione informatica.
\end{description}
\bigskip
\Huge M \bigskip
\hrule
\smallskip
\normalsize
\begin{description}
	\item[Manutenzione:] processo correttivo e di sviluppo che avviene dopo il rilascio del prodotto finale.
	\item[Milestone:] traguardi intermedi nello svolgimento di un progetto che indicano il raggiungimento di obiettivi precedentemente stabiliti. Il raggiungimento delle milestones viene decretato tramite documenti ufficiali redatti dai vari componenti del team e grazie ad esse risulta possibile fornire una stima della bontà del progetto e del suo stato di avanzamento.
\end{description}
\bigskip
\Huge N \bigskip
\hrule
\smallskip
\normalsize
\begin{description}
	\item[NetBeans:] ambiente di sviluppo multi-linguaggio scritto interamente in Java. Per maggiori informazioni si può consultare il sito ufficiale all'indirizzo http://www.netbeans.org/
\end{description}
\bigskip
\Huge P \bigskip
\hrule
\smallskip
\normalsize
\begin{description}
	\item[Pattern:] nell'ingegneria del software è una soluzione progettuale generale a un problema ricorrente. Esso non è una libreria o un componente di software riusabile, quanto una descrizione o un modello da applicare per risolvere un problema che può presentarsi in diverse situazioni durante la progettazione e lo sviluppo del software.
	\item[Pdf:] il Portable Document Format, comunemente abbreviato Pdf, è un formato di file basato su un linguaggio di descrizione di pagina sviluppato da Adobe per rappresentare documenti in modo indipendente dall'hardware e dal software utilizzati per generarli o per visualizzarli. Ogni documento redatto dal team QuiXoft viene esportato in questo formato.
	\item[Plugin:] programma non autonomo che interagisce con un altro programma per ampliarne le funzioni.
	\item[Portabilità:] adattamento o una modifica di un componente, volto a consentirne l'uso in un ambiente di esecuzione diverso da quello originale. In una più moderna accezione, si parla di portabilità di un componente software o di un documento intendendo che può essere utilizzato su diverse piattaforme senza apportare alcuna modifica.
\end{description}
\bigskip
\Huge R \bigskip
\hrule
\smallskip
\normalsize
\begin{description}
	\item[Repository:] punto centrale in cui i dati vengono immagazzinati e manutenuti. Può essere un punto in cui sono localizzati database o file multipli per la distribuzione via rete, oppure una collocazione direttamente accessibile dall'utente. Utilizzato principalmente in un contesto di sviluppo concorrente.
	\item[Ridondanza:] presenza di ripetizioni e duplicazioni all'interno della medesima struttura. Può essere vista in accezione negativa se rilevata all'interno di un database (duplicazione inutile di dati) oppure all'interno di un file sorgente (ereditarietà mal gestita, per esempio).
	\item[Ruby:] linguaggio di programmazione completamente a oggetti ed interpretato.
	\item[Ruby on Rails:] framework open source per applicazioni web scritto in Ruby, la cui architettura è fortemente ispirata al paradigma Model-View-Controller (MVC). I suoi obiettivi sono la semplicità e la possibilità di sviluppare applicazioni di concreto interesse con meno codice rispetto ad altri framework. Il tutto con necessità di configurazione minimale.
\end{description}
\bigskip
\Huge S \bigskip
\hrule
\smallskip
\normalsize
\begin{description}
	\item[Server:] componente informatica che fornisce servizi ad altre componenti (tipicamente chiamate client) attraverso una rete. Il termine server può essere riferito sia alla componente software che eroga il servizio sia alla componente hardware.
	\item[Servizio web:] sistema software progettato per supportare l'interoperabilità tra diversi elaboratori su di una medesima rete. Caratteristica fondamentale di un servizio web è quella di offrire un'interfaccia software utilizzando la quale altri sistemi possono interagire con il servizio web stesso attivando le operazioni descritte nell'interfaccia tramite appositi messaggi.
	\item[Sorgente:] insieme di istruzioni appartenenti ad un determinato linguaggio di programmazione, utilizzato per realizzare un programma per computer.
	\item[Subversion:] sistema di controllo versione, noto anche come Svn.
\end{description}
\bigskip
\Huge T \bigskip
\hrule
\smallskip
\normalsize
\begin{description}
	\item[Template:] documento o programma dove, come in un foglio semicompilato cartaceo, su una struttura generica o standard esistono spazi temporaneamente "bianchi" da riempire successivamente.
	\item[Ticket:] assegnazione di un determinato compito ad un componente del team di sviluppo.
	\item[Top-down:] visione generale del sistema senza scendere nel dettaglio di alcuna delle sue parti. Ogni parte del sistema è successivamente rifinita aggiungendo maggiori dettagli dalla progettazione. Ogni nuova parte così ottenuta può quindi essere nuovamente rifinita, specificando ulteriori dettagli finché la specifica completa è sufficientemente dettagliata da validare il modello. Il modello top-down è spesso progettato con l'ausilio di scatole nere che semplificano il riempimento ma non consentono di capirne il meccanismo elementare.
\end{description}
\bigskip
\Huge U \bigskip
\hrule
\smallskip
\normalsize
\begin{description}
	\item[UML:] acronimo inglese per Unified Modeling Language, "linguaggio di modellazione unificato". E' un linguaggio di modellazione e specifica basato sul paradigma object-oriented. Un modello UML è costituito da una collezione organizzata di diagrammi correlati, costruiti componendo elementi grafici (con significato formalmente definito), elementi testuali formali, ed elementi di testo libero. Ha una semantica molto precisa e un grande potere descrittivo.
\end{description}
\bigskip
\Huge V \bigskip
\hrule
\smallskip
\normalsize
\begin{description}
	\item[Versionamento:] termine che sta ad indicare il controllo versione, la gestione cioè di versioni multiple di un insieme di informazioni.
\end{description}
\bigskip
\Huge W \bigskip
\hrule
\smallskip
\normalsize
\begin{description}
	\item[W3C:] abbreviazione di World Wide Web Consortium, associazione con lo scopo di migliorare i protocolli e linguaggi per il WWW e di aiutare il Web a sviluppare tutte le sue potenzialità. Mette a disposizione degli ottimi strumenti di validazione per pagine web.
\end{description}
\bigskip
\Huge X \bigskip
\hrule
\smallskip
\normalsize
\begin{description}
	\item[XHTML:] acronimo inglese per eXtensible HyperText Markup Language; successore diretto e versione più aggiornata dell'HTML
\end{description}
\modifiche
\end{document}
