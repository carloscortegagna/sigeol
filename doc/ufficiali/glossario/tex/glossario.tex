\documentclass[11pt,a4paper]{article}
\usepackage{amsmath}
\usepackage{amsfonts}
\usepackage{amssymb}
\usepackage{fancyhdr}
\usepackage{lastpage}
\usepackage{graphicx}
\usepackage{ucs}
\usepackage[utf8x]{inputenc}
\usepackage[italian]{babel}

\renewcommand{\headrulewidth}{0.6pt}
\renewcommand{\footrulewidth}{0.6pt}
% impostazione dello stile per le pagine interne del documento
\lhead{QuiXoft - Progetto SIGEOL}
\chead{}
\rhead{\includegraphics[scale=0.15]{logo.png} }
\lfoot{Glossario v0.1.0}
\cfoot{}
\rfoot{\thepage \ di \pageref{LastPage}}
% ridefinizione dello stile plain per il frontespizio
\fancypagestyle{plain}{
\fancyhf
}
% impostazione dello stile per l'indice
\fancypagestyle{indice}{
\lhead{\leftmark}
\chead{}
\rhead{\includegraphics[scale=0.15]{logo.png}}
\lfoot{Glossario v0.1.0}
\cfoot{}
\rfoot{}
}
\headheight = 46pt
%definizione del comando "\modfiche" per la creazione del diario delle modifiche
\newcommand{\modifiche} 
{
\newpage
\begin{center}
\textbf{Diario delle modifiche} \\
\bigskip
\begin{tabular}{|c|c|p{0.55\textwidth}|}
\hline
\textsc{Data} & \textsc{Versione} & \textsc{Modifica} \\
\hline
\hline
\textit{02-12-2008} & 0.1.0 & Prima bozza del glossario \\
\hline
\end{tabular}
\end{center}
}
%definizione del comando "\info" per la creazione delle informazioni del documento
\newcommand{\info} {
\bigskip
\begin{tabbing}
	\hspace*{0.3\textwidth} \= \hspace*{0.5\textwidth} \kill
	\parbox{0.3\textwidth}{\textbf{Verifica: }} \> \parbox{0.5\textwidth}{Freo Matteo} \\
	\parbox{0.3\textwidth}{\textbf{Approvazione: }} \> \parbox{0.5\textwidth}{Grosselle Alessandro} \\
	\parbox{0.3\textwidth}{\textbf{Stato: }} \> \parbox{0.5\textwidth}{Preliminare} \\
	\parbox{0.3\textwidth}{\textbf{Uso: }} \> \parbox{0.5\textwidth}{Interno} \\
	\parbox{0.3\textwidth}{\textbf{Distribuzione: }} \> \parbox{0.5\textwidth}{QuiXoft} \\
\end{tabbing}
}
%definizione del comando "\frontespizio" per la creazione del frontespizio
\newcommand{\frontespizio} {
\thispagestyle{plain}
\title{\begin{Huge}\textsc{Progetto SIGEOL}\end{Huge} \\ \textit{Glossario \\ v0.1.0}}
\author{Redazione: Carlo Scortegagna}
\maketitle
\medskip
\begin{center}
\includegraphics[scale=0.5]{logo.png} \\
\textit{quixoft.sol@gmail.com}
\end{center}
\medskip
\info
\newpage
}
%definizione del comando "\indice" per la creazione dell'indice
\newcommand{\indice} {
\thispagestyle{indice}
\tableofcontents
\newpage
}
\pagestyle{fancy}
\begin{document}
\frontespizio
\setcounter{page}{1}
\smallskip
\flushleft \Huge A \bigskip
\hrule
\smallskip
\normalsize
\begin{description}
	\item[Ambiente di sviluppo:] spiegazione definizione.
	\item[Amministratore:] spiegazione definizione.
	\item[Analista] spiegazione definizione.
\end{description}
\bigskip
\Huge C \bigskip
\hrule
\smallskip
\normalsize
\begin{description}
	\item[Commit:] spiegazione definizione.
\end{description}
\bigskip
\Huge E \bigskip
\hrule
\smallskip
\normalsize
\begin{description}
	\item[Estensione:] spiegazione definizione.
\end{description}
\bigskip
\Huge F \bigskip
\hrule
\smallskip
\normalsize
\begin{description}
	\item[Forum:] spiegazione definizione.
\end{description}
\bigskip
\Huge C \bigskip
\hrule
\smallskip
\normalsize
\begin{description}
	\item[Indentazione:] spiegazione definizione.
	\item[Intestazione:] spiegazione definizione.
\end{description}
\bigskip
\Huge L \bigskip
\hrule
\smallskip
\normalsize
\begin{description}
	\item[Latex:] spiegazione definizione.
	\item[Layout:] spiegazione definizione.
	\item[Linguaggio di Programmazione:] spiegazione definizione.
\end{description}
\bigskip
\Huge M \bigskip
\hrule
\smallskip
\normalsize
\begin{description}
	\item[Manutenzione:] spiegazione definizione.
	\item[Milestone:] spiegazione definizione.
\end{description}
\bigskip
\Huge P \bigskip
\hrule
\smallskip
\normalsize
\begin{description}
	\item[Pdf:] spiegazione definizione.
	\item[Portabilità:] spiegazione definizione.
	\item[Progettista:] spiegazione definizione.
	\item[Programmatore:] spiegazione definizione.
\end{description}
\bigskip
\Huge R \bigskip
\hrule
\smallskip
\normalsize
\begin{description}
	\item[Repository:] spiegazione definizione.
	\item[Responsabile:] spiegazione definizione.
	\item[Ruby:] spiegazione definizione.
\end{description}
\bigskip
\Huge S \bigskip
\hrule
\smallskip
\normalsize
\begin{description}
	\item[Sigeol:] spiegazione definizione.
	\item[Software:] spiegazione definizione.
	\item[Sorgente:] spiegazione definizione.
	\item[Subversion:] spiegazione definizione.
\end{description}
\bigskip
\Huge T \bigskip
\hrule
\smallskip
\normalsize
\begin{description}
	\item[Template:] spiegazione definizione.
	\item[Ticket:] spiegazione definizione.
	\item[Tracciabilità:] spiegazione definizione.
\end{description}
\bigskip
\Huge V \bigskip
\hrule
\smallskip
\normalsize
\begin{description}
	\item[Verificatore:] spiegazione definizione.
	\item[Versionamento:] spiegazione definizione.
	\item[Versione:] spiegazione definizione.
\end{description}
\bigskip
\modifiche
\end{document}
