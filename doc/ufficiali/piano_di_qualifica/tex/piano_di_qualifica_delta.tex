\documentclass[11pt,a4paper]{article}
\usepackage{amsmath}
\usepackage{amsfonts}
\usepackage{amssymb}
\usepackage{fancyhdr}
\usepackage{lastpage}
\usepackage{graphicx}
\usepackage{ucs}
\usepackage[utf8x]{inputenc}
\usepackage[italian]{babel}
\usepackage[colorlinks=true,linkcolor=black]{hyperref}

\renewcommand{\headrulewidth}{0.6pt}
\renewcommand{\footrulewidth}{0.6pt}
% impostazione dello stile per le pagine interne del documento
\lhead{\leftmark}
\chead{}
\rhead{\includegraphics[scale=0.15]{logo.png} }
\lfoot{Piano di Qualifica (Delta)}
\cfoot{}
\rfoot{\thepage \ di \pageref{LastPage}}
% ridefinizione dello stile plain per il frontespizio
\fancypagestyle{plain}{
\fancyhf
}
% impostazione dello stile per l'indice
\fancypagestyle{indice}{
\lhead{\leftmark}
\chead{}
\rhead{\includegraphics[scale=0.15]{logo.png}}
\lfoot{Piano di Qualifica (Delta)}
\cfoot{}
\rfoot{}
}
\headheight = 46pt
%definizione del comando "\modfiche" per la creazione del diario delle modifiche
\newcommand{\modifiche} 
{
\newpage
\begin{center}
\textbf{Diario delle modifiche} \\
\bigskip
\begin{tabular}{|c|c|p{0.61\textwidth}|}
\hline
\textsc{Data} & \textsc{Versione} & \textsc{Modifica} \\
\hline
\hline
\textit{04-06-2009} & Delta & Aggiunte le descrizioni dei tool di misurazione delle metriche  \\
\hline
\end{tabular}
\end{center}
}
%definizione del comando "\info" per la creazione delle informazioni del documento
\newcommand{\info} {
\bigskip
\begin{tabbing}
	\hspace*{0.3\textwidth} \= \hspace*{0.5\textwidth} \kill
	\parbox{0.3\textwidth}{\textbf{Verifica: }} \> \parbox{0.5\textwidth}{Scarpa Davide} \\
	\parbox{0.3\textwidth}{\textbf{Approvazione: }} \> \parbox{0.5\textwidth}{Beggiato Andrea} \\
	\parbox{0.3\textwidth}{\textbf{Stato: }} \> \parbox{0.5\textwidth}{Formale} \\
	\parbox{0.3\textwidth}{\textbf{Uso: }} \> \parbox{0.5\textwidth}{Esterno} \\
	\parbox{0.3\textwidth}{\textbf{Distribuzione: }} \> \parbox{0.5\textwidth}{QuiXoft} \\
	\> \parbox{0.5\textwidth}{Rossi Francesca} \\
	\> \parbox{0.5\textwidth}{Vardanega Tullio} \\
	\> \parbox{0.5\textwidth}{Conte Renato} \\
\end{tabbing}
}
%definizione del comando "\frontespizio" per la creazione del frontespizio
\newcommand{\frontespizio} {
\thispagestyle{plain}
\title{\begin{Huge}\textsc{Progetto SIGEOL}\end{Huge} \\ \textit{Piano di Qualifica \\ (Delta)}}
\author{Redazione: Carlo Scortegagna, Barbiero Mattia}
\maketitle
\medskip
\begin{center}
\includegraphics[scale=0.5]{logo.png} \\
\textit{quixoft.sol@gmail.com} \\
\end{center}
\medskip
\info
\begin{center}
\textbf{Sommario} \\
Aggiornamento del Piano di Qualifica per il progetto ''SIGEOL'', contenente solamente le modifiche rispetto al documento consegnato alla Revisione di Qualifica.
\end{center}
\newpage
}
%definizione del comando "\indice" per la creazione dell'indice
\newcommand{\indice} {
\thispagestyle{indice}
\tableofcontents
\newpage
}
\pagestyle{fancy}
\begin{document}
\frontespizio
\indice
\setcounter{page}{1}
\section{Introduzione}
\subsection{Scopo del documento}
Lo scopo del presente documento è di notificare gli aggiornamenti che sono stati apportati rispetto al documento consegnato alla precedente revisione.

Tali modifiche rappresentano il proseguimento delle attività di verifica e validazione del materiale prodotto e la descrizione degli \underline{ambienti di prova} e di \underline{collaudo}.
\section{Resoconto delle attività di verifica}
Ogni risultato ottenuto dalle attività di verifica, validazione e qualifica dovrà essere attentamente elencato in questa sezione, in modo da assicurare che tutti i problemi e le relative soluzioni siano tracciate per garantire la massima qualità possibile del prodotto finale.
\subsection{Tracciamento componenti - requisiti}

\subsection{Dettaglio delle verifiche tramite test}
Per l'esecuzione delle attività di verifica tramite test verranno utilizzati i moduli e le classi rese disponibili dal framework Rails, e più precisamente ogni test di unità dovrà estendere la classe ActiveSupport::TestCase ed ogni test funzionale dovrà estendere la classe ActionController::TestCase.

\subsubsection{Unit Tests}
Nello sviluppo di un'applicazione tramite il framework Rails, i test di unità (unit test) sono specifici per la verifica degli elementi appartenenti al componente Model. Data la particolare importanza di questa componente in quanto gestisce la persistenza dei dati, il team QuiXoft ha scelto di effettuare gli unit tests attraverso l'uso di fixtures, ovvero con istanze reali di un determiato model opportunamente salvate in file con estensione \verb|yml| all'interno della directory \verb|test/fixtures|. Prima dell'esecuzione della verifica il database di test sarà inizializzato con i dati contenuti nelle fixtures che vengono utilizzate in quel particolare test.

Ogni classe che implementa un insieme di test per un particolare model dovrà essere denominata \verb|NomeModelTest| ed essere salvata su di un file chiamato \verb|nome_model_test.rb| all'interno della directory \verb|test/unit|.

Un esempio di unit test per il model \verb|address| e delle relative fixtures utilizzate è dato dalla seguente porzione di codice:

nel file \verb|address_test.rb|
\begin{verbatim}
class AddressTest < ActiveSupport::TestCase
  fixtures :addresses, :buildings

  def setup
    @a=Address.new
  end
  
  def test_correct_address
    @a.telephone="049-9050231"
    @a.street="Via Belzoni 6"
    @a.city="Villafranca veronese"
    assert @a.save
  end

  def test_destroy_address_in_building_address_id_must_nil
    addresses(:address_2).delete
    !assert(buildings(:building_1).address_id)
  end
end
\end{verbatim}

nel file \verb|adresses.yml|
\begin{verbatim}
address_2:
  city: Vicenza
  telephone: 049-9075393
  street: via rossini 9
\end{verbatim}

nel file \verb|buildings.yml|
\begin{verbatim}
building_1:
  name: Torre Archimede
  address: address_2
\end{verbatim}


Di seguito sono riportate le tabelle che riassumono l'intera campagna di test delle classi appartenenti al componente model.

\begin{center}
\textbf{Test su AcademicOrganization}
\begin{small}
\begin{tabular}[t]{|p{2.0cm}|p{4.0cm}|p{4.0cm}|c|}
\hline
\textsc{Id caso di prova} & \textsc{Descrizione} & \textsc{Obiettivo} & \textsc{Esito}\\ 
\hline 
\hline
 1.1 & 
 Istanza di un oggetto con attributi nulli.& 
 Il sistema deve riconoscere l'oggetto come non valido. & 
 \checkmark \\
\hline\hline
 2.1 & 
 Il contenuto dell'attributo number dell'oggetto d'istanza è 0.& 
 number può contenere solo valori interi compresi tra 1 e 4. Non essendo valido l'oggetto non dovrà essere salvato nel database.& 
 \checkmark \\
 \hline
 2.2 & 
 number contiene il valore 7.& 
 Lo stesso del caso di prova 2.1.& 
 \checkmark \\
 \hline
 2.3 & 
 number contiene il valore 3.& 
 number contiene un valore corretto e quindi non devono essere riscontrati errori di validazione su di esso.& 
 \checkmark \\
 \hline\hline
 3.1& 
 Il contenuto dell'attributo name dell'oggetto d'istanza è 12a34.& 
 name può contenere solo caratteri alfabetici. Non essendo valido l'oggetto non dovrà essere salvato nel database.& 
 \checkmark \\
 \hline
 3.2& 
 name contiene un valore valido.& 
 Non devono essere riscontrati errori di validazione su name.& 
 \checkmark \\
 \hline \hline
 4.1&
 Agli attributi name e number dell'oggetto da salvare nel db vengono assegnati gli stessi valori di una tupla già presente.&
 Non devono esistere più tuple con la stessa coppia di valori contenuti in name e number. Non essendo valido l'oggetto non dovrà essere salvato nel database.&
 \checkmark \\
 \hline
 \end{tabular}
\end{small}
\end{center}

\begin{center}
\textbf{Test su Address}
\begin{small}
\begin{tabular}[t]{|p{2.0cm}|p{4.0cm}|p{4.0cm}|c|}
\hline
\textsc{Id caso di prova} & \textsc{Descrizione} & \textsc{Obiettivo} & \textsc{Esito}\\ 
\hline 
\hline
 5.1 & 
 Istanza di un oggetto con attributi nulli.& 
 Il sistema deve riconoscere l'oggetto come non valido. & 
 \checkmark \\
\hline\hline
 6.1& 
 Il contenuto dell'attributo telephone è 12345-123456.& 
 Il prefisso ha più di quattro cifre e quindi non è valido. L'oggetto per questo motivo non deve essere salvato nel db.&
 \checkmark \\
 \hline
 6.2 & 
 telephone contiene il valore 1234-12345.& 
 Il prefisso è corretto ma il numero contiene meno di sei cifre. L'oggetto per questo motivo non deve essere salvato nel db.& 
 \checkmark \\
 \hline
 6.3 & 
 telephone contiene il valore 1a23-12345.& 
 Contiene un carattere alfabetico. L'oggetto per questo motivo non deve essere salvato nel db.& 
 \checkmark \\
 \hline
 6.4 & 
 telephone rispetta l'espressione regolare.& 
 Non devono essere riscontrati errori di validazione su telephone.& 
 \checkmark \\
 \hline\hline
 7.1& 
 Cancellazione di un indirizzo associato ad uno user.& 
 Eliminato l'indirizzo, lo user associato deve avere il contenuto della chiave esterna(address\_id) a nil.& 
 \checkmark \\
 \hline \hline
 8.1&
 Cancellazione di un indirizzo associato ad un palazzo.&
 Eliminato l'indirizzo, il palazzo associato deve avere il contenuto della chiave esterna(address\_id) a nil.&
 \checkmark \\
 \hline \hline
 9.1&
 Creazione e salvataggio di un indirizzo valido.&
 L'oggetto deve essere salvato nel database.& 	
 \checkmark \\
 \hline 
 \end{tabular}
\end{small}
\end{center}

\newpage 
\begin{center}
\textbf{Test su BooleanConstraint}
\begin{small}
\begin{tabular}[t]{|p{2.0cm}|p{4.0cm}|p{4.0cm}|c|}
\hline
\textsc{Id caso di prova} & \textsc{Descrizione} & \textsc{Obiettivo} & \textsc{Esito}\\ 
\hline 
\hline
 10.1 & 
 Istanza di un oggetto con attributi nulli.& 
 Il sistema deve riconoscere l'oggetto come non valido. & 
 \checkmark \\
\hline\hline
 11.1& 
 L'attributo bool dell'oggetto d'istanza contiene il valore 3.& 
 Il valore 3 deve essere convertito a false.&
 \checkmark \\
 \hline
 11.2 & 
 bool assume il valore 1& 
 Il valore 1 deve essere convertito a true.& 
 \checkmark \\
 \hline
 11.3 & 
 bool assume il valore 0.& 
 Il valore 0 deve essere convertito a false.& 
 \checkmark \\
 \hline\hline
 12.1 & 
 L'attributo isHard dell'oggetto d'istanza contiene il valore 11.& 
 isHard deve contenere solo valori interi compresi tra 0 e 10. L'oggetto per questo motivo non deve essere salvato nel db.& 
 \checkmark \\
 \hline
 12.2& 
 isHard contiene il valore negativo -1.& 
 Lo stesso del caso di prova 12.1& 
 \checkmark \\
 \hline
 12.3&
 isHard contiene il valore 1.1.&
 Lo stesso del caso di prova 12.1&
 \checkmark \\
 \hline
 12.4&
 isHard contiene il valore 0.&
 Non devono essere riscontrati errori di validazione su isHard.& 	
 \checkmark \\
 \hline 
 \end{tabular}
\end{small}
\end{center}

\begin{center}
\textbf{Test su Building}
\begin{small}
\begin{tabular}[t]{|p{2.0cm}|p{4.0cm}|p{4.0cm}|c|}
\hline
\textsc{Id caso di prova} & \textsc{Descrizione} & \textsc{Obiettivo} & \textsc{Esito}\\ 
\hline 
\hline
 13.1 & 
 Istanza di un oggetto con attributi nulli.& 
 Il sistema deve riconoscere l'oggetto come non valido. & 
 \checkmark \\
\hline\hline
 14.1& 
 L'attributo name dell'oggetto da salvare assume un valore uguale a quello di un oggetto già salvato.  & 
 Non devono esistere due palazzi con lo stesso nome. Quindi l'istanza del caso di prova non essendo valida non deve essere salvata nel db. &
 \checkmark \\
 \hline\hline
 15.1 & 
 Eliminazione di un oggetto di tipo Building.& 
 Non deve più esistere nel database il palazzo, l'indirizzo associato e tutte le classi appartenenti.& 
 \checkmark \\
 \hline
 \end{tabular}
 \end{small}
 \end{center}
\newpage
\begin{center}
\textbf{Test su Capability}
\begin{small}
\begin{tabular}[t]{|p{2.0cm}|p{4.0cm}|p{4.0cm}|c|}
\hline
\textsc{Id caso di prova} & \textsc{Descrizione} & \textsc{Obiettivo} & \textsc{Esito}\\ 
\hline 
\hline
 16.1 & 
 Istanza di un oggetto con attributi nulli.& 
 Il sistema deve riconoscere l'oggetto come non valido. & 
 \checkmark \\
\hline\hline
 17.1 & 
 L'attributo name dell'oggetto da salvare assume un valore uguale a quello di un oggetto già salvato.& 
 Una capability non può avere lo stesso nome di un'altra. Quindi l'istanza del caso di prova non essendo valida non deve essere salvata nel db. &
 \checkmark \\
 \hline 
 \end{tabular}
\end{small}
\end{center}

\begin{center}
\textbf{Test su Classroom}
\begin{small}
\begin{tabular}[t]{|p{2.0cm}|p{4.0cm}|p{4.0cm}|c|}
\hline
\textsc{Id caso di prova} & \textsc{Descrizione} & \textsc{Obiettivo} & \textsc{Esito}\\ 
\hline 
\hline
 18.1 & 
 Istanza di un oggetto con attributi nulli.& 
 Il sistema deve riconoscere l'oggetto come non valido. & 
 \checkmark \\
\hline\hline
 19.1& 
 L'attributo name dell'oggetto da salvare assume un valore uguale a quello di un oggetto già salvato. Inoltre entrambi possiedono lo stesso valore di building\_id(chiave esterna per building)& 
 In un stesso palazzo non possono esistere classi con lo stesso nome. L'istanza del caso di prova non essendo valida non deve essere salvata nel db.& 
 \checkmark \\
 \hline
 20.1 & 
 Eliminazione di un oggetto di tipo Classroom.& 
 Non devono più esistere nel database la classe e tutti i vincoli associati.& 
 \checkmark \\
 \hline
 \end{tabular}
\end{small}
\end{center}

\newpage
\begin{center}
\textbf{Test su Curriculum}
\begin{small}
\begin{tabular}[t]{|p{2.0cm}|p{4.0cm}|p{4.0cm}|c|}
\hline
\textsc{Id caso di prova} & \textsc{Descrizione} & \textsc{Obiettivo} & \textsc{Esito}\\ 
\hline 
\hline
 21.1 & 
 Istanza di un oggetto con attributi nulli.& 
 Il sistema deve riconoscere l'oggetto come non valido. & 
 \checkmark \\
\hline\hline
 22.1& 
 L'attributo name dell'oggetto da salvare assume un valore uguale a quello di un oggetto già salvato. Inoltre entrambi possiedono lo stesso valore di graduate\_course\_id(chiave esterna per graduate\_course)& 
 In un stesso corso di laurea non possono esistere due curricula con lo stesso nome. L'istanza del caso di prova non essendo valida non deve essere salvata nel db.& 
 \checkmark \\
 \hline\hline
 23.1 & 
 Un insegnamento(oggetto di tipo Teacher) è associato a due curricula. Cancellazione di uno dei due.& 
 Dopo l'operazione di eliminazione, l'insegnamento deve essere ancora presente nel db.& 
 \checkmark \\
 \hline
 23.2 & 
 Cancellazione dell'altro curriculum.& 
 Dopo l'operazione di eliminazione l'insegnamento non deve essere più presente nel db.& 
 \checkmark \\
 \hline\hline
 24.1 & 
 Eliminazione di un oggetto di tipo Curriculum associato ad un insegnamento.& 
 Non devono più esistere nel database il curriculum e l'insegnamento associato dato che non apparteneva ad altri curricula.& 
 \checkmark \\
 \hline
 \end{tabular}
\end{small}
\end{center}

\begin{center}
\textbf{Test su DidacticOffice}
\begin{small}
\begin{tabular}[t]{|p{2.0cm}|p{4.0cm}|p{4.0cm}|c|}
\hline
\textsc{Id caso di prova} & \textsc{Descrizione} & \textsc{Obiettivo} & \textsc{Esito}\\ 
\hline 
\hline
 25.1 & 
 Eliminazione di un oggetto di tipo DidacticOffice(Segreteria didattica).& 
 Non devono più esistere nel database la segreteria e lo user associato.& 
 \checkmark \\
 \hline
 \end{tabular}
\end{small}
\end{center}

\newpage
\begin{center}
\textbf{Test su ExpiryDate}
\begin{small}
\begin{tabular}[t]{|p{2.0cm}|p{4.0cm}|p{4.0cm}|c|}
\hline
\textsc{Id caso di prova} & \textsc{Descrizione} & \textsc{Obiettivo} & \textsc{Esito}\\ 
\hline 
\hline
 26.1 & 
 Istanza di un oggetto con attributi nulli.& 
 Il sistema deve riconoscere l'oggetto come non valido. & 
 \checkmark \\
\hline\hline
 27.1& 
 L'attributo date dell'oggetto d'istanza contiene una data precedente alla data di sistema.& 
 Per essere valida la data deve essere maggiore o uguale alla data di sistema. L'oggetto per questo motivo non deve essere salvato nel db.&  
 \checkmark \\
 \hline\hline
 28.1 & 
 L'attributo period dell'oggetto d'istanza contiene il valore decimale 1.1& 
 period deve contenere un intero compreso tra uno e quattro. L'oggetto per questo motivo non deve essere salvato nel db.& 
 \checkmark \\
 \hline
 28.2 & 
 period contiene il valore negativo -1.& 
 Lo stesso del caso di prova 28.1.& 
 \checkmark \\
 \hline
 28.3 & 
 period contiene il valore 11.& 
 Lo stesso del caso di prova 28.1.& 
 \checkmark \\
 \hline
 28.4 & 
 periodo contiene il valore 6.& 
 Non devono essere riscontrati errori di validazione su period.& 
 \checkmark \\
 \hline
 \end{tabular}
\end{small}
\end{center}

\newpage
\begin{center}
\textbf{Test su GraduateCourse}
\begin{small}
\begin{tabular}[t]{|p{2.0cm}|p{4.0cm}|p{4.0cm}|c|}
\hline
\textsc{Id caso di prova} & \textsc{Descrizione} & \textsc{Obiettivo} & \textsc{Esito}\\ 
\hline 
\hline
 29.1 & 
 Istanza di un oggetto con attributi nulli.& 
 Il sistema deve riconoscere l'oggetto come non valido. & 
 \checkmark \\
\hline\hline
 30.1& 
 L'attributo duration dell'oggetto d'istanza contiene un valore negativo pari a -1.& 
 Per essere valido duration deve contenere un intero compreso tra uno e sei. L'oggetto per questo motivo non deve essere salvato nel db.&  
 \checkmark \\
 \hline
 30.2 & 
 duration contiene il valore decimale 1.1.& 
 Lo stesso del caso di prova 30.1.& 
 \checkmark \\
 \hline
 30.3 & 
 duration contiene il valore 7.& 
 Lo stesso del caso di prova 30.1.& 
 \checkmark \\
 \hline
 30.4 & 
 duration contiene il valore 6.& 
 Non devono essere riscontrati errori di validazione su duration.&
 \checkmark \\ 
 \hline\hline
 31.1 & 
 Eliminazione di un corso di laurea(GraduateCourse) associato a due curricula. Entrambi sono associati ad uno stesso insegnamento(Teaching).& 
 Non devono più esistere nel database il corso di laurea, i due curricula e l'insegnamento associato.&
 \checkmark \\ 
 \hline 
 \end{tabular}
\end{small}
\end{center}

\newpage
\begin{center}
\textbf{Test su Period}
\begin{small}
\begin{tabular}[t]{|p{2.0cm}|p{4.0cm}|p{4.0cm}|c|}
\hline
\textsc{Id caso di prova} & \textsc{Descrizione} & \textsc{Obiettivo} & \textsc{Esito}\\ 
\hline 
\hline
 32.1 & 
 Istanza di un oggetto con attributi nulli.& 
 Il sistema deve riconoscere l'oggetto come non valido. & 
 \checkmark \\
\hline\hline
 33.1& 
 L'attributo year dell'oggetto d'istanza contiene un valore negativo pari a -1.& 
 Per essere valido year deve contenere un intero compreso tra uno e sei. L'oggetto per questo motivo non deve essere salvato nel db.&  
 \checkmark \\
 \hline
 33.2 & 
 year contiene il valore decimale 1.2.& 
 Lo stesso del caso di prova 33.1.& 
 \checkmark \\
 \hline
 33.3 & 
 year contiene il valore 7.& 
 Lo stesso del caso di prova 33.1.& 
 \checkmark \\
 \hline
 33.4 & 
 year contiene il valore 6.& 
 Non devono essere riscontrati errori di validazione su year.&
 \checkmark \\ 
 \hline\hline
 34.1& 
 L'attributo subperiod dell'oggetto d'istanza contiene un valore negativo pari a -1.& 
 Per essere valido subperiod deve contenere un intero compreso tra uno e quattro. L'oggetto per questo motivo non deve essere salvato nel db.&  
 \checkmark \\
 \hline
 34.2 & 
 subperiod contiene il valore decimale 1.2.& 
 Lo stesso del caso di prova 34.1.& 
 \checkmark \\
 \hline
 34.3 & 
 subperiod contiene il valore 5.& 
 Lo stesso del caso di prova 34.1.& 
 \checkmark \\
 \hline
 34.4 & 
 subperiod contiene il valore 4.& 
 Non devono essere riscontrati errori di validazione su subperiod.&
 \checkmark \\ 
 \hline\hline
 35.1 & 
 A subperiod ed year vengono assegnati gli stessi valori di un oggetto già salvato.& 
 Non possono esistere due periodi con gli stessi valori di subperiod ed year. L'istanza del caso di prova non essendo valida non deve essere salvata nel db.& 
\checkmark \\ 
 \hline
 \end{tabular}
\end{small}
\end{center}

\newpage
\begin{center}
\textbf{Test su QuantityConstraint}
\begin{small}
\begin{tabular}[t]{|p{2.0cm}|p{4.0cm}|p{4.0cm}|c|}
\hline
\textsc{Id caso di prova} & \textsc{Descrizione} & \textsc{Obiettivo} & \textsc{Esito}\\ 
\hline \hline
 36.1 & 
 Istanza di un oggetto con attributi nulli.& 
 Il sistema deve riconoscere l'oggetto come non valido. & 
 \checkmark \\
 \hline \hline
 37.1 & 
 L'attributo isHard dell'oggetto d'istanza contiene il valore 11.& 
 isHard deve contenere solo valori interi compresi tra 0 e 10. L'oggetto per questo motivo non deve essere salvato nel db.& 
 \checkmark \\
 \hline
 37.2& 
 isHard contiene il valore -1.& 
 Lo stesso del caso di prova 12.1& 
 \checkmark \\
 \hline
 37.3&
 isHard contiene il valore 1.1.&
 Lo stesso del caso di prova 12.1&
 \checkmark \\
 \hline
 37.4&
 isHard contiene il valore 0.&
 Non devono essere riscontrati errori di validazione su isHard.& 	
 \checkmark \\ 
 \hline\hline
 38.1 & 
 L'attributo quantity dell'oggetto d'istanza contiene il valore 1001.& 
 quantity deve contenere solo valori interi compresi tra 1 e 1000. L'oggetto per questo motivo non deve essere salvato nel db.& 
 \checkmark \\
 \hline
 38.2 & 
 quantity contiene il valore -1.& 
 Lo stesso del caso di prova 38.1& 
 \checkmark \\
 \hline
 38.3 & 
 quantity contiene il valore 1.1.& 
 Lo stesso del caso di prova 38.1& 
 \checkmark \\
 \hline
 38.4 & 
 quantity contiene il valore 1.& 
 Non devono essere riscontrati errori di validazione su quantity.& 
 \checkmark \\
 \hline
 \end{tabular}
\end{small}
\end{center}

\newpage
\begin{center}
\textbf{Test su TeacherMailer}
\begin{small}
\begin{tabular}[t]{|p{2.0cm}|p{4.0cm}|p{4.0cm}|c|}
\hline
\textsc{Id caso di prova} & \textsc{Descrizione} & \textsc{Obiettivo} & \textsc{Esito}\\ 
\hline \hline
 39.1 & 
 Creazione di una mail contenente le istruzioni per l'attivazione di uno User tramite il metodo activate\_teacher(sender, receiver). Sender e Receiver sono due oggetti di tipo User opportunamente creati che corrispondono al mittente e al destinatario.& 
 L'indirizzo del destinatario della mail deve essere uguale al contenuto dell'attributo mail di receiver.& 
 \checkmark \\
 \hline
 39.2 & 
 Stessa descrizione per il caso di prova 39.1.&
 L'indirizzo del mittente della mail deve essere uguale al contenuto dell'attributo mail di sender.& 
 \checkmark \\
 \hline
 39.3& 
 Stessa descrizione per il caso di prova 39.1.& 
 Il subject della mail deve essere il seguente: Creazione account SIGEOL.& 
 \checkmark \\
 \hline
 39.4&
 Stessa descrizione per il caso di prova 39.1.&
 L'url necessario all'attivazione dell'account, presente nel corpo della mail, deve essere originato correttamente.&
 \checkmark \\
 \hline 
 \end{tabular}
\end{small}
\end{center}

\begin{center}
\textbf{Test su Teacher}
\begin{small}
\begin{tabular}[t]{|p{2.0cm}|p{4.0cm}|p{4.0cm}|c|}
\hline
\textsc{Id caso di prova} & \textsc{Descrizione} & \textsc{Obiettivo} & \textsc{Esito}\\ 
\hline \hline
 40.1 & 
 Aggiornamento di un oggetto di tipo Teacher lasciando degli attributi nulli.& 
 Il sistema deve riconoscere l'oggetto come non valido e quindi non aggiornare la corrispondente tupla nel database.& 
 \checkmark \\
 \hline \hline
 41.1 & 
 Eliminazione di un oggetto Teacher.& 
 Non devono più esistere nel database il docente, lo user associato e tutti i vincoli e le preferenze.& 
 \checkmark \\
 \hline
 \end{tabular}
\end{small}
\end{center}

\newpage
\begin{center}
\textbf{Test su Teaching}
\begin{small}
\begin{tabular}[t]{|p{2.0cm}|p{4.0cm}|p{4.0cm}|c|}
\hline
\textsc{Id caso di prova} & \textsc{Descrizione} & \textsc{Obiettivo} & \textsc{Esito}\\ 
\hline 
\hline
 42.1 & 
 Istanza di un oggetto con attributi nulli.& 
 Il sistema deve riconoscere l'oggetto come non valido. & 
 \checkmark \\
\hline\hline
 43.1& 
 cfu, labhours, classhours e studentsNumber contengono il valore 12.5.& 
 Tutti e quattro devono contenere un valore numerico intero positivo. L'oggetto non è valido e non deve essere salvato nel db.&  
 \checkmark \\
 \hline \hline
 44.1 & 
 cfu, labhours, classhours e studentsNumber contengono il valore -12.& 
 Lo stesso del caso di prova 43.1.& 
 \checkmark \\
 \hline \hline
 45.1 & 
 cfu assume il valore 21, labhours e classhours il valore 51 e studentsNumber il valore 1001.& 
 Tutti e quattro gli attributi superano di un'unità le soglie imposte e quindi deve essere riscontrato un errore di validazione su ognuno di essi.& 
 \checkmark \\
 \hline \hline
 46.1& 
 Creazione di un insegnamento associato ad un periodo non compatibile con l'organizzazione accademica del corso di laurea dell'insegnamento.& 
 L'oggetto d'istanza non essendo valido non deve essere salvato nel database.&
 \checkmark \\ 
 \hline\hline
 \end{tabular}
\end{small}
\end{center}

\newpage
\begin{center}
\textbf{Test su TemporalConstraint}
\begin{small}
\begin{tabular}[t]{|p{2.0cm}|p{4.0cm}|p{4.0cm}|c|}
\hline
\textsc{Id caso di prova} & \textsc{Descrizione} & \textsc{Obiettivo} & \textsc{Esito}\\ 
\hline 
\hline
 47.1 & 
 Istanza di un oggetto con attributi nulli.& 
 Il sistema deve riconoscere l'oggetto come non valido. & 
 \checkmark \\
\hline\hline
 48.1& 
 L'attributo day dell'oggetto d'istanza contiene il valore 6.& 
 day deve contenere solo valori interi compresi tra 1 e 5. L'oggetto per questo motivo non deve essere salvato nel db.& 
 \checkmark \\
 \hline
 48.2& 
 day contiene il valore 0.& 
 Lo stesso del caso di prova 48.1.& 
 \checkmark \\
 \hline
 48.3& 
 day contiene il valore 1.5.& 
 Lo stesso del caso di prova 48.1.&
 \checkmark \\
 \hline
 48.4& 
 day contiene il valore 1.& 
 Non deve essere riscontrato nessun errore di validazione su day.& 
 \checkmark \\
 \hline \hline
 49.1& 
 L'attributo isHard dell'oggetto d'istanza contiene il valore 11.& 
 isHard deve contenere solo valori interi compresi tra 0 e 10. L'oggetto per questo motivo non deve essere salvato nel db.& 
 \checkmark \\
 \hline
 49.2& 
 isHard contiene il valore -1.& 
 Lo stesso del caso di prova 49.1& 
 \checkmark \\
 \hline
 49.3&
 isHard contiene il valore 1.1.&
 Lo stesso del caso di prova 49.1&
 \checkmark \\
 \hline
 49.4&
 isHard contiene il valore 0.&
 Non devono essere riscontrati errori di validazione su isHard.& 	
 \checkmark \\ 
 \hline\hline
 50.1& 
 Il valore contenuto in startHour assume un valore più grande di quello in endHour& 
 Ovviamente l'ora d'inizio(startHour) dell'indisponibilità non può essere più grande dell'ora fine(endHour). Quindi non essendo valido l'oggetto non deve essere salvato nel db.&
 \checkmark \\ 
 \hline
 \end{tabular}
\end{small}
\end{center}

\newpage
\begin{center}
\textbf{Test su TimetableEntry}
\begin{small}
\begin{tabular}[t]{|p{2.0cm}|p{4.0cm}|p{4.0cm}|c|}
\hline
\textsc{Id caso di prova} & \textsc{Descrizione} & \textsc{Obiettivo} & \textsc{Esito}\\ 
\hline 
\hline
 51.1 & 
 Istanza di un oggetto con attributi nulli.& 
 Il sistema deve riconoscere l'oggetto come non valido. & 
 \checkmark \\
\hline\hline
 52.1& 
 L'attributo day dell'oggetto d'istanza contiene il valore 6.& 
 day deve contenere solo valori interi compresi tra 1 e 5. L'oggetto per questo motivo non deve essere salvato nel db.& 
 \checkmark \\
 \hline
 52.2& 
 day contiene il valore 0.& 
 Lo stesso del caso di prova 52.1.& 
 \checkmark \\
 \hline
 52.3& 
 day contiene il valore 1.5.& 
 Lo stesso del caso di prova 52.1.&
 \checkmark \\
 \hline
 52.4 & 
 day contiene il valore 1.& 
 Non deve essere riscontrato nessun errore di validazione su day.& 
 \checkmark \\
 \hline \hline
 53.1& 
 Il valore contenuto in startTime assume un valore più grande di quello in endTime& 
 Ovviamente l'ora d'inizio(startTime) di un riga dello schema d'orario non può essere più grande dell'ora fine(endHour). Quindi non essendo valido l'oggetto non deve essere salvato nel db.&
 \checkmark \\ 
 \hline\hline
 54.1& 
 Creazione di un oggetto TimetableEntry con gli stessi valori di day, StartTime, EndTime, classroom\_id e timetable\_id di un oggetto già salvato. & 
 Per evitare sovrapposizioni, in uno stesso schema d'orario identificato da timetable\_id non possono esistere righe con ugual giorno, ora di inizio, ora di fine e classe occupata. L'istanza appena creata non essendo valida non deve essere salvata nel db.&
 \checkmark \\ 
 \hline
 \end{tabular}
\end{small}
\end{center}

\newpage
\begin{center}
\textbf{Test su Timetable}
\begin{small}
\begin{tabular}[t]{|p{2.0cm}|p{4.0cm}|p{4.0cm}|c|}
\hline
\textsc{Id caso di prova} & \textsc{Descrizione} & \textsc{Obiettivo} & \textsc{Esito}\\ 
\hline 
\hline
 55.1 & 
 Istanza di un oggetto con attributi nulli.& 
 Il sistema deve riconoscere l'oggetto come non valido. & 
 \checkmark \\
\hline\hline
 56.1& 
 L'attributo year dell'oggetto d'istanza contiene il valore 20000-20.& 
 il contenuto di year non è valido. L'oggetto per questo motivo non deve essere salvato nel db.& 
 \checkmark \\
 \hline
 56.2& 
 year contiene 2007-08.& 
 year, per essere valido deve assumere nei primi quattro caratteri(2007) un valore successivo all'anno di sistema-1. L'anno di sistema nel caso di prova corrisponde a 2009. Non essendo valido, l'oggetto non deve essere salvato nel db& 
 \checkmark \\
 \hline
 56.3& 
 year contiene 2008-10. & 
 year per essere valido doveva avere nelle ultime due cifre i caratteri 09. Non essendo valido, l'oggetto non deve essere salvato nel db&
 \checkmark \\
 \hline
 56.4 & 
 year contiene 2008-09.& 
 Non deve essere riscontrato nessun errore di validazione su year.& 
 \checkmark \\
 \hline \hline
 57.1& 
 Creazione di un oggetto Timetable con gli stessi valori di year, graduate\_course\_id(chiave esterna per il corso di laurea) e period\_id(chiave esterna per il periodo) di un oggetto già salvato.& 
 Non possono esistere per uno stesso corso di laurea, periodo e anno più schemi d'orario.& 
 \checkmark \\ 
 \hline
 \end{tabular}
\end{small}
\end{center}

\newpage
\begin{center}
\textbf{Test su User}
\begin{small}
\begin{tabular}[t]{|p{2.0cm}|p{4.0cm}|p{4.0cm}|c|}
\hline
\textsc{Id caso di prova} & \textsc{Descrizione} & \textsc{Obiettivo} & \textsc{Esito}\\ 
\hline 
\hline
 58.1 & 
 Istanza di un oggetto con attributi nulli.& 
 Il sistema deve riconoscere l'oggetto come non valido. & 
 \checkmark \\
\hline\hline
 59.1& 
 L'attributo password dell'oggetto d'istanza è vuoto.& 
 password deve essere di almeno sei caratteri di tipo alfanumerico più il carattere'.'. L'oggetto per questo motivo non deve essere salvato nel db.& 
 \checkmark \\
 \hline
 59.2& 
 password contiene la stringa prova.& 
 prova contiene meno di sei caratteri. Non essendo valido, l'oggetto non deve essere salvato nel db& 
 \checkmark \\
 \hline
 59.3& 
 password contiene pro.va. & 
 Non deve essere riscontrato nessun errore di validazione su password.&
 \checkmark \\
 \hline \hline
 60.1& 
 L'attributo mail dell'oggetto d'istanza contiene la stringa prova?id=1@math.unipd.it& 
 mail contiene al suo interno un carattere non valido:?. Non essendo valido, l'oggetto non deve essere salvato nel db. & 
 \checkmark \\ 
 \hline
 60.2& 
 L'attributo mail contiene la stringa agrossel@math.unipd.it.& 
 Non deve essere riscontrato nessun errore di validazione su mail.& 
 \checkmark \\ 
 \hline \hline
 61.1& 
 Creazione di uno user con lo stesso valore dell'attributo mail di un altro oggetto già salvato.& 
 Non possono esistere due user con la stessa mail; quindi l'oggetto appena istanziato non è valido e non deve essere salvato.& 
 \checkmark \\ 
 \hline \hline
 62.1& 
 Creazione di uno user con password uguale ad alessandro.& 
 L'oggetto istanziato viene correttamente salvato nel database. Il contenuto di password deve essere uguale alla stringa alessandro crittografata con algoritmo SHA\-1.& 
 \checkmark \\ 
 \hline \hline
 63.1& 
 Creazione di uno user senza specificarlo.& 
 Un oggetto di tipo user deve appartenere o ad una segreteria o ad un insegnante; se non è associato a nulla l'istanza non è valida e non deve essere salvata nel db.& 
 \checkmark \\
 \hline
 \end{tabular}
\end{small}
\end{center}

\begin{center}
\begin{small}
\begin{tabular}[t]{|p{2.0cm}|p{4.0cm}|p{4.0cm}|c|}
\hline
\textsc{Id caso di prova} & \textsc{Descrizione} & \textsc{Obiettivo} & \textsc{Esito}\\ 
\hline 
\hline
64.1& 
 Al metodo authenticate vengono passati uno user ed una password non corretti.& 
 Il metodo deve ritornare un valore booleano pari a false.& 
 \checkmark \\ 
 \hline
 64.2& 
 Ora vengono passati uno user ed una password corretti.& 
 Il metodo deve ritornare un valore booleano pari a true.& 
 \checkmark \\
 \hline \hline
 65.1& 
 Eliminazione di un oggetto User.& 
 Non devono più esistere nel database lo user, l'indirizzo e il docente o la segreteria associata.& 
 \checkmark \\
 \hline \hline
 66.1& 
 Creazione di uno user associato a tutte le possibili capabilities(privilegi).& 
 I metodi che iniziano con manage\_ di un oggetto User ritornano true se l'istanza possiede quel determinato privilegio. In questo caso di prova ogni metodo deve ritornare true. & 
 \checkmark \\
 \hline
 67.1& 
 Creazione di uno user associato ad una segreteria didattica.& 
 Lo user appartiene ad una segreteria quindi il metodo own\_by\_didactic\_office? deve ritornare true e own\_by\_teacher? false. & 
 \checkmark \\
 \hline
 \end{tabular}
\end{small}
\end{center}

\begin{center}
\textbf{Test su Belong}
\begin{small}
\begin{tabular}[t]{|p{2.0cm}|p{4.0cm}|p{4.0cm}|c|}
\hline
\textsc{Id caso di prova} & \textsc{Descrizione} & \textsc{Obiettivo} & \textsc{Esito}\\ 
\hline
 69.1& 
 Creazione di un oggetto di tipo Belong che associa un curriculum ad un insegnamento già associati.& 
 Esistendo già l'associazione, l'istanza non deve essere salvata.&
 \checkmark \\
 \hline
\end{tabular}
\end{small}
\end{center}

\subsubsection{Functional Tests}
Nello sviluppo di un'applicazione tramite il framework Rails, i test funzionali (functional tests) sono specifici per la verifica degli elementi appartenenti alla componente Controller. Dato che gli unit tests sono stati effettuati tramite istanze reali, il team QuiXoft ha scelto di utilizzare la strategia dei Mock objects per la verifica delle azioni presenti nei controller.

Ogni classe che implementa un insieme di test per un particolare controller dovrà essere denominata \verb|NomeControllerTest| ed essere salvata su di un file chiamato \verb|nome_controller_controller_test.rb| all'interno della directory \verb|test/functional|.

Un esempio di functional test per il controller \verb|sessions| è dato dalla seguente porzione di codice: \\
nel file \verb|sessions_controller_test.rb|
\begin{verbatim}
class SessionsControllerTest < ActionController::TestCase

  test "Guest usa New" do
    get :new
    assert_template "new"
    assert_response :success
  end

  test "Immissione di email e password validi" do
    user = stub(:id => :an_id, :mail => "a_mail",
                :password => "a_password")
    User.stubs(:authenticate).returns(user)
    user.stubs(:active?).returns(true)
    post :create, :mail => user.mail, :password => user.password
    assert_equal session[:user_id], :an_id
    assert_redirected_to timetables_url
  end
end
\end{verbatim}

Di seguito sono riportate le tabelle che riassumono la campagna di test delle classi appartenenti al componente controller. 
Dato il gran numero di test effettuati(186 tests) si elencheranno solo i più significativi.

\newpage
\begin{center}
\textbf{Test su BuildingsController}
\begin{small}
\begin{tabular}[t]{|p{2.0cm}|p{4.0cm}|p{4.0cm}|c|}
\hline
\textsc{Id test} & \textsc{Descrizione} & \textsc{Obiettivo} & \textsc{Esito}\\ 
\hline
da 1 a 5&
Un utente non autentificato(guest) richiede l'esecuzione di azioni riservate ad utenti con privilegi.&
L'utente dev'essere indirizzato alla pagina di login.&
\checkmark \\
\hline
\hline
7&
Un utente autentificato con privilegio di modificare edifici richiede l'esecuzione dell'azione \verb|new| .&
L'utente dev'essere indirizzato alla pagina di creazione edificio.&
\checkmark \\
\hline
8&
Lo stesso tipo di utente del test 7 richiede l'esecuzione dell'azione \verb|edit| .&
L'utente dev'essere indirizzato alla pagina di modifica dei dati relativi ad un edificio.&
\checkmark \\
\hline
9&
Lo stesso tipo di utente del test 7 crea correttamente un edificio.&
L'utente dev'essere indirizzato alla pagina di amministazione edifici.&
\checkmark \\
\hline
da 10 a 11&
Lo stesso tipo di utente del test 7 tenta di creare un edificio con attributi non validi.&
L'utente non dev'essere indirizzato, deve rimanere nella pagina di creazione edificio.&
\checkmark \\
\hline
12&
Lo stesso tipo di utente del test 7 modifica correttamente un edificio.&
L'utente dev'essere indirizzato alla pagina di amministazione edifici.&
\checkmark \\
\hline
13&
Lo stesso tipo di utente del test 7 tenta di modificare un edificio con attributi non validi.&
L'utente non dev'essere indirizzato, deve rimanere nella pagina di creazione edificio.&
\checkmark \\
\hline
14&
Lo stesso tipo di utente del test 7 elimina correttamente un edificio.&
L'utente dev'essere indirizzato alla pagina di amministrazione edifici.& 
\checkmark \\
\hline
da 15 a 20&
Un utente autentificato privo del privilegio di modificare edifici richiede l'esecuzione di azioni riservate ad utenti che ne siano in possesso.&
L'utente dev'essere indirizzato alla pagina principale.&
\checkmark \\
\hline
\end{tabular}
\end{small}


\newpage
\begin{center}
\textbf{Test su ClassroomsController }
\begin{small}
\begin{tabular}[t]{|p{2.0cm}|p{4.0cm}|p{4.0cm}|c|}
\hline
\textsc{Id test} & \textsc{Descrizione} & \textsc{Obiettivo} & \textsc{Esito}\\ 
\hline 
\hline
 da 21 a 32 e da 47 a 51& 
 Un utente non autenticato(guest) richiede l'esecuzione di azioni riservate ad utenti con privilegi. & 
 Il guest deve essere indirizzato alla pagina di login. & 
 \checkmark \\
 \hline \hline
 33& 
 Un utente autenticato con il privilegio di poter modificare le aule richiede l'esecuzione dell'azione new.& 
 L'utente deve essere indirizzato alla pagina di creazione di un'aula. & 
 \checkmark \\
 \hline
 35& 
 Lo stesso tipo di utente del test 33 crea correttamente una classe.& 
 L'utente deve essere indirizzato alla pagina di amministazione aule. & 
 \checkmark \\
 \hline
 36& 
 Lo stesso tipo di utente del test 33 crea una classe con attributi non validi.& 
 L'utente non deve essere indirizzato ma rimanere nella pagina di creazione della classe.& 
 \checkmark \\
 \hline
 37& 
 Lo stesso tipo di utente del test 33 modifica correttamente una classe.& 
 L'utente deve essere indirizzato alla pagina di amministazione aule. & 
 \checkmark \\
 \hline
 38& 
 Lo stesso tipo di utente del test 33 modifica una classe con attributi non validi.& 
 L'utente non deve essere indirizzato ma rimanere nella pagina di creazione della classe.&
 \checkmark \\
 \hline
 39& 
 Lo stesso tipo di utente del test 33 elimina una classe.& 
 L'utente deve essere indirizzato alla pagina di amministazione aule.&
 \checkmark \\
 \hline
 44& 
 Lo stesso tipo di utente del test 33 crea un vincolo associato ad una determinata classe.& 
 Deve essere renderizzata la pagina di modifica aule.&
 \checkmark \\
 \hline
 45& 
 Lo stesso tipo di utente del test 33 elimina un vincolo associato ad una determinata classe.& 
 Deve essere renderizzata la pagina di modifica aule.&
 \checkmark \\
 \hline
 \end{tabular}
\end{small}
\end{center}

\newpage
\textbf{Test su CurriculumsController}
\begin{small}
\begin{tabular}[t]{|p{2.0cm}|p{4.0cm}|p{4.0cm}|c|}
\hline
\textsc{Id test} & \textsc{Descrizione} & \textsc{Obiettivo} & \textsc{Esito}\\ 
\hline
\hline
da 51 a 58&
Un utente non autentificato(guest) richiede l'esecuzione di azioni riservate ad utenti con privilegi.&
L'utente dev'essere indirizzato alla pagina di login.&
\checkmark \\
\hline
\hline
59&
Un utente autentificato con privilegio di modificare curricula richiede l'esecuzione dell'azione \verb|new| .&
L'utente dev'essere indirizzato alla pagina di creazione curriculum.&
\checkmark \\
\hline
60&
Lo stesso tipo di utente del test 59 richiede l'esecuzione dell'azione \verb|edit| .&
L'utente dev'essere indirizzato alla pagina di modifica dei dati relativi ad un curriculum.&
\checkmark \\
\hline
61&
Lo stesso tipo di utente del test 59 crea correttamente un curriculum.&
L'utente dev'essere indirizzato alla pagina di amministazione curricula.&
\checkmark \\
\hline
62&
Lo stesso tipo di utente del test 59 tenta di creare un curriculum con attributi non validi.&
L'utente non dev'essere indirizzato, deve rimanere nella pagina di creazione curriculum.&
\checkmark \\
\hline
63&
Lo stesso tipo di utente del test 59 modifica correttamente un curriculum.&
L'utente dev'essere indirizzato alla pagina di amministazione curricula.&
\checkmark \\
\hline
64&
Lo stesso tipo di utente del test 59 tenta di modificare un curriculum con attributi non validi.&
L'utente non dev'essere indirizzato, deve rimanere nella pagina di modifica curriculum.&
\checkmark \\
\hline
65&
Lo stesso tipo di utente del test 59 richiede l'esecuzione dell'azione \verb|edit_teachings| .&
L'utente dev'essere indirizzato alla pagina di amministrazione degli insegnamenti di un curriculum.&
\checkmark \\
\hline
da 66 a 68&
Lo stesso tipo di utente del test 59 modifica gli insegnamenti di un curriculum.&
L'utente dev'essere indirizzato alla pagina di amministrazione degli insegnamenti di un curriculum.&
\checkmark \\
\hline
69&
Lo stesso tipo di utente del test 59 elimina correttamente un curriculum.&
L'utente dev'essere indirizzato alla pagina di amministrazione curricula.&
\checkmark \\
\hline
\hline
da 74 a 80&
Un utente autentificato privo del privilegio di modificare curricula richiede l'esecuzione di azioni riservate ad utenti che ne siano in possesso.&
L'utente dev'essere indirizzato alla pagina principale.&
\checkmark \\
\hline
\end{tabular}
\end{small}

\newpage
\begin{center}
\textbf{Test su GraduateCoursesController }
\begin{small}
\begin{tabular}[t]{|p{2.0cm}|p{4.0cm}|p{4.0cm}|c|}
\hline
\textsc{Id test} & \textsc{Descrizione} & \textsc{Obiettivo} & \textsc{Esito}\\ 
\hline \hline 
 da 81 a 86& 
 Un utente non autenticato(guest) richiede l'esecuzione di azioni riservate ad utenti con privilegi. & 
 Il guest deve essere indirizzato alla pagina di login. & 
 \checkmark \\
 \hline \hline
 91 e 93& 
 Un utente di tipo segreteria(privilegio "didactic\_office\_required") crea correttamente un corso di laurea.& 
 L'utente deve essere indirizzato alla pagina di amministrazione dei corsi di laurea. & 
 \checkmark \\
 \hline
 92 e 94& 
 Lo stesso tipo di utente del test 91 crea un corso di laurea con attributi non validi.& 
 L'utente deve essere indirizzato alla pagina di amministrazione dei corsi di laurea riportando un errore nella creazione.& 
 \checkmark \\
 \hline \hline
 95& 
 Un utente autenticato con il privilegio di modificare i corsi di laurea("manage\_graduate\_courses\_ required") ne modifica uno.& 
 L'utente deve essere indirizzato alla pagina di amministrazione dei corsi di laurea riportando un messaggio di avvenuta modifica.& 
 \checkmark \\
 \hline
 96& 
 Lo stesso tipo di utente del test 95 modifica un suo corso di laurea invalidando qualche attributo.& 
 Deve essere renderizzata la pagina di modifica corso di laurea riportando quali errori son stati commessi. & 
 \checkmark \\
 \hline \hline
 97& 
 Un utente autenticato con i privilegi "manage\_graduate\_courses\_ required" e "didactic\_office\_required" elimina un suo corso di laurea.& 
 L'utente deve essere indirizzato alla pagine di amministrazione del corso di laurea.&
 \checkmark \\
 \hline \hline
 98& 
 Lo stesso tipo di utente del test 95 tenta di modificare un corso di laurea che non gli appartiene.& 
 L'utente deve essere indirizzato alla pagina principale riportando il seguente messaggio: "Non puoi modificare questo corso di laurea".&
 \checkmark \\
 \hline
\end{tabular}
\end{small}
\end{center}

\begin{center}
\begin{small}
\begin{tabular}[t]{|p{2.0cm}|p{4.0cm}|p{4.0cm}|c|}
\hline
\textsc{Id test} & \textsc{Descrizione} & \textsc{Obiettivo} & \textsc{Esito}\\
\hline \hline
100& 
 Lo stesso tipo di utente del test 97 tenta di eliminare un corso di laurea che non gli appartiene.& 
 L'utente deve essere indirizzato alla pagina principale riportando il seguente messaggio: "Non puoi modificare questo corso di laurea".&
 \checkmark \\
 \hline
 102& 
 Uno user che non è segreteria didattica(quindi non ha il privilegio "didactic\_office\_required") tenta di creare un nuovo corso di laurea.& 
 L'utente deve essere indirizzato alla pagina principale riportando il seguente messaggio: "Non possiedi i privilegi per effettuare questa operazione".&
 \checkmark \\
 \hline
 103& 
 Uno user che non è segreteria didattica(quindi non ha il privilegio "didactic\_office\_required") tenta di eliminare un corso di laurea.& 
 Lo stesso del test 102.&
 \checkmark \\
 \hline
 106& 
 Uno user che non possiede il privilegio di modifica dei corsi di laurea(quindi non ha il privilegio "manage\_graduate\_courses\_ required") tenta di modificare un corso di laurea.& 
 Lo stesso del test 102.&
 \checkmark \\
 \hline
\end{tabular}
\end{small}
\end{center}

\newpage
\textbf{Test su SessionsController}
\begin{small}
\begin{tabular}[t]{|p{2.0cm}|p{4.0cm}|p{4.0cm}|c|}
\hline
\textsc{Id test} & \textsc{Descrizione} & \textsc{Obiettivo} & \textsc{Esito}\\ 
\hline
\hline
108&
Un utente non autentificato(guest) richiede l'esecuzione dell'azione \verb|new| .&
L'utente non dev'essere indirizzato, deve rimanere nella pagina di login.&
\checkmark \\
\hline
109&
Un utente non autentificato(guest) inserisce correttamente username e password.&
L'utente dev'essere indirizzato alla pagina principale.&
\checkmark \\
\hline
110&
Un utente non autentificato(guest) inserisce username e/o password scorretti.&
L'utente non dev'essere indirizzato, deve rimanere nella pagina di login.&
\checkmark \\
\hline
\hline
111&
Un utente autentificato richiede l'esecuzione dell'azione \verb|destroy| .&
L'utente dev'essere indirizzato alla pagina di login.&
\checkmark \\
\hline
\end{tabular}
\end{small}

\newpage
\begin{center}
\textbf{Test su TeachersController }
\begin{small}
\begin{tabular}[t]{|p{2.0cm}|p{4.0cm}|p{4.0cm}|c|}
\hline
\textsc{Id test} & \textsc{Descrizione} & \textsc{Obiettivo} & \textsc{Esito}\\ 
\hline \hline 
 115 & 
 Un docente attiva correttamente il proprio user. & 
 L'utente deve essere indirizzato alla pagina principale riportando il seguente messaggio: "Account attivato correttamente". & 
 \checkmark \\
 \hline
 116& 
 Attivazione di un account già esistente& 
 L'utente deve essere indirizzato alla pagina principale riportando il seguente messaggio: "L'utente non esiste od è già attivo".& 
 \checkmark \\
 \hline \hline
 122& 
 Un utente autenticato con il privilegio di modificare docenti("manage\_teachers\_required") ne invita uno.& 
 L'utente deve essere indirizzato alla pagina principale riportando il seguente messaggio: "Docente invitato con successo".& 
 \checkmark \\
 \hline
 126& 
 Un utente autenticato con il privilegio di modificare privilegi("manage\_capabilities\_required") ne aggiunge uno ad un docente.& 
 L'utente deve essere indirizzato alla pagina di modifica privilegi.& 
 \checkmark \\
 \hline
 128& 
 Un utente dello stesso tipo del test 122 assegna ad un docente un corso di laurea.& 
 L'utente deve essere indirizzato alla pagina di modifica corsi di laurea.& 
 \checkmark \\
 \hline
 129& 
 Un utente dello stesso tipo del test 122 tenta di associare un docente che non appartiene a nessun suo corso di laurea ad un altro corso di laurea.& 
 L'utente deve essere indirizzato alla pagina principale riportando il seguente messaggio: "Questo docente non appartiene a nessun tuo corso di laurea".&
 \checkmark \\
 \hline
 130& 
 Lo stesso tipo di utente del test 126 tenta di modificare i privilegi di un docente che non appartiene a nessun suo corso di laurea.& 
 Lo stesso del test 129.&
 \checkmark \\
 \hline
 131& 
 Lo stesso tipo di utente del test 126 tenta di modificare i privilegi di un docente che non appartiene a nessun suo corso di laurea.& 
 Lo stesso del test 129.&
 \checkmark \\
 \hline
\end{tabular}
\end{small}
\end{center}

\begin{center}
\begin{small}
\begin{tabular}[t]{|p{2.0cm}|p{4.0cm}|p{4.0cm}|c|}
\hline
\textsc{Id test} & \textsc{Descrizione} & \textsc{Obiettivo} & \textsc{Esito}\\ 
\hline \hline 
133& 
 Uno user autenticato come docente(privilegio same\_teacher\_required) crea una preferenza valida.& 
 Deve essere renderizzata la pagina di modifica preferenza riportando nell'elenco quella appena creata.&
 \checkmark \\
 \hline
 134& 
 Lo stesso tipo di user del test 133 crea un vincolo valido.& 
 Deve essere renderizzata la pagina di modifica vincolo riportando nell'elenco quello appena creata.&
 \checkmark \\
 \hline
 135& 
 Lo stesso tipo di user del test 133 elimina una preferenza.& 
 Deve essere renderizzata la pagina di modifica preferenza eliminando nell'elenco quella appena cancellata.&
 \checkmark \\
 \hline
 136& 
 Lo stesso tipo di user del test 133 elimina un vincolo.& 
 Deve essere renderizzata la pagina di modifica vincolo eliminando nell'elenco quello appena cancellato.&
 \checkmark \\
 \hline
 140& 
 Lo stesso tipo di user del test 133 modifica i propri dati personali.& 
 L'utente deve essere indirizzato alla pagina principale riportando il seguente messaggio: "Dati personali aggiornati correttamente".&
 \checkmark \\
 \hline
 da 142 a 145& 
 Uno user autenticato senza il privilegio di modifica docenti(manage\_teachers\_ required) richiede l'esecuzione di azioni che ne richiedono il possesso.& 
 L'utente deve essere indirizzato alla pagina principale riportando il seguente messaggio: "Non possiedi i privilegi per effettuare questa operazione".&
 \checkmark \\
 \hline
 da 146 a 147& 
 Uno user autenticato senza il privilegio di modifica privilegi(manage\_capabilities\_ required) richiede l'esecuzione di azioni che ne richiedono il possesso.& 
 L'utente deve essere indirizzato alla pagina principale riportando il seguente messaggio: "Non possiedi i privilegi per effettuare questa operazione".&
 \checkmark \\
 \hline
\end{tabular}
\end{small}
\end{center}

\newpage
\textbf{Test su TeachingsController}
\begin{small}
\begin{tabular}[t]{|p{2.0cm}|p{4.0cm}|p{4.0cm}|c|}
\hline
\textsc{Id test} & \textsc{Descrizione} & \textsc{Obiettivo} & \textsc{Esito}\\ 
\hline
\hline
da 151 a 159&
Un utente non autentificato(guest) richiede l'esecuzione di azioni riservate ad utenti con privilegi.&
L'utente dev'essere indirizzato alla pagina di login.&
\checkmark \\
\hline
\hline
161&
Un utente autentificato con privilegio di modificare insegnamenti richiede l'esecuzione dell'azione \verb|new| .&
L'utente dev'essere indirizzato alla pagina di creazione insegnamento.&
\checkmark \\
\hline
162&
Lo stesso tipo di utente del test 161 richiede l'esecuzione dell'azione \verb|edit| .&
L'utente dev'essere indirizzato alla pagina di modifica dei dati relativi ad un insegnamento.&
\checkmark \\
\hline
163&
Lo stesso tipo di utente del test 161 crea correttamente un insegnamento.&
L'utente dev'essere indirizzato alla pagina di selezione docente.&
\checkmark \\
\hline
164&
Lo stesso tipo di utente del test 161 tenta di creare un insegnamento con attributi non validi.&
L'utente non dev'essere indirizzato, deve rimanere nella pagina di creazione insegnamento.&
\checkmark \\
\hline
165&
Lo stesso tipo di utente del test 161 richiede l'esecuzione dell'azione \verb|select_teacher| .&
L'utente dev'essere indirizzato alla pagina di selezione docente.&
\checkmark \\
\hline
166&
Lo stesso tipo di utente del test 161 assegna correttamente un docente ad un insegnamento.&
L'utente dev'essere indirizzato alla pagina di amministrazione insegnamenti.&
\checkmark \\
\hline
167&
Lo stesso tipo di utente del test 161 non assegna correttamente un docente ad un insegnamento.&
L'utente non dev'essere indirizzato, deve rimanere nella pagina di selezione docente.&
\checkmark \\
\hline
168&
Lo stesso tipo di utente del test 161 modifica correttamente un insegnamento.&
L'utente dev'essere indirizzato alla pagina di amministazione insegnamenti.&
\checkmark \\
\hline
\end{tabular}
\end{small}

\newpage
\begin{small}
\begin{tabular}[t]{|p{2.0cm}|p{4.0cm}|p{4.0cm}|c|}
\hline
169&
Lo stesso tipo di utente del test 161 tenta di modificare un insegnamento con attributi non validi.&
L'utente non dev'essere indirizzato, deve rimanere nella pagina di modifica insegnamento.&
\checkmark \\
\hline
170&
Lo stesso tipo di utente del test 161 elimina correttamente un insegnamento.&
L'utente dev'essere indirizzato alla pagina di amministrazione insegnamenti.&
\checkmark \\
\hline
\hline
da 174 a 181&
Un utente autentificato privo del privilegio di modificare curricula richiede l'esecuzione di azioni riservate ad utenti che ne siano in possesso.&
L'utente dev'essere indirizzato alla pagina principale.&
\checkmark \\
\hline
\hline
\end{tabular}
\end{small}

\newpage
\textbf{Test su UsersController}
\begin{small}
\begin{tabular}[t]{|p{2.0cm}|p{4.0cm}|p{4.0cm}|c|}
\hline
\textsc{Id test} & \textsc{Descrizione} & \textsc{Obiettivo} & \textsc{Esito}\\ 
\hline
\hline
da 182 a 183&
Un utente non autentificato(guest) richiede l'esecuzione di azioni riservate ad utenti autentificati.&
L'utente dev'essere indirizzato alla pagina di login.&
\checkmark \\
\hline
\hline
184&
Un utente autentificato richiede l'esecuzione dell'azione \verb|edit| .&
L'utente viene indirizzato alla pagina di modifica password.&
\checkmark \\
\hline
185&
Un utente autentificato modifica correttamente la password.&
L'utente viene indirizzato alla pagine principale.&
\checkmark \\
\hline
186&
Un utente autentificato non modifica correttamente la password.&
L'utente non viene indirizzato, deve rimanere nella pagina di modifica password.&
\checkmark \\
\hline
187&
Un utente autentificato richiede l'esecuzione dell'azione \verb|destroy| .&
L'utente viene indirizzato alla pagina di login.&
\checkmark \\
\hline
\end{tabular}
\end{small}
\end{center}


\subsection{Metriche}
Il progetto `Sigeol`, al termine della fase di programmazione, è stato testato con numerosi strumenti dedicati a Ruby on Rails con lo scopo di generare delle metriche precise che indichino il livello qualitativo del codice prodotto dal team QuiXoft.

E' stato scelto di utilizzare Metric\_Fu 1.0.2, raccolta di gemme per Ruby on Rails scaricabile da http://metric-fu.rubyforge.org/

Nei capitoli seguenti saranno analizzate una ad una tutte le metriche utilizzate:
\subsubsection{Rcov}
Lo strumento Rcov è stato usato per misurare la copertura dei test sul codice. Il team QuiXoft utilizzerà la metrica C0, ovvero sarà controllata la copertura di ogni istruzione, e quindi di ogni riga di codice dell'intero progetto.

La situazione di copertura dei test sul codice è riportato nel file \textit{coverage.pdf} allegato al presente documento.

\begin{large}INSERIRE I RISULTATI DI RCOV\end{large}
\subsubsection{Flay}
Qualsiasi progetto Rails dovrebbe seguire il più possibile il principio DRY: Don't Repeat Yourself.
Anche se questo deve essere inteso come principio generale, dovremmo anche evitare di avere ripetizioni all’interno del medesimo file.

Flay permette di tenere sotto controllo le duplicazioni, analizzando le “similarità strutturali” (branch, cicli, etc) presenti nel codice: se due parti di codice sono simili allora potrebbero essere buone candidate per un refactoring.

Il risultato di tale metrica di analisi si può consultare nel file \textit{flay.html} allegato al presente documento.

Analizzando i risultati si possono notare numerose ripetizioni: la maggior parte non sono tuttavia evitabili, in quanto diversi metodi all'interno degli stessi controller hanno un redirect alla stessa pagina, e per forza di cose il codice che effettua il redirect deve essere ripetuto.
Nonostante ciò, la quantità di ripetizioni non è stata ritenuta preoccupante dai membri del team QuiXoft, ne sono stati rilevati eventuali problemi per la futura manutenzione del codice.
\subsubsection{Flog}
Lo strumento Flog applica una metrica ABC al codice del progetto `Sigeol` al fine di misurarne la complessità.
La metrica ABC misura la distanza euclidea dall’origine nello spazio tridimensionale formato da:
\begin{itemize}
 \item Assignments
 \item Branch
 \item Condition
\end{itemize}
Più il codice risulta “lineare” minore sarà il valore di tale metrica ad esso applicato (e tenderà ad essere più gestibile). Al contrario, nel caso di codice ricco di cicli, sottocicli e diversi branch, mostrerebbe un valore elevato, indicando un codice, almeno in teoria, più soggetto a bachi.

Il risultato di tale metrica di analisi si può consultare nel file \textit{flog.html} allegato al presente documento.

La seguente tabella, messa a disposizione degli sviluppatori di Flog, indica la scala di valori con cui misurare la bontà dei risultati, prendendo in esame il punteggio per metodo (seconda colonna del file allegato):

\begin{center}
\begin{tabular}{|c|c|}
\hline
\textsc{Punteggio} & \textsc{Risultato} \\
\hline
\hline
\textit{0 - 10} & Ottimo  \\
\hline
\textit{11 - 20} & Buono  \\
\hline
\textit{21 - 40} & Abbastanza buono (da valutare un possibile refactoring)  \\
\hline
\textit{41 - 60} & Accettabile ma sconsigliato  \\
\hline
\textit{60 - 100} & Pericoloso  \\
\hline
\textit{100 o più} & Non accettabile  \\
\hline
\end{tabular}
\end{center}
Osservando i risultati ottenuti, si può notare che solamente 2 classi rientrano nella fascia 21-40, mentre tutte le altre possono considerarsi buone o ottime. Le 2 classi imputate sono teacher\_controller e classroom\_controller: ne è stato valutato un refactoring, ma si è deciso di mantenerle cosi come sono in quanto la loro complessità è stata ritenuta perfettamente accettabile dai membri del team QuiXoft.
\subsubsection{Saikuro}
Saikuro è uno strumento per misurare la complessità ciclomatica, metrica strutturale relativa al flusso di controllo di un programma che rappresenta la sua complessità logica, cioè lo sforzo per realizzarlo e comprenderlo.

Il risultato di tale metrica di analisi si può consultare nel file \textit{saikuro.html} allegato al presente documento.

La seguente tabella indica la scala di punteggi con cui valutare i risultati ottenuti:
\begin{center}
\begin{tabular}{|c|c|}
\hline
\textsc{Punteggio} & \textsc{Risultato} \\
\hline
\hline
\textit{0 - 10} & Ottimo  \\
\hline
\textit{11 - 20} & Buono  \\
\hline
\textit{21 - 40} & Sufficiente  \\
\hline
\textit{41 o più} & Insufficiente  \\
\hline
\end{tabular}
\end{center}
Esaminando il file allegato, si può notare come tutti i metodi rientrino nelle prime 2 categorie, indicando una limitata complessità ciclomatica, a tutto vantaggio della facilità di gestione e di manutenzione del codice del progetto SIGEOL.
\subsubsection{Reek}
Reek è uno strumeto di analisi del codice che ci dice se e dove compaiono pattern “sospetti”, come ad esempio:
\begin{itemize}
 \item metodi troppo lunghi
 \item classi troppo ampie
 \item nomi criptici
 \item liste di paramentri eccessivamente lunghe
 \item duplicazioni
\end{itemize}
Il risultato di tale metrica di analisi si può consultare nel file \textit{reek.html} allegato al presente documento.

Al termine della fase di programmazione molte parti di codice del progetto `Sigeol` sono state modificate e ottimizzate tenendo conto delle considerazioni nate consultando il risultato di Reek, ma molte delle segnalazioni ancora presenti nel file allegato sono state ritenute poco significative o addirittura in contraddizione con scelte fatte durante lo sviluppo: si è ritenuta quindi irrealizzabile l'idea di eliminare completamente tutti i problemi segnalati da questo strumento.
\subsubsection{Roodi}
Roodi è l’acronimo di “Ruby Object Oriented Design Inferometer”. E’ uno strumento che analizza il nostro codice relativamente a:
\begin{itemize}
 \item nomi delle classi
 \item nomi dei metodi
 \item complessità ciclomatica (sia a livello di metodi che di blocchi)
 \item blocchi “rescue” vuoti
 \item cicli del tipo “for”
 \item lunghezza dei metodi
\end{itemize}
Il risultato di tale metrica di analisi si può consultare nel file \textit{roodi.html} allegato al presente documento.

Roodi segnala parecchi metodi con lunghezza superiore alle 20 righe: il team QuiXoft ha ritenuto che tale valore sia piuttosto limitante, e per questo ha dato un peso relativamente basso a tale segnalazione. Il numero di righe per metodo non supera mai comunque le 40 righe, valore ritenuto un ottimo compromesso per quanto riguarda la leggibilità e la manutenibilità del codice.
\subsubsection{Churn}
Churn tiene traccia dei file soggetti a maggiori modifiche. Un file il cui contenuto cambia spesso è un sintomo di probabile necessità di refactoring: forse è necessario rivedere le entità definite, oppure introdurre un diverso design pattern.
Churn inferisce i cambiamenti attraverso l’analisi dei log dei Subversion.

Il risultato di tale metrica di analisi si può consultare nel file \textit{churn.html} allegato al presente documento.

Anche in questo caso il teacher\_controller risulta essere il file modificato più volte: il numero delle modifiche resta comunque accettabile, ed è giustificato dalla notevole complessità delle operazioni che tale controller è tenuto a gestire.
Tutti gli altri file sono stati limitatamente modificati, indice di una corretta progettazione.
\modifiche
\end{document}
