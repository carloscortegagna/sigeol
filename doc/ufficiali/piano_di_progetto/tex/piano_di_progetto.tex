\documentclass[11pt,a4paper]{article}
\usepackage{amsmath}
\usepackage{amsfonts}
\usepackage{amssymb}
\usepackage{fancyhdr}
\usepackage{lastpage}
\usepackage{graphicx}
\usepackage{ucs}
\usepackage[utf8x]{inputenc}
\usepackage[italian]{babel}

\renewcommand{\headrulewidth}{0.6pt}
\renewcommand{\footrulewidth}{0.6pt}
% impostazione dello stile per le pagine interne del documento
\lhead{\leftmark}
\chead{}
\rhead{\includegraphics[scale=0.15]{logo.png} }
\lfoot{Piano di Progetto v0.1.0}
\cfoot{}
\rfoot{\thepage \ di \pageref{LastPage}}
% ridefinizione dello stile plain per il frontespizio
\fancypagestyle{plain}{
\fancyhf
}
% impostazione dello stile per l'indice
\fancypagestyle{indice}{
\lhead{\leftmark}
\chead{}
\rhead{\includegraphics[scale=0.15]{logo.png}}
\lfoot{Piano di Progetto v0.1.0}
\cfoot{}
\rfoot{}
}
\headheight = 46pt
%definizione del comando "\modfiche" per la creazione del diario delle modifiche
\newcommand{\modifiche} 
{
\newpage
\begin{center}
\textbf{Diario delle modifiche} \\
\bigskip
\begin{tabular}{|c|c|p{0.69\textwidth}|}
\hline
\textsc{Data} & \textsc{Versione} & \textsc{Modifica} \\
\hline
\hline
\textit{29-11-2008} & 0.1.0 & Prima bozza del documento \\
\hline
\end{tabular}
\end{center}
}
%definizione del comando "\info" per la creazione delle informazioni del documento
\newcommand{\info} {
\bigskip
\begin{tabbing}
	\hspace*{0.3\textwidth} \= \hspace*{0.5\textwidth} \kill
	\parbox{0.3\textwidth}{\textbf{Verifica: }} \> \parbox{0.5\textwidth}{Freo Matteo} \\
	\parbox{0.3\textwidth}{\textbf{Approvazione: }} \> \parbox{0.5\textwidth}{Grosselle Alessandro} \\
	\parbox{0.3\textwidth}{\textbf{Stato: }} \> \parbox{0.5\textwidth}{Preliminare} \\
	\parbox{0.3\textwidth}{\textbf{Uso: }} \> \parbox{0.5\textwidth}{Esterno} \\
	\parbox{0.3\textwidth}{\textbf{Distribuzione: }} \> \parbox{0.5\textwidth}{QuiXoft} \\
	\> \parbox{0.5\textwidth}{Rossi Francesca} \\
	\> \parbox{0.5\textwidth}{Vardanega Tullio} \\
\end{tabbing}
}
%definizione del comando "\frontespizio" per la creazione del frontespizio
\newcommand{\frontespizio} {
\thispagestyle{plain}
\title{\begin{Huge}\textsc{Progetto SIGEOL}\end{Huge} \\ \textit{Piano di Progetto \\ v0.1.0}}
\author{Redazione: Grosselle Alessandro}
\maketitle
\medskip
\begin{center}
\includegraphics[scale=0.5]{logo.png} \\
\textit{quixoft.sol@gmail.com}
\end{center}
\medskip
\info
\newpage
}
%definizione del comando "\indice" per la creazione dell'indice
\newcommand{\indice} {
\thispagestyle{indice}
\tableofcontents
\newpage
}
\pagestyle{fancy}
\begin{document}
\frontespizio
\indice
\setcounter{page}{1}
\section{Introduzione}
\subsection{Scopo del prodotto}
Lo scopo del prodotto ''SIGEOL'' è di fornire un sistema software per la gestione dell’orario di lezione a uso del CCS di Informatica.
\subsection{Scopo del documento}
Questo documento espone la pianificazione del progetto ''SIGEOL''.
Al suo interno è possibile trovare una stima delle tempistiche e dei costi del progetto; il tutto è illustrato attraverso l’utilizzo di tabelle e del diagramma di Gantt.
La ripartizione dei ruoli tiene conto del fatto che tutti devono avere circa lo stesso carico di lavoro.
\subsection{Glossario}
Il glossario è unico per tutti i documenti relativi al progetto di sviluppo del prodotto e viene fornito come allegato nel file Glossario.pdf.
\section{Organigramma}
\subsection{Accettazione delle componenti}

TABELLA

\subsection{Descrizione delle componenti}

TABELLA

\subsection{Criteri di assegnazione}
In ogni fase, tutti i membri assumeranno due ruoli.
Il cambio del ruolo avverrà a metà di ogni fase e in quel momento ogni componente dovrà aver speso tutte le ore assegnateli per svolgere il ruolo dato.
Al fine di evitare che un verificatore si trovi nella condizione di dover controllare il proprio operato, in ogni fase del progetto saranno presenti almeno due verificatori.
Si ipotizzano le seguenti date:\\

TABELLA\\

La tabella di seguito riportata mostra come è stata pianificata la rotazione interna dei ruoli.
I numeri tra parentesi vicini al ruolo sono le ore che la risorsa deve fare in quella fase.
Nell’ultima fase non tutti i componenti del team dovranno effettuare il cambio di ruolo.\\

TABELLA

\section{Pianificazione}
\subsection{Diagramma di Gantt}

DIAGRAMMA

\subsection{Ore stimate nel progetto per ruolo}
La tabella sottostante mostra la suddivisione delle ore di ogni ruolo per ogni fase.\\

TABELLA

\subsection{Impegno individuale per ogni ruolo}
La tabella sottostante rappresenta le ore di ogni componente del team in relazione al ruolo ricoperto.\\

TABELLA

\section{Conto economico preventivo}
\subsection{Costi stimati nel progetto per ruolo}
Dalla precedente pianificazione si può redigere la seguente tabella:\\

TABELLA

\subsection{Costi stimati nel progetto per risorsa}
La tabella sottostante rappresenta i costi di ogni componente del team in relazione al ruolo ricoperto:\\

TABELLA

\modifiche
\end{document}
