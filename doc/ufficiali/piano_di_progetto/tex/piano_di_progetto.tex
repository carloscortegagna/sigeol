\documentclass[11pt,a4paper]{article}
\usepackage{amsmath}
\usepackage{amsfonts}
\usepackage{amssymb}
\usepackage{fancyhdr}
\usepackage{lastpage}
\usepackage{graphicx}
\usepackage{ucs}
\usepackage[utf8x]{inputenc}
\usepackage[italian]{babel}
\usepackage{eurosym}

% \renewcommand{\headrulewidth}{0.6pt}
\renewcommand{\footrulewidth}{0.6pt}
% impostazione dello stile per le pagine interne del documento
\lhead{\leftmark}
\chead{}
\rhead{\includegraphics[scale=0.15]{logo.png} }
\lfoot{Piano di Progetto v0.6.0}
\cfoot{}
\rfoot{\thepage \ di \pageref{LastPage}}
% ridefinizione dello stile plain per il frontespizio
\fancypagestyle{plain}{
\fancyhf
}
% impostazione dello stile per l'indice
\fancypagestyle{indice}{
\lhead{\leftmark}
\chead{}
\rhead{\includegraphics[scale=0.15]{logo.png}}
\lfoot{Piano di Progetto v0.6.0}
\cfoot{}
\rfoot{}
}
\headheight = 46pt
%definizione del comando "\modfiche" per la creazione del diario delle modifiche
\newcommand{\modifiche} 
{
\newpage
\begin{center}
\textbf{Diario delle modifiche} \\
\bigskip
\begin{tabular}{|c|c|p{0.62\textwidth}|}
\hline
\textsc{Data} & \textsc{Versione} & \textsc{Modifica} \\
\hline
\hline
\textit{08-12-2008} & 0.6.0 & Aggiunti i riassunti delle ore \\
\hline
\textit{07-12-2008} & 0.5.0 & Aggiunti i riferimenti e le fasi \\
\hline
\textit{07-12-2008} & 0.4.1 & Correzioni varie \\
\hline
\textit{06-12-2008} & 0.4.0 & Inserimento delle revisioni \\
\hline
\textit{05-12-2008} & 0.3.0 & Inserimento del modello di ciclo di vita e dei ruoli \\
\hline
\textit{02-12-2008} & 0.2.0 & Inserimento tabelle e Gantt \\
\hline
\textit{29-11-2008} & 0.1.0 & Prima bozza del documento \\
\hline
\end{tabular}
\end{center}
}
%definizione del comando "\info" per la creazione delle informazioni del documento
\newcommand{\info} {
\bigskip
\begin{tabbing}
	\hspace*{0.3\textwidth} \= \hspace*{0.5\textwidth} \kill
	\parbox{0.3\textwidth}{\textbf{Verifica: }} \> \parbox{0.5\textwidth}{Freo Matteo} \\
	\parbox{0.3\textwidth}{\textbf{Approvazione: }} \> \parbox{0.5\textwidth}{Grosselle Alessandro} \\
	\parbox{0.3\textwidth}{\textbf{Stato: }} \> \parbox{0.5\textwidth}{Formale} \\
	\parbox{0.3\textwidth}{\textbf{Uso: }} \> \parbox{0.5\textwidth}{Esterno} \\
	\parbox{0.3\textwidth}{\textbf{Distribuzione: }} \> \parbox{0.5\textwidth}{QuiXoft} \\
	\> \parbox{0.5\textwidth}{Rossi Francesca} \\
	\> \parbox{0.5\textwidth}{Vardanega Tullio} \\
	\> \parbox{0.5\textwidth}{Conte Renato} \\
\end{tabbing}
}
%definizione del comando "\frontespizio" per la creazione del frontespizio
\newcommand{\frontespizio} {
\thispagestyle{plain}
\title{\begin{Huge}\textsc{Progetto SIGEOL}\end{Huge} \\ \textit{Piano di Progetto \\ v0.6.0}}
\author{Redazione: Scortegagna Carlo}
\maketitle
\medskip
\begin{center}
\includegraphics[scale=0.5]{logo.png} \\
\textit{quixoft.sol@gmail.com}
\end{center}
\medskip
\info
\begin{center}
\textbf{Sommario} \\
Documento contenente il piano di progetto \textit{SIGEOL}, commissionato dalla prof. Rossi Francesca.
\end{center}
\newpage
}
%definizione del comando "\indice" per la creazione dell'indice
\newcommand{\indice} {
\thispagestyle{indice}
\tableofcontents
\newpage
}
\pagestyle{fancy}
\begin{document}
\frontespizio
\indice
\setcounter{page}{1}
\section{Introduzione}
\subsection{Scopo del documento}
Questo documento espone la pianificazione dello sviluppo del progetto denominato ''SIGEOL''.
Al suo interno è possibile trovare una stima delle tempistiche e dei costi del progetto, il tutto illustrato attraverso l'utilizzo di tabelle e del diagramma di Gantt.

Si mostrerà anche come è stata pianificata la rotazione dei ruoli per ogni fase del progetto e la ripartizione del carico di lavoro individuale.
Il piano si evolve di pari passo col progetto e inizialmente il documento conterrà solamente previsioni e stime.

Data la scarsa esperienza del team QuiXoft non si esclude la possibilità che il piano di progetto potrà essere rivisto e rimodellato in corso d'opera.
\subsection{Scopo del prodotto}
Il prodotto ''SIGEOL'' si prefigge di automatizzare la generazione, la gestione, l'ottimizzazione e la consultazione degli orari di lezione. 

Per ulteriori informazioni riguardanti scopi e funzioni del prodotto si prega di fare riferimento al documento \textsc{Analisi dei Requisiti}.
\subsection{Riferimenti}
\begin{itemize}
 	\item Il ciclo di vita del \underline{software} \\ \textit{http://www.math.unipd.it/~tullio/IS-1/2008/Dispense/P04.pdf}
 	\item Gantt \\ \textit{http://gates.comm.virginia.edu/rrn2n/teaching/gantt.htm}
	\item GNU General Public License 2.0 \\ \textit{http://www.gnu.org/licenses/gpl-2.0.html}
	\item \textsc{Norme di Progetto}
\end{itemize}
\subsection{Glossario} 
Le definizioni dei termini specialistici usati nella stesura di questo e di tutti gli altri documenti possono essere trovate nel documento \textsc{Glossario} al fine di eliminare ogni ambiguità e di facilitare la comprensione dei temi trattati. Ogni termine la cui definizione è disponibile all'interno del glossario verrà marcato con una \underline{sottolineatura}.
\section{Panoramica di progetto}
\subsection{Vincoli}
Il progetto, per essere considerato accettabile dal committente, dovrà tassativamente rispettare i seguenti vincoli.
\subsubsection{Equa ripartizione del carico di lavoro}
Il piano di progetto prevede un'equa ripartizione del carico di lavoro tra tutti i componenti del team.
In linea di massima, le ore fatte in corso d'attività dovranno essere paragonabili a quelle attribuite a ciascun componente del gruppo nel piano.

Il piano di progetto di fase in fase verrà aggiornato inserendo le ore consuntive di ogni ruolo.
E' necessario che ogni componente abbia un carico di lavoro inferiore alle 105 ore. Per maggiori informazioni si prega di consultare la sezione 3.4.
\subsubsection{Budget Economico}
Il progetto dovrà affrontare una spesa pari ad almeno 13000\euro.

Non saranno tollerati preventivi inferiori a questa cifra, e nello stesso tempo una spesa preventivata che superi in modo cospiquo tale valore sarà fonte di penalizzazione in fase di revisione dei requisiti.
\subsubsection{Licenza}
Passata la revisione di accettazione il \underline{software} prodotto verrà rilasciato con licenza libera GNU GPL versione 2.0.

Come ogni licenza di \underline{software} libero, essa concede agli utenti il permes\-so di modificare il programma, di copiarlo e di ridistribuirlo con o senza modifiche, gratuitamente o a pagamento.
\subsection{Componenti}
Il team QuiXoft è formato dai seguenti componenti:
\begin{itemize}
\item Barbiero Mattia
\item Beggiato Andrea
\item Freo Matteo
\item Grosselle Alessandro
\item Scarpa davide
\item Scortegagna Carlo 
\end{itemize}
Ogni componente soddisfa i vincoli di propedeuticità ed è quindi libero di partecipare alla revisione dei requisiti.
\subsection{Ruoli}
Il progetto prevede i seguenti ruoli:
\begin{itemize}
\item \textbf{Responsabile:} è colui che si fa carico, per conto del suo team, dei risultati del progetto. Elabora ed emana piani e scadenze, approva l'emissione di documenti, coordina le attività del gruppo e si relaziona con il controllo di qualità interno al progetto.
E' responsabile della pianificazione di progetto, coordinando le attività del gruppo.
\item \textbf{Amministratore:} è responsabile della redazione e attuazione di piani e procedure di gestione per la qualità.
Gestisce la documentazione di progetto e controlla versioni e configurazioni del prodotto.
\item \textbf{Analista:} ha grande impatto sul successo del progetto poichè è responsabile dell'attività di analisi.
\item \textbf{Progettista:} ha forte impatto sugli aspetti tecnici e tecnologici del progetto. E' responsabile delle attività di progettazione.
\item \textbf{Programmatore:} è responsabile delle attività di codifica miranti alla realizzazione del prodotto e delle componenti di ausilio necessarie per l'esecuzione delle prove di verifica e validazione.
\item \textbf{Verificatore:}  partecipa all'intero ciclo di vita ed è responsabile alla verifica dei processi e della validazione del prodotto.
\end{itemize}
\subsection{Costo dei ruoli}
Ogni ruolo all'interno del team QuiXoft ha un determinato costo orario. Nella tabella sottostante ne vengono riportati i dettagli:
\medskip
\begin{center}
\begin{tabular}{|c|c|}
\hline
\textsc{Ruolo} & \textsc{\euro \ / ora}\\ \hline \hline
Responsabile & 30 \\ \hline
Amministratore & 20 \\ \hline
Analista & 25 \\ \hline
Progettista & 22 \\ \hline
Programmatore & 16 \\ \hline
Verificatore & 16 \\ \hline
\end{tabular}
\end{center}
\section{Pianificazione}
Verranno qui descritti gli aspetti riguardanti la pianificazione del progetto, il ciclo di vita prescelto e le stime sulla durata delle fasi.

Vista la natura del team di sviluppo, in cui tutti i componenti occuperanno a rotazione tutti i ruoli necessari, particolare attenzione sarà posta nel fissare date precise in cui cambiare ruoli e nel garantire l'assenza di conflitti di interesse.

Per necessità organizzative, è possibile che qualche componente possa esercitare due ruoli durante una singola rotazione; è comunque garantita la completa assenza di interessi tra i compiti assunti.
\bigskip

Per adattare alcune tabelle al documento, i nominativi dei ruoli verranno così abbreviati:
\begin{itemize}
\item Responsabile: Res
\item Amministratore: Amm
\item Analista: An
\item Progettista: Prog
\item Programmatore: Progr
\item Verificatore: Ver
\end{itemize}
\bigskip
Per lo stesso motivo anche i nominativi dei componenti verranno così abbrevviati:
\begin{itemize}
\item Beggiato Andrea: BA
\item Barbiero Mattia: BM
\item Freo Matteo: FM
\item Grosselle Alessandro: GA
\item Scarpa Davide: SD
\item Scortegagna Carlo: SC
\end{itemize}
\subsection{Modello di ciclo di vita}
Il modello scelto dal team QuiXoft è il modello a cascata.
Questo è una progressione sequenziale (in cascata) di fasi, senza ricicli, al fine di meglio controllare tempi e costi.
Ciascuna di queste fasi produce un ben preciso output che viene utilizzato come input per la fase successiva.

Con questo modello rigidamente sequienziale la pianificazione è molto precisa ma presenta uno spiacevole incoveniente: assume che i requisiti possano essere congelati alla fine della fase di analisi.

Data la poca esperienza del team QuiXoft è molto difficile che tutti i requisiti siano chiari alla fine dell'analisi.
Per questo motivo si è scelto di utilizzare una variante del modello a cascata prevedendo eventuali prototipazioni allo scopo di capire meglio i requisiti.

Oltretutto, il committente non esclude possibili variazioni in corso d'opera alle richieste, e quindi, a maggior ragione, il modello a cascata con protipazione risulta particolarmente adatto.
\subsection{Fasi}
Il modello a cascata prevede le seguenti fasi:
\begin{itemize}
\item Analisi
\item Progettazione
\item Realizzazione
\item Manutenzione
\end{itemize}
\medskip
La fase di \underline{manutenzione} non verrà tenuta dal team QuiXoft in quanto il progetto terminerà alla fine della fase di realizzazione, con l'accettazione del prodotto da parte del committente alla revisione di accettazione.

La \underline{manutenzione} verrà quindi fatta dalla comunità di utenti che utilizzeranno il prodotto, come previsto dalla licenza libera con cui verrà rilasciato.
\subsubsection{Analisi}
La fase di analisi ha lo scopo di determinare cosa farà il sistema. Essa è preceduta da una analisi di fattibilità, in cui si stabilisce se vale la pena (da un punto di vista tecnico ed economico) realizzare il sistema del quale si vanno definendo i requisiti.

In seguito avviene la definizione dei vincoli, delle funzioni, dei requisiti e di qualsiasi altra caratteristica che il sistema dovrà soddisfare.
L'individuazione dei requisiti si basa sul capitolato d'appalto e su eventuali incontri con il commitente.

La data d'inizio della fase di analisi è il 17/11/2008, data del primo incontro del team QuiXoft. Si prevede di terminare l'analisi il 09/01/09.

All'incirca a metà della fase di analisi il team dovrà affrontare la revisione dei requisiti, condotta dal committente.
Per questa revisione è prevista la discussione dell'\textsc{Analisi dei Requisiti}, del \textsc{Piano di Qualifica} e del presente documento, il \textsc{Piano di Progetto}.

Si riporta ora lo schema di attribuzione e di rotazione dei ruoli durante la fase di analisi.
La rotazione dei ruoli è tassativamente stabilita per il giorno 9/12/2008, ed è rappresentata nella seguente tabella dal passaggio dalla prima fase di analisi (Analisi [1]) alla seconda (Analisi [2]).
A fianco di ogni ruolo sono riportate, tra parentesi, le ore stimate da esercitare.
\\
\begin{center}
\begin{tabular}{|c||c|c|}
\hline
\textsc{Componenti} & \textsc{Analisi [1]} & \textsc{Analisi [2]} \\ \hline \hline
Barberio Mattia & An(20) & Res(11) \\ \hline
Beggiato Andrea & An(20) & Ver(10) \\ \hline
Freo Matteo & Ver(10) & An(20) \\ \hline
Grosselle Alessandro & Amm(13,5) & An(20) \\ \hline
Scarpa Davide & An(20) & Amm(13,5) \\ \hline
Scortegagna Carlo & Res(10) & An(20) \\ \hline
\end{tabular}
\end{center}
\bigskip
\subsubsection{Progettazione}
Sulla base della specifica dei requisiti prodotta dall'analisi, la fase di progettazione definisce come tali requisiti saranno soddisfatti, entrando nel merito della struttura che dovrà essere data al sistema \underline{software} che deve essere realizzato.

La fase di progettazione prevede inizialmente la definizione dell'architettura di sistema: si tratta di una progettazione ad altissimo livello, in cui si definisce solo la struttura complessiva del sistema in termini dei principali moduli di cui esso è composto e delle relazioni macroscopiche fra di essi.

In seguito alla progettazione architetturale vi sarà la definizione delle strutture interne di ciascun componente: questa fase, nota come progettazione di dettaglio, rappresenta una descrizione del sistema molto vicina alla codifica, ma sempre indipendente da decisioni realizzative quali, ad esempio, la scelta del \underline{linguaggio di programmazione}.

La data di inizio della fase di progettazione è fissata per il 12/01/2009 e la data di fine al 13/02/2009.

Verso la fine della fase il team affronterà la prima revisione interna, la Revisione del Progetto Preliminare (RPP), che metterà in luce una visione ad alto livello del sistema.
In occasione di tale revisione il team dovrà redarre la \textsc{Specifica Tecnica} del progetto e aggiornare il \textsc{Piano di Qualifica}.

Se la revisione andrà a buon fine, si attiverà la fase realizzativa del prodotto.



Si riporta ora lo schema di attribuzione e di rotazione dei ruoli durante la fase di progettazione.
La rotazione dei ruoli è fissata per il giorno 27/01/2009, ed è rappresentata nella seguente tabella dal passaggio dalla prima fase di progettazione (Progettazione [1]) alla seconda (Progettazione [2]).
A fianco di ogni ruolo sono riportate, tra parentesi, le ore stimate da esercitare.
\\
\begin{center}
\begin{tabular}{|c||c|c|}
\hline
\textsc{Componenti} & \textsc{Progettazione [1]} & \textsc{Progettazione [2]} \\ \hline \hline
Barberio Mattia & Prog(21,5) & Ver(20) \\ \hline
Beggiato Andrea & Amm(13,5)/Ver(10) & Prog(22) \\ \hline
Freo Matteo & Prog(21) & Res(10,5) \\ \hline
Grosselle Alessandro & Ver(20) & Prog(22) \\ \hline
Scarpa Davide & Res(10) & Prog(21,5) \\ \hline
Scortegagna Carlo & Prog(22) & Amm(13,5) \\ \hline
\end{tabular}
\end{center}
\bigskip
\subsubsection{Realizzazione}
La fase di realizzazione ha lo scopo di implementare i vari moduli definiti nella progettazione tramite un determinato \underline{linguaggio di programmazione}.

Alla codifica seguiranno i periodi di prova, integrazione e \underline{collaudo} del sistema.
Tali fasi di test e verifica sono necessarie per garantire la correttezza dell'implementazione dei singoli moduli e la correttezza del funzionamento complessivo del sistema.

Ogni volta che il modulo di un componente vieni istanziato, questo viene collaudato: se l'esito è positivo viene integrato al componente, altrimenti verrà segnalato al programmatore il malfunzionamento.

Per ulteriori informazioni sulla gestione dei malfunzionamenti si prega di consultare il documento \textsc{Piano di Qualifica}.

Non appena tutte le componenti saranno pronte, verranno a loro volta integrate tra loro e si avvierà la fase di \underline{collaudo} dell'intero sistema.

La data di inizio della fase di realizzazione è fissata per il 16/02/2009, la data di fine per il 24/03/2009.

Orientativamente a tre quarti della fase di realizzazione è prevista la Revisione di Qualifica (RQ), che ha lo scopo di approvare l'esito finale della verifica. In questa fase si dovrà portare la versione definitiva del \textsc{Piano di Qualifica} e la versione iniziale del \textsc{Manuale Utente}.

A fine fase ci sarà la Revisione di Accettazione (RA), in cui vi sarà il \underline{collaudo} del sistema da parte del committente e l'accertamento di soddisfacimento di tutti i requisiti previsti. In quest'occasione dovrà essere consegnata la versione definitiva del \textsc{Manuale Utente} e la versione aggiornata del \textsc{Piano di Qualifica}.

Se la revisione andrà a buon fine il prodotto verrà accettato e il progetto avrà termine.

Si riporta ora lo schema di attribuzione e di rotazione dei ruoli durante la fase di realizzazione.
La rotazione dei ruoli è tassativamente stabilita per il giorno 02/03/2009, ed è rappresentata nella seguente tabella dal passaggio dalla prima fase di realizzazione (Realizzazione [1]) alla seconda (Realizzazione [2]).
A fianco di ogni ruolo sono riportate, tra parentesi, le ore stimate da esercitare.
\\
\begin{center}
\begin{tabular}{|c||c|c|}
\hline
\textsc{Componenti} & \textsc{Realizzazione [1]} & \textsc{Realizzazione [2]} \\ \hline \hline
Barberio Mattia & Progr(18) & Amm(14) \\ \hline
Beggiato Andrea & Progr(18) & Res(11) \\ \hline
Freo Matteo & Amm(14,5) & Progr(18,5)/Ver(10) \\ \hline
Grosselle Alessandro & Res(10) & Progr(18) \\ \hline
Scarpa Davide & Progr(19,5) & Ver(20) \\ \hline
Scortegagna Carlo & Ver(20) & Progr(18) \\ \hline
\end{tabular}
\end{center}
\bigskip
\subsection{Riassunto delle ore}
Dai dati riguardanti le fasi appena elencate è possibile stilare la seguente tabella riassuntiva, che mostra la suddivisione del carico di lavoro (espresso in ore) per ogni componente del team QuiXoft, diviso per ruolo.

Si può altresi notare, osservando la colonna ''Totale'', che il carico di lavoro personale che ogni componente del team dovrà sopportare durante lo sviluppo del progetto non supera le 105 ore, in pieno rispetto del vincolo consultabile al punto 2.2.1.
\\ \bigskip
\begin{center}
\begin{tabular}{|c||c|c|c|c|c|c||c|}
\hline
\textsc{Componenti} & \textsc{Res} & \textsc{Amm} & \textsc{An} & \textsc{Prog} & \textsc{Ver} & \textsc{Progr} & \textsc{Totale}\\
\hline \hline
BM & 11 & 14 & 20 & 21.5 & 20 & 18 & 104.5 \\ \hline
BA & 11 & 13,5 & 20 & 22 & 20 & 18 & 104.5 \\ \hline
FM & 10.5 & 14.5 & 20 & 21 & 20 & 18.5 & 104.5 \\ \hline
GA & 10 & 13.5 & 20 & 22 & 20 & 18 & 103.5 \\ \hline
SD & 10 & 13.5 & 20 & 21.5 & 20 & 19.5 & 104.5\\ \hline
SC & 10 & 13.5 & 20 & 22 & 20 & 18 & 103.5\\ \hline
\end{tabular}
\end{center}
\bigskip \bigskip \bigskip

Risulta utile anche valutare le ore che ogni ruolo avrà assegnate durante le varie fasi del progetto: ciò può portare ad una stima del peso e dell'importanza dei vari ruoli nello sviluppo del progetto \underline{software} ''SIGEOL''.

La seguente tabella rappresenta appunto le ore che ogni ruolo svolgerà per periodo, il totale di ore per fase e il numero totale di ore che il team QuiXoft preventiva di impiegare per sviluppare completamente il progetto:
\begin{center}
\begin{tabular}{|c||c|c|c||c|}
\hline
\textsc{Componenti}& \textsc{Analisi} & \textsc{Progettazione} & \textsc{Realizzazione} & \textsc{Tot.} \\ \hline \hline
Responsabile & 21 & 20.5 & 21 & 62.5 \\ \hline
Amministratore & 27 & 27 & 28.5 & 82.5 \\ \hline
Analista & 120 & 0 & 0 & 120 \\ \hline
Progettista & 0 & 130 & 0 & 130 \\ \hline
Programmatore & 0 & 0 & 110 & 110 \\ \hline
Verificatore & 20 & 50 & 50 & 120 \\ \hline \hline
\textsc{Totale} & 188 & 227.5 & 209.5 & 625 \\ \hline
\end{tabular}
\end{center}
\bigskip
\subsection{Panoramica delle revisioni}
Durante lo sviluppo del progetto il team QuiXoft dovrà sostenere quattro revisioni: la prima e l'ultima sono esterne e saranno condotte dal committente, le rimanenti due sono interne al team, con il coinvolgimento del professore designato.

Lo scopo di una revisione interna è quello di valutare il progresso del progetto, determinando se esistono problematiche durante le fasi di sviluppo.
Questo tipo di revisione non ha effetto sanzionatorio.

In una revisione esterna sarà il committente ad esaminare e valutare tutte le attività svolte dal team in quella determinata fase.
Questo tipo di revisione ha invece effetto sanzionatorio.
\medskip

Le revisioni sono le seguenti:
\begin{itemize}
\item Revisione dei requisiti (RR)
\item Revisione del progetto preliminare (RPP)
\item Revisione di qualifica (RQ)
\item Revisione di accettazione (RA)
\end{itemize}
\subsubsection{Revisione dei requisiti (RR)}
In questa revisione vi sarà una discussione tra committente e fornitore riguardo i requisiti del sistema, allo scopo di verificare se entrambe le parti condividono una stessa visione generale del prodotto.
La fine della fase di analisi non coincide con la revisione dei requisiti; per questo motivo tutti i prodotti portati a questa revisione sono da considerarsi parziali.

Per questa fase il team dovrà produrre tre documenti:
\begin{itemize}
\item \textsc{Analisi dei Requisiti}: mostra la definizione e classificazione dei requisiti;
\item \textsc{Piano di Qualifica}: delinea la strategia generale di verifica e validazione;
\item \textsc{Piano di progetto}: mostra la pianificazione del progetto.
\end{itemize}
\subsubsection{Revisione del progetto preliminare(RPP)}
Il fornitore in questa fase avrà prodotto una visione ad alto livello del sistema attraverso l'uso di diagrammi delle classi e altre rappresentazioni architetturali.
Se la revisione andrà a buon fine si attiverà la fase realizzativa del prodotto.
Oltre a questo, il fornitore dovrà mostrare l'esistenza di strategie e tecnologie adeguate per l'implementazione del progetto.

Per questa fase si dovrà aggiornare il \textsc{Piano di Qualifica}, il \textsc{Piano di Progetto} e si dovrà produrre un nuovo documento:
\begin{itemize}
\item \textsc{Specifica Tecnica}: presenta l'archittetura generale del sistema identificando e descrivendo le sue componenti di alto livello.
\end{itemize}
\subsubsection{Revisione di qualifica(RQ)}
La funzione di questa revisione è approvare l'esito finale della verifica.

Per questa fase si dovrà aggiornare il \textsc{Piano di Qualifica}, inserendo il resoconto definitivo della campagna di verifica e la descrizione delle prove proposte per il \underline{collaudo}. Si dovrà poi produrre in modo parziale un nuovo documento:
\begin{itemize}
\item \textsc{Manuale Utente}: contiene le istruzioni per l'uso del sistema.
\end{itemize}
\subsubsection{Revisione di accettazione(RA)}
In questa revisione il sistema verrà collaudato dal committente e di conseguenza accettato o meno.
Vi sarà inoltre l'accertamento del soddisfacimento di tutti i requisiti pattuiti.

Per questa fase si dovranno aggiornare in modo definitivo il \textsc{Piano di Qualifica} e il \textsc{Manuale Utente}.
Se la revisione andrà a buon fine il progetto verrà accettato e il progetto potrà considerarsi terminato.
\subsection{Diagramma di Gantt}
Al presente documento viene allegato il file \textsc{gantt.pdf}, contenente la pianificazione delle attività per ogni fase.

Il diagramma in allegato è caratterizzato dalla rappresentazione dell'arco temporale totale del progetto sull'asse orrizzontale (in settimane) e dalla rappresentazione delle mansioni o attività che costituiscono il progetto sull'asse verticale.
\section{Analisi dei rischi}
\subsection{Assenza componente per medio/lungo periodo}
Ogni componente ha sempre un determinato ruolo in tutte le fasi del progetto.
Se un componente risultasse indisponibile per un lasso di tempo più o meno lungo causerebbe un rallentamento del progetto, se non addirittura uno stallo.

E' compito dell'\underline{Amministratore} riassegnare il determinato ruolo ad uno o più componenti del team e modificare il piano di progetto cercando di mantenere inalterati i costi e la data di fine progetto.
\subsection{Mancanza di conoscenze tecniche}
Si utilizzeranno strumenti e tecnologie che per alcuni componenti risulteranno nuove.
L'\underline{Amministratore} metterà a disposizione guide e manuali per poter formare il team.

Lo studio è personale e non è previsto nel piano di progetto.
Tuttavia, se dovesse essere necessario, verrà fatta una riunione che delinearà i concetti di massima dell'eventuale strumento o tecnologia.
\subsection{Calendario delle attività inneficiente}
Le attività, vista la scarsa esperienza, sono state pianificate basandosi su precedenti calendari di altri gruppi.

E' compito dell'\underline{Amministratore} correggere eventuali errori di pianificazione, cercando di mantenere inalterati i costi e la data di fine. progetto.
\subsection{Analisi dei requisiti inefficiente}
Nella prima fase avviene lo studio approfondito dei requisiti.
Finita questa fase, inizia quella di progettazione.

Data la scarsa esperienza del team QuiXoft il rischio di variazione dei requisiti dopo il loro studio esiste. Tuttavia si cerca di renderlo il più basso possibile ruotando il ruolo di \underline{Analista} e rendendo la ricerca dei requisiti il più completa ed efficiente possibile.
\subsection{Gestione della qualità inadeguata}
L'accertamento della qualità del processo e del prodotto è garantita dalla presenza del ruolo del \underline{Verificatore} per tutto il periodo di concezione e sviluppo del progetto.

Se le ore necessarie al \underline{Verificatore} in una determinata fase dovessero essere troppo poche, ne verrà immediatamente avvisato l'\underline{Amministratore}, il quale modificherà di conseguenza il \textsc{Piano di Progetto}, dedicando più ore alle fasi di verifica.
\section{Conto economico preventivo}
\subsection{Costi stimati nel progetto per ruolo}
Dalla precedente pianificazione si può redigere la seguente tabella, che mostra i costi stimati (in \euro) per ruolo in relazione alle varie fasi del progetto.

In pieno rispetto al vincolo ''Budget economico'' della sezione 2.1.2, il costo totale del progetto è stimato in 13065\euro.
\begin{center}
\begin{tabular}{|c||c|c|c||c|}
\hline
\textsc{Componenti}& \textsc{Analisi} & \textsc{Progettazione} & \textsc{Realizzazione} & \textsc{Tot.} \\ \hline \hline
Responsabile & 630 & 615 & 630 & 1875 \\ \hline
Amministratore & 540 & 540 & 570 & 1650 \\ \hline
Analista & 3000 & 0 & 0 & 3000 \\ \hline
Progettista & 0 & 2860 & 0 & 2860 \\ \hline
Programmatore & 0 & 0 & 1760 & 1760 \\ \hline
Verificatore & 320 & 800 & 800 & 1920 \\ \hline \hline
\textsc{Totale} & 4490 & 4815 & 3760 & 13065 \\ \hline
\end{tabular}
\end{center}
\subsection{Costi stimati nel progetto per componente}
La tabella sottostante rappresenta i costi (in \euro) di ogni componente del team in relazione al ruolo ricoperto.

Emerge un peso economico pressochè identico tra i vari membri del team QuiXoft al termine di tutte le fasi di sviluppo.
\begin{center}
\begin{tabular}{|c||c|c|c|c|c|c||c|}
\hline
\textsc{Componenti} & \textsc{Res} & \textsc{Amm} & \textsc{An} & \textsc{Prog} & \textsc{Ver} & \textsc{Progr} & \textsc{Totale}\\
\hline \hline
BM & 330 & 280 & 500 & 473 & 320 & 288 & 2191\\ \hline
BA & 330 & 270 & 500 & 484 & 320 & 288 & 2192\\ \hline
FM & 315 & 290 & 500 & 462 & 320 & 296 & 2183\\ \hline
GA & 300 & 270 & 500 & 484 & 320 & 288 & 2162\\ \hline
SD & 300 & 270 & 500 & 473 & 320 & 312 & 2175\\ \hline
SC & 300 & 270 & 500 & 484 & 320 & 288 & 2162\\ \hline
\end{tabular}
\end{center}
\modifiche
\end{document}