\documentclass[11pt,a4paper]{article}
\usepackage{amsmath}
\usepackage{amsfonts}
\usepackage{amssymb}
\usepackage{fancyhdr}
\usepackage{lastpage}
\usepackage{graphicx}
\usepackage{ucs}
\usepackage[utf8x]{inputenc}
\usepackage[italian]{babel}

\renewcommand{\headrulewidth}{0.6pt}
\renewcommand{\footrulewidth}{0.6pt}
% impostazione dello stile per le pagine interne del documento
\lhead{\leftmark}
\chead{}
\rhead{\includegraphics[scale=0.15]{logo.png} }
\lfoot{Piano di Progetto v0.2.0}
\cfoot{}
\rfoot{\thepage \ di \pageref{LastPage}}
% ridefinizione dello stile plain per il frontespizio
\fancypagestyle{plain}{
\fancyhf
}
% impostazione dello stile per l'indice
\fancypagestyle{indice}{
\lhead{\leftmark}
\chead{}
\rhead{\includegraphics[scale=0.15]{logo.png}}
\lfoot{Piano di Progetto v0.2.0}
\cfoot{}
\rfoot{}
}
\headheight = 46pt
%definizione del comando "\modfiche" per la creazione del diario delle modifiche
\newcommand{\modifiche} 
{
\newpage
\begin{center}
\textbf{Diario delle modifiche} \\
\bigskip
\begin{tabular}{|c|c|p{0.69\textwidth}|}
\hline
\textsc{Data} & \textsc{Versione} & \textsc{Modifica} \\
\hline
\hline
\textit{29-11-2008} & 0.1.0 & Prima bozza del documento \\
\hline
\hline
\textit{02-12-2008} & 0.2.0 & Inserimento tabelle e Gantt \\
\hline
\end{tabular}
\end{center}
}
%definizione del comando "\info" per la creazione delle informazioni del documento
\newcommand{\info} {
\bigskip
\begin{tabbing}
	\hspace*{0.3\textwidth} \= \hspace*{0.5\textwidth} \kill
	\parbox{0.3\textwidth}{\textbf{Verifica: }} \> \parbox{0.5\textwidth}{Freo Matteo} \\
	\parbox{0.3\textwidth}{\textbf{Approvazione: }} \> \parbox{0.5\textwidth}{Grosselle Alessandro} \\
	\parbox{0.3\textwidth}{\textbf{Stato: }} \> \parbox{0.5\textwidth}{Preliminare} \\
	\parbox{0.3\textwidth}{\textbf{Uso: }} \> \parbox{0.5\textwidth}{Esterno} \\
	\parbox{0.3\textwidth}{\textbf{Distribuzione: }} \> \parbox{0.5\textwidth}{QuiXoft} \\
	\> \parbox{0.5\textwidth}{Rossi Francesca} \\
	\> \parbox{0.5\textwidth}{Vardanega Tullio} \\
\end{tabbing}
}
%definizione del comando "\frontespizio" per la creazione del frontespizio
\newcommand{\frontespizio} {
\thispagestyle{plain}
\title{\begin{Huge}\textsc{Progetto SIGEOL}\end{Huge} \\ \textit{Piano di Progetto \\ v0.2.0}}
\author{Redazione: Grosselle Alessandro}
\maketitle
\medskip
\begin{center}
\includegraphics[scale=0.5]{logo.png} \\
\textit{quixoft.sol@gmail.com}
\end{center}
\medskip
\info
\newpage
}
%definizione del comando "\indice" per la creazione dell'indice
\newcommand{\indice} {
\thispagestyle{indice}
\tableofcontents
\newpage
}
\pagestyle{fancy}
\begin{document}
\frontespizio
\indice
\setcounter{page}{1}
\section{Introduzione}
\subsection{Scopo del prodotto}
Lo scopo del prodotto ''SIGEOL'' è di fornire un sistema software per la gestione dell’orario di lezione a uso del CCS di Informatica.
\subsection{Scopo del documento}
Questo documento espone la pianificazione del progetto ''SIGEOL''.
Al suo interno è possibile trovare una stima delle tempistiche e dei costi del progetto; il tutto è illustrato attraverso l’utilizzo di tabelle e del diagramma di Gantt.
La ripartizione dei ruoli tiene conto del fatto che tutti devono avere circa lo stesso carico di lavoro.
\subsection{Glossario}
Il glossario è unico per tutti i documenti relativi al progetto di sviluppo del prodotto e viene fornito come allegato nel file Glossario.pdf.
\section{Organigramma}
\subsection{Accettazione delle componenti}

\begin{tabular}{|c|c|c|}
\hline
Nominativo  & Data & Firma(leggibile) \\ \hline
Barbiero Mattia & 12/11/08 & \multicolumn{1}{l|}{} \\ \hline
Beggiato Andrea & 12/11/08 &  \\ \hline
Freo Matteo & 12/11/08 &  \\ \hline
Grosselle Alessandro & 12/11/08 &  \\ \hline
Scarpa Davide & 12/11/08 &  \\ \hline
Scortecagna Carlo & 12/11/08 &  \\ \hline
\end{tabular}


\subsection{Descrizione delle componenti}

\begin{tabular}{|c|c|c|}
\hline
Nominativo  & Matricola & e-mail \\ \hline
Barbiero Mattia & 540546 & barraemme@gmail.com \\ \hline
Beggiato Andrea & 541738 & saprus@hotmail.it \\ \hline
Freo Matteo & 560838 & tizzy369@hotmail.it \\ \hline
Grosselle Alessandro & 540501 & ale7forever@gmail.com \\ \hline
Scarpa Davide & 544006 & goldmagic@libero.it \\ \hline
Scortecagna Carlo & 545070 & carloscortecagna@gmail.com \\ \hline
\end{tabular}

\subsection{Criteri di assegnazione}
In ogni fase, tutti i membri assumeranno due ruoli.
Il cambio del ruolo avverrà a metà di ogni fase e in quel momento ogni componente dovrà aver speso tutte le ore assegnateli per svolgere il ruolo dato.
Si ipotizzano le seguenti date:\\

\begin{tabular}{|c|c|c|}
\hline
Fase & Inizio  & Termine \\ \hline
Analisi dei requisiti & 17/11/08 & 23/12/08 \\ \hline
Progettazione & 07/01/09 & 14/02/09 \\ \hline
Codifica & 16/02/09 & 11/03/09 \\ \hline
Verifica e validazione & 17/03/09 & 24/03/09 \\ \hline
\end{tabular}
\\

La tabella di seguito mostra come è stata pianificata la rotazione interna dei ruoli.
I numeri tra parentesi vicini al ruolo sono le ore che la risorsa deve spendere in quella fase.

\begin{tabular}{|l|}
\hline
Legenda: \\ \hline
Re=Responsabile \\ \hline
Am=Amministratore \\ \hline
An=Analista \\ \hline
Prog=Progettista \\ \hline
Progr=Programmatore \\ \hline
Ver=Verificatore \\ \hline
\end{tabular}\\\\

\begin{tabular}{|c|c|c|}
\hline
 & Analisi dei requisiti & Progettazione \\ \hline
Barbiero Mattia & An(20)/Ve(5) & Am(13,5)/ Prog(21,5) \\ \hline
Beggiato Andrea & An(20)/Ver(5) & Re(5)/Prog(22) \\ \hline
Freo Matteo & An(20)/Re(11) & Prog(21)/Am(6,5) \\ \hline
Grosselle Alessandro & Am(14)/An(20) & Ver(10)/ Prog(22) \\ \hline
Scarpa Davide & An(20)/Am(5) & Prog(21,5)/Re(9) \\ \hline
Scortecagna Carlo & Re(11)/An(20) & Prog(22)/Ver(10) \\ \hline
\end{tabular}\\\\

\begin{tabular}{|c|c|c|}
\hline
 & Codifica & Ver. \& val. \\ \hline
Barbiero Mattia & Re(5)/Progr(19) & Ver(15)/Re(6) \\ \hline
Beggiato Andrea & Progr(18)/Am(13,5) & Re(6)/Ver(15) \\ \hline
Freo Matteo & Progr(18,5)/Ver(10) & Ver(10)/Am(8) \\ \hline
Grosselle Alessandro & Progr(13)/Re(9,5) & Progr(5)/Ver(10) \\ \hline
Scarpa Davide & Ver(10)/Progr(18,5) & Am(9)/Ver(10) \\ \hline
Scortecagna Carlo & Am(13)/Progr(13) & Ver(10)/Progr(5) \\ \hline
\end{tabular}


\section{Pianificazione}
\subsection{Diagramma di Gantt}

\includegraphics[scale=0.5]{gantt.png} \\

\subsection{Ore stimate nel progetto per ruolo}
La tabella sottostante mostra la suddivisione delle ore di ogni ruolo per ogni fase.\\

\begin{tabular}{|c|c|c|c|c|c|}
\hline
 & Analisi & Progettazione & Codifica & Ver.\&Val. & Totale \\ \hline
Responsabile & 22 & 14 & 14,5 & 12 & 62,5 \\ \hline
Amministratore & 19 & 20 & 26,5 & 17 & 82,5 \\ \hline
Analista & 120 & 0 & 0 & 0 & 120 \\ \hline
Progettista & 0 & 130 & 0 & 0 & 130 \\ \hline
Programmatore & 0 & 0 & 100 & 10 & 110 \\ \hline
Verificatore & 10 & 20 & 20 & 70 & 120 \\ \hline
\end{tabular}
\\

\subsection{Impegno individuale per ogni ruolo}
La tabella sottostante rappresenta le ore di ogni componente del team in relazione al ruolo ricoperto.\\

\begin{tabular}{|c|c|c|c|c|c|c|c|}
\hline
 & Barbiero & Beggiato & Freo & Grosselle & Scarpa & Scortecagna & Totale \\ \hline
Responsabile & 11 & 11 & 11 & 9,5 & 9 & 11 & 62,5 \\ \hline
Amministratore & 13,5 & 13,5 & 14,5 & 14 & 14 & 13 & 82,5 \\ \hline
Analista & 20 & 20 & 20 & 20 & 20 & 20 & 120 \\ \hline
Progettista & 21,5 & 22 & 21 & 22 & 21,5 & 22 & 130 \\ \hline
Programmatore & 19 & 18 & 18,5 & 18 & 18,5 & 18 & 110 \\ \hline
Verificatore & 20 & 20 & 20 & 20 & 20 & 20 & 120 \\ \hline
\end{tabular}

\section{Conto economico preventivo}
\subsection{Costi stimati nel progetto per ruolo}
Dalla precedente pianificazione si può redigere la seguente tabella:\\

\begin{tabular}{|c|c|c|c|c|c|}
\hline
 & Analisi & Progettazione & Codifica & Verifica\&Valid. & Totale \\ \hline
Responsabile & 660 & 420 & 435 & 360 & 1875 \\ \hline
Amministratore & 380 & 400 & 530 & 340 & 1650 \\ \hline
Analista & 3000 & 0 & 0 & 0 & 3000 \\ \hline
Progettista & 0 & 2860 & 0 & 0 & 2860 \\ \hline
Programmatore & 0 & 0 & 1600 & 160 & 1760 \\ \hline
Verificatore & 160 & 320 & 320 & 1120 & 1920 \\ \hline
Totale & 4200 & 4000 & 3045 & 1820 & 13065 \\ \hline
\end{tabular}

\subsection{Costi stimati nel progetto per risorsa}
La tabella sottostante rappresenta i costi di ogni componente del team in relazione al ruolo ricoperto:\\

\begin{tabular}{|c|c|c|c|c|c|c|}
\hline
\multicolumn{1}{|l|}{} & \multicolumn{1}{l|}{        Barbiero} & \multicolumn{1}{l|}{          Beggiato} & \multicolumn{1}{l|}{              Freo} & \multicolumn{1}{l|}{               Grosselle} & \multicolumn{1}{l|}{          Scarpa} & \multicolumn{1}{l|}{  Scortecagna} \\ \hline
Responsabile & 330 & 330 & 330 & 285 & 270 & 330 \\ \hline
Amministratore & 270 & 270 & 290 & 280 & 280 & 260 \\ \hline
Analista & 500 & 500 & 500 & 500 & 500 & 500 \\ \hline
Progettista & 473 & 484 & 462 & 484 & 473 & 484 \\ \hline
Programmatore & 304 & 288 & 296 & 288 & 296 & 288 \\ \hline
Verificatore & 320 & 320 & 320 & 320 & 320 & 320 \\ \hline
Totale & 2197 & 2192 & 2198 & 2157 & 2139 & 2182 \\ \hline
\end{tabular}

\modifiche
\end{document}
