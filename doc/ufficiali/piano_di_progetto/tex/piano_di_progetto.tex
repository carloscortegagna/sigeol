\documentclass[11pt,a4paper]{article}
\usepackage{amsmath}
\usepackage{amsfonts}
\usepackage{amssymb}
\usepackage{fancyhdr}
\usepackage{lastpage}
\usepackage{graphicx}
\usepackage{ucs}
\usepackage[utf8x]{inputenc}
\usepackage[italian]{babel}
\usepackage{eurosym}

% \renewcommand{\headrulewidth}{0.6pt}
\renewcommand{\footrulewidth}{0.6pt}
% impostazione dello stile per le pagine interne del documento
\lhead{\leftmark}
\chead{}
\rhead{\includegraphics[scale=0.15]{logo.png} }
\lfoot{Piano di Progetto v0.5.0}
\cfoot{}
\rfoot{\thepage \ di \pageref{LastPage}}
% ridefinizione dello stile plain per il frontespizio
\fancypagestyle{plain}{
\fancyhf
}
% impostazione dello stile per l'indice
\fancypagestyle{indice}{
\lhead{\leftmark}
\chead{}
\rhead{\includegraphics[scale=0.15]{logo.png}}
\lfoot{Piano di Progetto v0.5.0}
\cfoot{}
\rfoot{}
}
\headheight = 46pt
%definizione del comando "\modfiche" per la creazione del diario delle modifiche
\newcommand{\modifiche} 
{
\newpage
\begin{center}
\textbf{Diario delle modifiche} \\
\bigskip
\begin{tabular}{|c|c|p{0.62\textwidth}|}
\hline
\textsc{Data} & \textsc{Versione} & \textsc{Modifica} \\
\hline
\hline
\textit{07-12-2008} & 0.5.0 & Aggiunti i riferimenti e le fasi \\
\hline
\textit{07-12-2008} & 0.4.1 & Correzioni varie \\
\hline
\textit{06-12-2008} & 0.4.0 & Inserimento delle revisioni \\
\hline
\textit{05-12-2008} & 0.3.0 & Inserimento del modello di ciclo di vita e dei ruoli \\
\hline
\textit{02-12-2008} & 0.2.0 & Inserimento tabelle e Gantt \\
\hline
\textit{29-11-2008} & 0.1.0 & Prima bozza del documento \\
\hline
\end{tabular}
\end{center}
}
%definizione del comando "\info" per la creazione delle informazioni del documento
\newcommand{\info} {
\bigskip
\begin{tabbing}
	\hspace*{0.3\textwidth} \= \hspace*{0.5\textwidth} \kill
	\parbox{0.3\textwidth}{\textbf{Verifica: }} \> \parbox{0.5\textwidth}{Freo Matteo} \\
	\parbox{0.3\textwidth}{\textbf{Approvazione: }} \> \parbox{0.5\textwidth}{Grosselle Alessandro} \\
	\parbox{0.3\textwidth}{\textbf{Stato: }} \> \parbox{0.5\textwidth}{Preliminare} \\
	\parbox{0.3\textwidth}{\textbf{Uso: }} \> \parbox{0.5\textwidth}{Esterno} \\
	\parbox{0.3\textwidth}{\textbf{Distribuzione: }} \> \parbox{0.5\textwidth}{QuiXoft} \\
	\> \parbox{0.5\textwidth}{Rossi Francesca} \\
	\> \parbox{0.5\textwidth}{Vardanega Tullio} \\
	\> \parbox{0.5\textwidth}{Conte Renato} \\
\end{tabbing}
}
%definizione del comando "\frontespizio" per la creazione del frontespizio
\newcommand{\frontespizio} {
\thispagestyle{plain}
\title{\begin{Huge}\textsc{Progetto SIGEOL}\end{Huge} \\ \textit{Piano di Progetto \\ v0.5.0}}
\author{Redazione: Scortegagna Carlo}
\maketitle
\medskip
\begin{center}
\includegraphics[scale=0.5]{logo.png} \\
\textit{quixoft.sol@gmail.com}
\end{center}
\medskip
\info
\begin{center}
\textbf{Sommario} \\
Documento contenente il piano di progetto \textit{SIGEOL}, commissionato dalla prof. Rossi Francesca.
\end{center}
\newpage
}
%definizione del comando "\indice" per la creazione dell'indice
\newcommand{\indice} {
\thispagestyle{indice}
\tableofcontents
\newpage
}
\pagestyle{fancy}
\begin{document}
\frontespizio
\indice
\setcounter{page}{1}
\section{Introduzione}
\subsection{Scopo del documento}
Questo documento espone la pianificazione dello sviluppo del progetto denominato ''SIGEOL''.
Al suo interno è possibile trovare una stima delle tempistiche e dei costi del progetto, il tutto illustrato attraverso l'utilizzo di tabelle e del diagramma di Gantt.

Si mostrerà anche come è stata pianificata la rotazione dei ruoli per ogni fase del progetto e la ripartizione del carico di lavoro individuale.
Il piano si evolve di pari passo col progetto e inizialmente il documento conterrà solamente previsioni e stime.

Data la scarsa esperienza del team QuiXoft non si esclude la possibilità che il piano di progetto potrà essere rivisto e rimodellato in corso d'opera.
\subsection{Scopo del prodotto}
Il prodotto ''SIGEOL'' si prefigge di automatizzare la generazione, la gestione, l'ottimizzazione e la consultazione degli orari di lezione. 

Per ulteriori informazioni riguardanti scopi e funzioni del prodotto si prega di fare riferimento al documento \textsc{Analisi dei Requisiti}.
\subsection{Riferimenti}
\begin{itemize}
 	\item Il ciclo di vita del software \\ \textit{http://www.math.unipd.it/~tullio/IS-1/2008/Dispense/P04.pdf}
 	\item Gantt \\ \textit{http://gates.comm.virginia.edu/rrn2n/teaching/gantt.htm}
	\item GNU General Public License 2.0 \\ \textit{http://www.gnu.org/licenses/gpl-2.0.html}
	\item \textsc{Norme di Progetto}
\end{itemize}
\subsection{Glossario} 
Le definizioni dei termini specialistici usati nella stesura di questo e di tutti gli altri documenti possono essere trovate nel documento \textsc{Glossario} al fine di eliminare ogni ambiguità e di facilitare la comprensione dei temi trattati. Ogni termine la cui definizione è disponibile all'interno del glossario verrà marcato con una \underline{sottolineatura}.
\section{Panoramica di progetto}
\subsection{Vincoli}
Il progetto, per essere considerato accettabile dal committente, dovrà tassativamente rispettare i seguenti vincoli.
\subsubsection{Equa ripartizione del carico di lavoro}
Il piano di progetto prevede un'equa ripartizione del carico di lavoro tra tutti i componenti del team.
In linea di massima, le ore fatte in corso d'attività dovranno essere paragonabili a quelle attribuite a ciascun componente del gruppo nel piano.

Il piano di progetto di fase in fase verrà aggiornato inserendo le ore consuntive di ogni ruolo.
E' stato pianificato che ogni componente avrà un carico di lavoro poco inferiore alle 105 ore. Per maggiori informazioni si prega di consultare la sezione 4.
\subsubsection{Budget Economico}
Il progetto dovrà affrontare una spesa pari ad almeno 13000\euro.
\subsubsection{Licenza}
Passata la revisione di accettazione il software prodotto verrà rilasciato con licenza libera GNU GPL versione 2.0.

Come ogni licenza di software libero, essa concede ai licenziatari (da qui in poi indicati come utenti) il permesso di modificare il programma, di copiarlo e di ridistribuirlo con o senza modifiche, gratuitamente o a pagamento.
\subsection{Componenti}
Il team QuiXoft è formato dai seguenti componenti:
\begin{itemize}
\item Barbiero Mattia
\item Beggiato Andrea
\item Freo Matteo
\item Grosselle Alessandro
\item Scarpa davide
\item Scortegagna Carlo 
\end{itemize}
Ogni componente soddisfa i vincoli di propedeuticità ed è quindi libero di partecipare alla revisione dei requisiti.
\subsection{Ruoli}
Il progetto prevede i seguenti ruoli:
\begin{itemize}
\item \textbf{Responsabile:} è colui che si fa carico, per conto del suo team, dei risultati del progetto. Elabora ed emana piani e scadenze, approva l'emissione di documenti, coordina le attività del gruppo e si relaziona con il controllo di qualità interno al progetto.
E' responsabile della pianificazione di progetto, coordinando le attività del gruppo.
\item \textbf{Amministratore:} è responsabile della redazione e attuazione di piani e procedure di gestione per la qualità.
Gestisce la documentazione di progetto e controlla versioni e configurazioni del prodotto.
\item \textbf{Analista:} ha grande impatto sul successo del progetto poichè è responsabile dell'attività di analisi.
\item \textbf{Progettista:} ha forte impatto sugli aspetti tecnici e tecnologici del progetto. E' responsabile delle attività di progettazione.
\item \textbf{Programmatore:} è responsabile delle attività di codifica miranti alla realizzazione del prodotto e delle componenti di ausilio necessarie per l'esecuzione delle prove di verifica e validazione.
\item \textbf{Verificatore:}  partecipa all'intero ciclo di vita ed è responsabile alla verifica dei processi e della validazione del prodotto.
\end{itemize}
\subsection{Costo dei ruoli}
Ogni ruolo all'interno del team QuiXoft ha un determinato costo orario. Nella tabella sottostante ne vengono riportati i dettagli:
\begin{center}
\begin{tabular}{|c|c|}
\hline
\textsc{Ruolo} & \textsc{\euro \ / ora}\\ \hline \hline
Responsabile & 30 \\ \hline
Amministratore & 20 \\ \hline
Analista & 25 \\ \hline
Progettista & 22 \\ \hline
Programmatore & 16 \\ \hline
Verificatore & 16 \\ \hline
\end{tabular}
\end{center}
\section{Pianificazione}
Verranno ora descritti gli aspetti riguardanti la pianificazione del progetto, come il tipo di ciclo di vita e le stime sulla durata delle fasi.
A metà di ogni fase è prevista tassativamente una rotazione dei ruoli.
E' possibile che qualche componente possa esercitare due ruoli contemporaneamente; è garantita la completa assenza di interessi tra i compiti assunti.

In questa sezione si parlerà di revisioni; Per maggiori dettagli su queste visionare la sezione 3.4.

Per adattare alcune tabelle al documento, i nominativi dei ruoli verranno così abbreviati:
\begin{itemize}
\item Responsabile: Res
\item Amministratore: Amm
\item Analista: An
\item Progettista: Prog
\item Programmatore: Progr
\item Verificatore: Ver
\end{itemize}
Per lo stesso motivo anche i nominativi delle risorse verranno così abbravviati:
\begin{itemize}
\item Beggiato Andrea: BA
\item Barbiero Mattia: BM
\item Freo Matteo: FM
\item Grosselle Alessandro: GA
\item Scarpa Davide: SD
\item Scortegagna Carlo: SC
\end{itemize}
\subsection{Modello di ciclo di vita}
Il modello scelto dal team QuiXoft è il modello a cascata.
Questo è una progressione sequenziale (in cascata) di fasi, senza ricicli, al fine di meglio controllare tempi e costi.
Con questo modello rigidamente sequienziale la pianificazione è molto precisa ma presenta uno spiacevole incoveniente: assume che i requisiti possano essere congelati alla fine della fase di analisi.

Data la poca esperienza del team QuiXoft è molto difficile che tutti i requisiti siano chiari alla fine dell'analisi.
Per questo motivo si è scelto di utilizzare una variante del modello a cascata prevedendo eventuali prototipazioni allo scopo di capire meglio i requisiti.
Oltretutto i committenti non escludono possibili variazioni alle richieste e quindi a maggior ragione il modello a cascata con protipazione risulta particolarmente adatto.
\subsection{Fasi}
Il modello a cascata prevede le seguenti fasi:
\begin{itemize}
\item Analisi
\item Progettazione
\item Realizzazione
\item Manutenzione
\end{itemize}
La fase di manutenzione non verrà tenuta dal team QuiXoft perchè il progetto terminerà alla fine della fase di realizzazione con l'accettazione del prodotto da parte del committente alla revisione di accettazione.
\subsubsection{Analisi}
In questa fase avviene la definizione di vincoli,funzioni e di qualsiasi altra caratteristica che il sistema dovrà soddisfare.
L'individuazione dei requisiti si basa sul capitolato d'appalto e su eventuali ''interviste'' al commitente.

La data d'inizio di tale fase è il 17/11/2008, data del primo incontro del team QuiXoft. Si prevede di terminare l'analisi il 09/01/09.
La rotazione è prevista per la data 9/12/2008. 
Verso metà di tale fase il team dovrà affrontare la revisione dei requisiti condotta dal committente. 

Per questa revisione è previsto il documento contenente la classificazione e discussione dei requisiti, chiamato \textsc{Analisi dei Requisiti} e una prima redazione del \textsc{Piano di Qualifica}. 

Si riporta ora la rotazione dei ruoli in questa fase; affianco ad ogni ruolo sono riportate tra parentesi le ore stimate da esercitare.
\\\\
\begin{tabular}{|c|c|c|}
\hline
\textsc{Componenti} & \textsc{Prima Parte} & \textsc{Seconda Parte} \\ \hline \hline
Barberio Mattia & An(20) & Res(11) \\ \hline
Beggiato Andrea & An(20) & Ver(10) \\ \hline
Freo Matteo & Ver(10) & An(20) \\ \hline
Grosselle Alessandro & Amm(13,5) & An(20) \\ \hline
Scarpa Davide & An(20) & Amm(13,5) \\ \hline
Scortegagna Carlo & Res(10) & An(20) \\ \hline
\end{tabular}
\subsubsection{Progettazione}
La fase di progettazione prevede inizialmente la definizione dell'architettura di sistema, cioè le componenti che lo caratterizzano e le relazioni tra queste(progettazione architetturale).
Successivamente vi sarà la definizione delle struttura interna di ciascun componente(progettazione di dettaglio).

La data di inizio progettazione è stimata intorno al 12/01/2009; la data di fine invece intorno al 13/02/2009.
la rotazione è prevista per il 27/01/2009.

Verso la fine della fase il team affronterà la prima revisione interna(Revisione del Progetto Preliminare)che mostrerà al committente una visione ad alto livello del sistema. Il team per tale scopo, dovrà produrre un documento chiamato specifica tecnica e dovrà aggiornare il \textsc{Piano di Qualifica}. Se la revisione andrà a buon fine si attiverà la fase realizzativa del prodotto.

Si riporta ora la rotazione dei ruoli in questa fase; affianco ad ogni ruolo sono riportate tra parentesi le ore stimate da esercitare.\\\\
\begin{tabular}{|c|c|c|}
\hline
\textsc{Componenti} & \textsc{Prima Parte} & \textsc{Seconda Parte} \\ \hline \hline
Barberio Mattia & Prog(21,5) & Ver(20) \\ \hline
Beggiato Andrea & Amm(13,5)/Ver(10) & Prog(22) \\ \hline
Freo Matteo & Prog(21) & Res(10,5) \\ \hline
Grosselle Alessandro & Ver(20) & Prog(22) \\ \hline
Scarpa Davide & Res(10) & Prog(21,5) \\ \hline
Scortegagna Carlo & Prog(22) & Amm(13,5) \\ \hline
\end{tabular}
\subsubsection{Realizzazione}
La fase di realizzazione ha lo scopo di implementare i vari componenti definiti nella progettazione e di definire ed eseguire 'casi di prova' sia per i singoli moduli che per l'intero sistema con l'intento di rilevare malfunzionamenti.
Ogni volta che il modulo di un componente vieni istanziato, questo viene collaudato e se l'esito è positivo viene integrato al componente altrimenti verrà segnalato al programmatore il malfunzionamento. Per la gestione dei malfunzionamenti vedere il documento \textsc{Piano di Qualifica}.
Non appena tutte le componenti saranno pronte, verranno a loro volta integrate tra loro e si avvierà la fase di collaudo dell'intero sistema.  
La data di inizio realizzazione è stimata intorno al 16/02/2009; la data di fine invece intorno al 24/03/2009.
la rotazione è prevista per il 02/03/2009.

A tre quarti fase è prevista la revisione di qualifica che ha lo scopo di approvare l'esito finale della verifica. In questa fase si dovrà portare la versione definitiva del \textsc{Piano di Qualifica} e la versione iniziale del \textsc{Manuale Utente}.

A fine fase ci sarà la revisione di accettazione in cui vi sarà il collaudo del sistema da parte del committente e l'accertamento di soddisfacimento di tutti i requisiti previsti. Se la revisione andrà a buon fine il prodotto verrà accettato e il progetto terminato.    

Si riporta ora la rotazione dei ruoli in questa fase; affianco ad ogni ruolo sono riportate tra parentesi le ore stimate da esercitare.\\\\
\begin{tabular}{|c|c|c|}
\hline
\textsc{Componenti} & \textsc{Prima Parte} & \textsc{Seconda Parte} \\ \hline \hline
Barberio Mattia & Progr(18) & Amm(14) \\ \hline
Beggiato Andrea & Progr(18) & Res(11) \\ \hline
Freo Matteo & Amm(14,5) & Progr(18,5)/Ver(10) \\ \hline
Grosselle Alessandro & Res(10) & Progr(18) \\ \hline
Scarpa Davide & Progr(19,5) & Ver(20) \\ \hline
Scortegagna Carlo & Ver(20) & Progr(18) \\ \hline
\end{tabular}
\subsection{Panoramica delle revisioni}
Il team QuiXoft dovrà sostenere quattro revisioni; la prima e l'ultima sono esterne e saranno condotte del cliente. Le altre due sono interne al team, con il coinvolgimento del committente.

Lo scopo di una revisione interna è quello di mostrare il progresso del progetto al cliente, determinando se esistono, nuovi bisogni e correzioni da fare. Questo tipo di revisione non ha effetto sanzionatorio.

In una revisione esterna il cliente esaminerà e valuterà tutte le attività svolte dal team in quella determinata fase evidenziando, se esistono, le problemiche trovate. Questo tipo di revisione ha effetto sanzionatorio. 

Le revisioni sono le seguenti:
\begin{itemize}
\item Revisione dei requisiti(RR)
\item Revisione del progetto preliminare(RPP)
\item Revisione di qualifica(RQ)
\item Revisione di accettazione(RA)
\end{itemize}
\subsubsection{Revisione dei requisiti(RR)}
In questa revisione vi sarà una discussione tra cliente e fornitore dei requisiti del sistema, allo scopo di verificare se entrambe le parti condividono una stessa visione generale del prodotto.
La fine della fase di analisi non coincide con la revisione dei requisiti; per questo motivo tutti i prodotti portati a questa revisione sono da considerarsi parziali.

Per questa fase il team dovrà produrre tre documenti:
\begin{itemize}
\item \textsc{Analisi dei Requisiti}: mostra la definizione e classificazione dei requisiti. Il file in formato pdf del documento è nominato analisi\_dei\_requisiti.pdf. 
\item \textsc{Piano di Qualifica}: delinea la strategia generale di verifica e validazione. Il file in formato pdf del documento è nominato piano\_di\_qualifica.pdf.  
\item Piano di progetto: mostra la pianificazione del progetto(vedi scopo del documento, sezione 1). Il file in formato pdf del documento è piano\_di\_progetto.pdf. 
\end{itemize}
\subsubsection{Revisione del progetto preliminare(RPP)}
Il fornitore in questa fase mostrerà al cliente una visione ad alto livello del sistema attraverso l'uso di diagrammi delle classi e altre rappresentazioni architetturali.
Se la revisione andrà a buon fine si attiverà la fase realizzativa del prodotto.
Oltre a questo, il fornitore dovrà mostrare l'esistenza di strategie,tecnologie adeguate per l'implementazione del progetto.

Per questa fase si dovrà aggiornare il \textsc{Piano di Qualifica}, il piano di progetto e produrre un nuovo documento:
\begin{itemize}
\item Specifica tecnica: presenta l'archittetura generale del sistema identificando e descrivendo le sue componenti di alto livello. Il file in formato pdf del documento è specifica\_tecnica.pdf. 
\end{itemize}
\subsubsection{Revisione di qualifica(RQ)}
La funzione di questa revisione è approvare l'esito finale della verifica.

Per questa fase si dovrà aggiornare il \textsc{Piano di Qualifica}, inserendo il resoconto definitivo della campagna di verifica e la descrizione delle prove proposte per il collaudo. Si dovrà poi produrre in modo parziale un nuovo documento:
\begin{itemize}
\item \textsc{Manuale Utente}: Descrive le istruzioni per l'uso del sistema. Il file in formato pdf del documento è manuale\_utente.pdf.
\end{itemize}
\subsubsection{Revisione di accettazione(RA)}
In questa revisione il sistema verrà collaudato dal committente e di conseguenza accettato o meno. 
Inoltre vi sarà anche l'accertamento di soddisfacimento di tutti i requisiti pattuiti.

Per questa fase si dovranno aggiornare in modo definitivo il \textsc{Piano di Qualifica} e il \textsc{Manuale Utente}.
Se la revisione andrà a buon fine il prodetto verrà accettato e il progetto sarà terminato.
\subsection{Riassunto delle ore}
Viene ora riportata una tabella che mostra la suddivisione del carico di lavoro(espresso in ore) ad ogni componente del team.
Si è cercato di suddividere il carico nel modo più equo possibile, stando attenti a non andare oltre le 105 ore a persona.
\\\\
\begin{tabular}{|c|c|c|c|c|c|c|}
\hline
\textsc{Ruoli} & \textsc{BM} & \textsc{BA} & \textsc{FM} & \textsc{GA} & \textsc{SD} & \textsc{SC} \\ \hline \hline
Responsabile & 11 & 11 & 10,5 & 10 & 10 & 10 \\ \hline
Amministratore & 14 & 13,5 & 14,5 & 13,5 & 13,5 & 13,5 \\ \hline
Analista & 20 & 20 & 20 & 20 & 20 & 20 \\ \hline
Progettista & 21,5 & 22 & 21 & 22 & 21,5 & 22 \\ \hline
Progettista & 21,5 & 22 & 21 & 22 & 21,5 & 22 \\ \hline
Verificatore & 20 & 20 & 20 & 20 & 20 & 20 \\ \hline
Progettista & 21,5 & 22 & 21 & 22 & 21,5 & 22 \\ \hline
Totale & 104,5 & 104,5 & 104,5 & 103,5 & 104,5 & 103,5 \\ \hline
\end{tabular}
\subsection{Diagramma di Gantt}
Assieme a questo documento viene allegato il file diagramma\_di\_gantt.png che mostra la pianificazione delle attività per ogni fase. 

\section{Analisi dei rischi}
\subsection{Assenza componente per medio/lungo periodo}
Ogni componente ha sempre un determinato ruolo in tutte le fasi del progetto.
Se un componente risultasse indisponibile per un lasso di tempo più o meno lungo causerebbe un rallentamento del progetto se non addirittura uno stallo.
E' compito dell'amministratore riassegnare il determinato ruolo ad uno o più componenti del team e modificare il piano di progetto cercando di mantenere inalterati i costi e la data di fine progetto.
\subsection{Mancanza di conoscenze tecniche}
Si utilizzeranno strumenti e tecnologie che per alcuni componenti risulteranno nuove.
L'amministratore metterà a disposizione guide e manuali per poter formare il team. Lo studio è personale e non è previsto nel piano di progetto. Tuttavia se dovesse essere necessario verrà fatta una riunione che delinearà i concetti di massima dell'eventuale strumento o tecnologia.
\subsection{Calendario delle attività inneficiente}
Le attività, vista la scarsa esperienza, sono state pianificate basandosi su precedenti calendari di altri gruppi.
è compito dell'amministratore correggere eventuali errori di pianificazione, cercando di mantenere inalterati i costi e la data di fine progetto.
\subsection{Analisi dei requisiti inefficiente}
Nella prima fase avviene lo studio approfondito dei requisiti. Finita questa fase, inizia quella di progettazione. Data la scarsa esperienza del team QuiXoft il rischio di variazione dei requisiti dopo il loro studio esiste. Tuttavia si cerca di renderlo il più basso possibile ruotando il ruolo di analista e rendendo la ricerca dei requisiti il più completa ed efficiente possibile. 
\subsection{Gestione della qualità inadeguata}
L'accertamento della qualità del processo e del prodotto è garantita dalla presenza del ruolo del Verificatore per tutto il periodo di concezione e sviluppo del progetto. Se le ore necessarie al verificatore in una determinata fase dovessero essere troppo poche, verrà immediatamente avvisato l'amministratore che modificherà il piano di progetto, aumentando le ore al verificatore. 
\section{Conto economico preventivo}
\subsection{Costi stimati nel progetto per ruolo}
Dalla precedente pianificazione si può redigere la seguente tabella che mostra i costi stimati(in \euro) di ogni ruolo.
Il costo del progetto è stato stimato intorno ai 13065\euro, in pieno rispetto al vincolo ''Budget economico''.
\\\\
\begin{tabular}{|c|c|c|c|c|}
\hline
\textsc{Componenti}& \textsc{Analisi} & \textsc{Progettazione} & \textsc{Realizzazione} & \textsc{Tot.} \\ \hline \hline
Responsabile & 630 & 615 & 630 & 1875 \\ \hline
Amministratore & 540 & 540 & 570 & 1650 \\ \hline
Analista & 3000 & 0 & 0 & 3000 \\ \hline
Progettista & 0 & 2860 & 0 & 2860 \\ \hline
Programmatore & 0 & 0 & 1760 & 1760 \\ \hline
Verificatore & 320 & 800 & 800 & 1920 \\ \hline
Tot. & 4490 & 4815 & 3760 & 13065 \\ \hline
\end{tabular}
\subsection{Costi stimati nel progetto per risorsa}
La tabella sottostante rappresenta i costi(in \euro) di ogni componente del team in relazione al ruolo ricoperto.
Per maggiori informazioni sulle abbreviazioni vedere la sezione 3.\\\\
\begin{tabular}{|c|c|c|c|c|c|c|}
\hline
\textsc{Ruoli} & \textsc{BM} & \textsc{BA} & \textsc{FM} & \textsc{GA} & \textsc{SD} & \textsc{SC} \\ \hline \hline
Analista & 500 & 500 & 500 & 500 & 500 & 500 \\ \hline
Verificatore & 320 & 320 & 320 & 320 & 320 & 320 \\ \hline
Responsabile & 330 & 330 & 315 & 300 & 300 & 300 \\ \hline
Amministratore & 280 & 270 & 290 & 270 & 270 & 270 \\ \hline
Progettista & 473 & 484 & 462 & 484 & 473 & 484 \\ \hline
Programmatore & 288 & 288 & 296 & 288 & 312 & 288 \\ \hline
Totale & 2191 & 2192 & 2183 & 2162 & 2175 & 2162 \\ \hline
\end{tabular}
\modifiche
\end{document}