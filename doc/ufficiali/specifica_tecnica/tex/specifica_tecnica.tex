\documentclass[11pt,a4paper]{article}
\usepackage{amsmath}
\usepackage{amsfonts}
\usepackage{amssymb}
\usepackage{fancyhdr}
\usepackage{lastpage}
\usepackage{graphicx}
\usepackage{ucs}
\usepackage[utf8x]{inputenc}
\usepackage[italian]{babel}

\renewcommand{\headrulewidth}{0.6pt}
\renewcommand{\footrulewidth}{0.6pt}
% impostazione dello stile per le pagine interne del documento
\lhead{\leftmark}
\chead{}
\rhead{\includegraphics[scale=0.15]{logo.png} }
\lfoot{Specifica tecnica v0.0.1}
\cfoot{}
\rfoot{\thepage \ di \pageref{LastPage}}
% ridefinizione dello stile plain per il frontespizio
\fancypagestyle{plain}{
\fancyhf
}
% impostazione dello stile per l'indice
\fancypagestyle{indice}{
\lhead{\leftmark}
\chead{}
\rhead{\includegraphics[scale=0.15]{logo.png}}
\lfoot{Specifica tecnica v0.0.1}
\cfoot{}
\rfoot{}
}
\headheight = 46pt
%definizione del comando "\modifiche" per la creazione del diario delle modifiche
\newcommand{\modifiche} 
{
\newpage
\begin{center}
\textbf{Diario delle modifiche} \\
\bigskip
\begin{tabular}{|c|c|p{0.51\textwidth}|}
\hline
\textsc{Data} & \textsc{Versione} & \textsc{Modifica} \\
\hline
\hline
\textit{18 gennaio 2009} & 0.0.1 & Creazione dell'indice \\
\hline
\end{tabular}
\end{center}
}
%definizione del comando "\info" per la creazione delle informazioni del documento
\newcommand{\info} {
\bigskip
\begin{tabbing}
	\hspace*{0.3\textwidth} \= \hspace*{0.5\textwidth} \kill
	\parbox{0.3\textwidth}{\textbf{Verifica: }} \> \parbox{0.5\textwidth}{Freo Matteo} \\
	\parbox{0.3\textwidth}{\textbf{Approvazione: }} \> \parbox{0.5\textwidth}{Scortegagna Carlo} \\
	\parbox{0.3\textwidth}{\textbf{Stato: }} \> \parbox{0.5\textwidth}{Formale} \\
	\parbox{0.3\textwidth}{\textbf{Uso: }} \> \parbox{0.5\textwidth}{Esterno} \\
	\parbox{0.3\textwidth}{\textbf{Distribuzione: }} \> \parbox{0.5\textwidth}{QuiXoft} \\
							\> \parbox{0.5\textwidth}{Rossi Francesca} \\
							\> \parbox{0.5\textwidth}{Vardanega Tullio} \\
							\> \parbox{0.5\textwidth}{Conte Renato} \\
\end{tabbing}
}
%definizione del comando "\frontespizio" per la creazione del frontespizio
\newcommand{\frontespizio} {
\thispagestyle{plain}
\title{\begin{Huge}\textsc{Progetto SIGEOL}\end{Huge} \\ \textit{Specifica Tecnica \\ v0.0.1}}
\author{Redazione: }
\maketitle
\medskip
\begin{center}
\includegraphics[scale=0.5]{logo.png} \\
\textit{quixoft.sol@gmail.com}
\end{center}
\medskip
\info
\begin{center}
\textbf{Sommario} \\
Documento contenente la specifica tecnica per il progetto ''SIGEOL'' commissionato dalla prof. Rossi Francesca.
\end{center}
\newpage
}
%definizione del comando "\indice" per la creazione dell'indice
\newcommand{\indice} {
\thispagestyle{indice}
\tableofcontents
\newpage
}
\pagestyle{fancy}
\begin{document}
\frontespizio
\indice
\setcounter{page}{1}
\section{Introduzione}
\subsection{Scopo del documento}
Il presente documento denominato \textsc{Specifica tecnica} ha lo scopo di mostrare la struttura del progetto \textit{SIGEOL} e descrivere i componenti che fanno parte. 
\subsection{Scopo del prodotto}
Il progetto sotto analisi, denominato \textit{SIGEOL}, si prefigge di automatizzare la generazione, la gestione, l'ottimizzazione e la consultazione degli orari di lezione. 
\subsection{Glossario}
Le definizioni dei termini specialistici usati nella stesura di questo e di tutti gli altri documenti possono essere trovate nel documento \textsc{Glossario} al fine di eliminare ogni ambiguità e di facilitare la comprensione dei temi trattati. Ogni termine la cui definizione è disponibile all’interno del Glossario verrà marcato con una \underline{sottolineatura}.

\section{Definizione del prodotto}
\subsection{Metodo e formalismo di specifica}
Lo strumento principale nel redarre la specifica tecnica sarà il linguaggio \underline{UML}, che permetterà di realizzare i diagrammi delle classi, di sequenza, di collaborazione e di attivita'.


La decomposizione architetturale utilizzata sara' di tipo \underline{Top-down}.
E' prevista una descrizione generale dell’architettura del sistema, alla quale seguiranno le specifiche dettagliate dei suoi componenti.


Per semplificare la progettazione, si utilizzaranno i seguenti pattern:
\begin{itemize}
 \item MVC: si basa sulla separazione tra i componenti software del sistema, che gestiscono il modo in cui presentare i dati e i componenti che gestiscono i dati stessi. 
 \item Facade:permette attraverso un'interfaccia più semplice, l'accesso a sottosistemi che espongono interfacce complesse e molto diverse tra loro, nonché a blocchi di codice complessi.
\end{itemize}


Il sistema verrà implementato utilizzando Ruby on rails, un framework la cui architettura è fortemente ispirata al paradigma Model-View-Controller. 
\subsection{Presentazione dell'architettura generele del sistema}
Per presentare l'architettura generale del sistema \textit{Sigeol} si utilizzerà il pattern Model-View-Controller(MVC).


\section{Descrizione dei singoli componenti}
\subsection{Obiettivo,struttura e funzione dei componenti}
\subsection{Relazioni d'uso di altre componenti}
\subsection{Interfacce con e relazioni di uso da altre componenti}
\subsection{Attività svolte e dati trattati}
\section{Stime di fattibilità e di bisogno di risorse}
\section{Tracciamento della relazione componenti-requisiti}
\modifiche
\end{document}
