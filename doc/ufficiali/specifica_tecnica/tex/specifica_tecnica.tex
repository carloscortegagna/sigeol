\documentclass[11pt,a4paper]{article}
\usepackage{amsmath}
\usepackage{amsfonts}
\usepackage{amssymb}
\usepackage{fancyhdr}
\usepackage{lastpage}
\usepackage{graphicx}
\usepackage{ucs}
\usepackage[utf8x]{inputenc}
\usepackage[italian]{babel}
\usepackage[colorlinks=true,linkcolor=black]{hyperref}

\renewcommand{\headrulewidth}{0.6pt}
\renewcommand{\footrulewidth}{0.6pt}
% impostazione dello stile per le pagine interne del documento
\lhead{\leftmark}
\chead{}
\rhead{\includegraphics[scale=0.15]{logo.png} }
\lfoot{Specifica tecnica v0.0.1}
\cfoot{}
\rfoot{\thepage \ di \pageref{LastPage}}
% ridefinizione dello stile plain per il frontespizio
\fancypagestyle{plain}{
\fancyhf
}
% impostazione dello stile per l'indice
\fancypagestyle{indice}{
\lhead{\leftmark}
\chead{}
\rhead{\includegraphics[scale=0.15]{logo.png}}
\lfoot{Specifica tecnica v0.0.1}
\cfoot{}
\rfoot{}
}
\headheight = 46pt
%definizione del comando "\modifiche" per la creazione del diario delle modifiche
\newcommand{\modifiche} 
{
\newpage
\begin{center}
\textbf{Diario delle modifiche} \\
\bigskip
\begin{tabular}{|c|c|p{0.51\textwidth}|}
\hline
\textsc{Data} & \textsc{Versione} & \textsc{Modifica} \\
\hline
\hline
\textit{20 gennaio 2009} & 0.1.0 & Creazione dell'indice \\
\hline
\end{tabular}
\end{center}
}
%definizione del comando "\info" per la creazione delle informazioni del documento
\newcommand{\info} {
\bigskip
\begin{tabbing}
	\hspace*{0.3\textwidth} \= \hspace*{0.5\textwidth} \kill
	\parbox{0.3\textwidth}{\textbf{Verifica: }} \> \parbox{0.5\textwidth}{Freo Matteo} \\
	\parbox{0.3\textwidth}{\textbf{Approvazione: }} \> \parbox{0.5\textwidth}{Scortegagna Carlo} \\
	\parbox{0.3\textwidth}{\textbf{Stato: }} \> \parbox{0.5\textwidth}{Formale} \\
	\parbox{0.3\textwidth}{\textbf{Uso: }} \> \parbox{0.5\textwidth}{Esterno} \\
	\parbox{0.3\textwidth}{\textbf{Distribuzione: }} \> \parbox{0.5\textwidth}{QuiXoft} \\
							\> \parbox{0.5\textwidth}{Rossi Francesca} \\
							\> \parbox{0.5\textwidth}{Vardanega Tullio} \\
							\> \parbox{0.5\textwidth}{Conte Renato} \\
\end{tabbing}
}
%definizione del comando "\frontespizio" per la creazione del frontespizio
\newcommand{\frontespizio} {
\thispagestyle{plain}
\title{\begin{Huge}\textsc{Progetto SIGEOL}\end{Huge} \\ \textit{Specifica Tecnica \\ v0.1.0}}
\author{Redazione: }
\maketitle
\medskip
\begin{center}
\includegraphics[scale=0.5]{logo.png} \\
\textit{quixoft.sol@gmail.com}
\end{center}
\medskip
\info
\begin{center}
\textbf{Sommario} \\
Documento contenente la specifica tecnica per il progetto ''SIGEOL'' commissionato dalla prof. Rossi Francesca.
\end{center}
\newpage
}
%definizione del comando "\indice" per la creazione dell'indice
\newcommand{\indice} {
\thispagestyle{indice}
\tableofcontents
\newpage
}
\pagestyle{fancy}
\begin{document}
\frontespizio
\indice
\setcounter{page}{1}
\section{Introduzione}
\subsection{Scopo del documento}
Il presente documento denominato \textsc{Specifica tecnica} ha lo scopo di mostrare la struttura del progetto \textit{SIGEOL} e descrivere i componenti che fanno parte. 
\subsection{Scopo del prodotto}
Il progetto sotto analisi, denominato \textit{SIGEOL}, si prefigge di automatizzare la generazione, la gestione, l'ottimizzazione e la consultazione degli orari di lezione. 
\subsection{Glossario}
Le definizioni dei termini specialistici usati nella stesura di questo e di tutti gli altri documenti possono essere trovate nel documento \textsc{Glossario} al fine di eliminare ogni ambiguità e di facilitare la comprensione dei temi trattati. Ogni termine la cui definizione è disponibile all’interno del Glossario verrà marcato con una \underline{sottolineatura}.

\section{Definizione del prodotto}
\subsection{Metodo e formalismo di specifica}
Lo strumento principale nel redarre la specifica tecnica sarà il linguaggio \underline{UML}, che permetterà di realizzare i diagrammi delle classi, di sequenza, di collaborazione e di attivita'.


La decomposizione architetturale utilizzata sara' di tipo \underline{Top-down}.
E' prevista una descrizione generale dell’architettura del sistema, alla quale seguiranno le specifiche dettagliate dei suoi componenti.


Per semplificare la progettazione, si utilizzaranno i seguenti pattern:
\begin{description}
 \item[\textbf{MVC}] 
Il pattern MVC (Model View Controller)si basa sulla separazione tra i componenti software del sistema, che gestiscono il modo in cui presentare i dati e i componenti che gestiscono i dati stessi. 
 \item[\textbf{Facade}]
Permette attraverso un'interfaccia più semplice, l'accesso a sottosistemi che espongono interfacce complesse e molto diverse tra loro, nonché a blocchi di codice complessi.
\item[\textbf{Rest}]
Representational state transfer (REST) è un tipo di architettura software per i sistemi di ipertesto distribuiti come il World Wide Web. 
REST si riferisce ad un insieme di principi di architetture di rete, i quali delineano come le risorse sono definite e indirizzate.


\item[\textbf{Convention Over Configuration}]
Convention Over Configuration è un paradigma di programmazione che prevede configurazione minima (o addirittura assente) per il programmatore che utilizza un framework che lo rispetti, obbligandolo a configurare solo gli aspetti che si differenziano dalle implementazioni standard o che non rispettano particolari convenzioni di denominazione o simili.
\item[\textbf{DRY}]
Le definizioni devono essere poste una volta soltanto
\item[\textbf{View Helper}]
Questo pattern disaccoppia il Business Logic dallo strato View, il che facilita
la manutenibilit`a. Aiuta a separare, in fase di sviluppo, la responsabilt`a
del web designer e dello sviluppatore.
\end{description}


Il sistema verrà implementato utilizzando Ruby on rails, un framework la cui architettura è fortemente ispirata al paradigma Model-View-Controller. 
Oltre a veicolare lo sviluppo di applicazioni secondo il pattern MVC, il RESTful Rails impone un’ulteriore disciplina nella codifica che garantisce maggiore compattezza, migliore leggibilità, “pretty-urling” e semplicità nella costruzione di API.

\subsection{Presentazione dell'architettura generale del sistema}
Per presentare l'architettura generale del sistema \textit{Sigeol} si utilizzerà il pattern Model-View-Controller(MVC).
In questo modello i ruoli di presentazione, controllo ed accesso ai dati vengono affidati a componenti diversi e sono tra di loro disaccoppiati.


\begin{center}
 \textbf{View}
\end{center}
E' il primo livello che si incontra e contiene i componenti che costituiscono l'interfaccia grafica, tramite la quale l'utente interagisce con il sistema \textit{Sigeol}.
La View delega al Controller l'esecuzione dei processi richiesti dall'utente dopo averne catturato gli input e la
scelta delle eventuali schermate da presentare.
ActionView gestisce l'aspetto delle pagine da restituire al client. Sono per lo piu' file .rhtml che contengono delle direttive scritte in Ruby immerse nel codice XHTML.

\begin{center}
 \textbf{Model}
\end{center}
Definisce i dati e le operazioni che possono essere eseguite su questi. Quindi definisce le regole di business per l'interazione con i dati, esponendo alla View ed al Controller rispettivamente le funzionalità per l'accesso e l'aggiornamento.
Viene gestito da una libreria chiamata ActiveRecord, che si occupa di tutta la logica di business e che permette l'accesso a numerosi DataBase bastati su SQL. Con questa libreria si stabilisce un collegamento tra le classi scritte in Ruby e le tabelle del DB.

\begin{center}
 \textbf{Controller}
\end{center}
Questo componente ha la responsabilità di trasformare le interazioni dell'utente della View in azioni eseguite dal Model. Ma il Controller non rappresenta un semplice "ponte" tra View e Model. Realizzando la mappatura tra input dell'utente e processi eseguiti dal Model e selezionando le schermate della View richieste, il Controller implementa la logica di controllo dell'applicazione.
I Controller sono rappresentati dalla libreria ActionController.
Gestisce le richieste del browser e facilita la comunicazione fra Model e View. Tutti i controller creati dall'utente ereditano da ActionController. L'ApplicationController raccoglie funzionalita' condivise nell'intera applicazione.
In particolare verranno inserite le azioni CRUD (Create-Read-Update-Delete) per l'oggetto e le varie funzioni che possono servirci per la nostra applicazione.


\includegraphics[scale=0.50]{images/SIGEOL.png}


\section{Descrizione dei singoli componenti}
\subsection{Obiettivo,struttura e funzione dei componenti}
\subsection{Relazioni d'uso di altre componenti}
\subsection{Interfacce con e relazioni di uso da altre componenti}
\subsection{Attività svolte e dati trattati}
\section{Stime di fattibilità e di bisogno di risorse}
\section{Tracciamento della relazione componenti-requisiti}
\modifiche
\end{document}
