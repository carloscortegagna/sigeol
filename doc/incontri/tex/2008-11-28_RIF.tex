\documentclass[11pt,a4paper]{article}
\usepackage{amsmath}
\usepackage{amsfonts}
\usepackage{amssymb}
\usepackage{fancyhdr}
\usepackage{lastpage}
\usepackage{graphicx}
\usepackage{ucs}
\usepackage[utf8x]{inputenc}
\usepackage[italian]{babel}

\renewcommand{\headrulewidth}{0.6pt}
\renewcommand{\footrulewidth}{0.6pt}
% impostazione dello stile per le pagine interne del documento
\fancypagestyle{plain} {
\lhead{\textit{28 novembre 2008}}
\chead{}
\rhead{\includegraphics[scale=0.15]{logo.png} }
\lfoot{Incontro interno facoltativo}
\cfoot{}
\rfoot{\thepage \ di \pageref{LastPage}}
}
\headheight = 46pt
\title{Resoconto dell'incontro interno facoltativo}
\author{Beggiato Andrea}
\date{28 novembre 2008}
\pagestyle{plain}
\begin{document}
\maketitle
\section{Scopo dell'incontro}
L'incontro facoltativo svoltosi con gli analisti ha avuto come scopo la ripartizione dei ruoli per la redazione del documento ufficiale denominato \textit{Analisi dei requisiti}. Si è tenuto inoltre un \textit{brainstorming} per delineare i requisiti che il sistema \textit{SIGEOL} dovrà rispettare.
\section{Membri presenti}
Sono risultati presenti, nel ruolo di analisti, Beggiato Andrea, Scarpa Davide e Barbiero Mattia.
\section{Resoconto dettagliato}
La prima questione che si è trattata concerne la struttura del documento \textit{Analisi dei requisiti} che è stata risolta con la redazione dell'indice ed a questo proposito è stata creata la versione 0.1.0 del documento sopra citato; successivamente si è aggiornato il repository.

Per quanto riguarda la ripartizione dei ruoli si è deciso all'unanimità quanto segue:
\begin{description}
 \item \textit{Beggiato Andrea} si occuperà della redazione dell'introduzione e della descrizione del prodotto;
 \item \textit{Scarpa Davide} si occuperà della redazione della parte riguardante i requisiti;
 \item \textit{Barbiero Mattia} si occuperà della redazione degli use cases.
 \end{description}
Inoltre si è deciso che al termine del proprio compito Beggiato Andrea e Scarpa Davide potranno assistere Barbiero Mattia a svolgere il proprio lavoro.
\end{document}