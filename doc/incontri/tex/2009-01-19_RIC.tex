\documentclass[11pt,a4paper]{article}
\usepackage{amsmath}
\usepackage{amsfonts}
\usepackage{amssymb}
\usepackage{fancyhdr}
\usepackage{lastpage}
\usepackage{graphicx}
\usepackage{ucs}
\usepackage[utf8x]{inputenc}
\usepackage[italian]{babel}

\renewcommand{\headrulewidth}{0.6pt}
\renewcommand{\footrulewidth}{0.6pt}
% impostazione dello stile per le pagine interne del documento
\fancypagestyle{plain} {
\lhead{\textit{19 gennaio 2009}}
\chead{}
\rhead{\includegraphics[scale=0.15]{logo.png} }
\lfoot{Incontro con il committente}
\cfoot{}
\rfoot{\thepage \ di \pageref{LastPage}}
}
\headheight = 46pt
\title{Resoconto dell'incontro con il committente}
\author{Scarpa Davide}
\date{19 gennaio 2009}
\pagestyle{plain}
\begin{document}
\maketitle
\section{Scopo dell'incontro}
In data 19 gennaio 2009, alle ore 10.30 si è svolto un incontro tra il Team QuiXoft e il committente con lo scopo di chiarire alcuni dubbi riguardanti il sistema SIGEOL.

Al termine si è poi svolto un incontro non previsto con il proponente, con lo scopo di discutere l'esito dell'incontro con il committente.
\section{Membri presenti}
In rappresentanza del team QuiXoft erano presenti all'incontro Beggiato Andrea, Freo Matteo, Grosselle Alessandro, Scarpa Davide e Scortegagna Carlo.

Era inoltre presente il committente prof.ssa Rossi Francesca.

Era presente al secondo incontro il proponente prof. Vardanega Tullio.
\section{Resoconto dettagliato}
La prima richiesta di chiarimento riguardava l'indipendenza o meno del sistema SIGEOL: il team QuiXoft ipotizzava la presenza di un database di dipartimento, contenente informazioni sui docenti quali username e password, informazioni sulle aule disponibili e sui corsi d'insegnamento. In tal caso il team avrebbe realizzato il sistema interfacciandosi al suddetto database, al fine di evitare duplicazioni di dati, inserimenti manuali non necessari e possibili inconsistenze.

Il committente ha risposto che il database ipotizzato non esiste, e le informazioni sopra citate sono memorizzate in un modo non utilizzabile per lo scopo del team.

La seconda richiesta di chiarimento riguardava le differenze tra le funzionalità dell'utente segreteria didattica e l'utente Presidente del CCS, suggerendo la possibilità di creare un utente amministratore distinto dai primi due.

Il committente ha risposto che non esistono differenze tra i due utenti, entrambi hanno le stesse funzioni da amministratore, pertanto non ne serve uno distinto.
Il committente ha inoltre posto l'accento sul fatto che anche se hanno le stesse funzionalità i sopra citati utenti devono rimanere separati.

Si è inoltre discusso della gestione dei vincoli dei docenti, giungendo alla conclusione che questi vengano accettati subito dal sistema, ma in caso opportuno la segreteria o il Presidente del CCS potranno rilassarne uno o più, mutandoli in preferenze.

Infine si è parlato del numero di ore settimanali di ogni corso d'insegnamento, che il sistema calcolerà automaticamente in base ai crediti del corso, lasciando comunque la possibilità di essere modificato.


Al secondo incontro si è discusso prevalentemente dell'indipendenza del sistema. Il proponente ha posto l'accento sul fatto che non esistendo il database ipotizzato dal team QuiXoft, quest'ultimo dovrà fare particolare attenzione a progettarne uno proprio in modo che possa venire facilmente utilizzato dal dipartimento in caso di necessità.
\end{document}
